%% Generated by Sphinx.
\def\sphinxdocclass{report}
\documentclass[letterpaper,10pt,english]{sphinxmanual}
\ifdefined\pdfpxdimen
   \let\sphinxpxdimen\pdfpxdimen\else\newdimen\sphinxpxdimen
\fi \sphinxpxdimen=.75bp\relax
\ifdefined\pdfimageresolution
    \pdfimageresolution= \numexpr \dimexpr1in\relax/\sphinxpxdimen\relax
\fi
%% let collapsible pdf bookmarks panel have high depth per default
\PassOptionsToPackage{bookmarksdepth=5}{hyperref}


\PassOptionsToPackage{warn}{textcomp}
\usepackage[utf8]{inputenc}
\ifdefined\DeclareUnicodeCharacter
% support both utf8 and utf8x syntaxes
  \ifdefined\DeclareUnicodeCharacterAsOptional
    \def\sphinxDUC#1{\DeclareUnicodeCharacter{"#1}}
  \else
    \let\sphinxDUC\DeclareUnicodeCharacter
  \fi
  \sphinxDUC{00A0}{\nobreakspace}
  \sphinxDUC{2500}{\sphinxunichar{2500}}
  \sphinxDUC{2502}{\sphinxunichar{2502}}
  \sphinxDUC{2514}{\sphinxunichar{2514}}
  \sphinxDUC{251C}{\sphinxunichar{251C}}
  \sphinxDUC{2572}{\textbackslash}
\fi
\usepackage{cmap}
\usepackage[T1]{fontenc}
\usepackage{amsmath,amssymb,amstext}
\usepackage{babel}



\usepackage{tgtermes}
\usepackage{tgheros}
\renewcommand{\ttdefault}{txtt}



\usepackage[Bjarne]{fncychap}
\usepackage{sphinx}

\fvset{fontsize=auto}
\usepackage{geometry}


% Include hyperref last.
\usepackage{hyperref}
% Fix anchor placement for figures with captions.
\usepackage{hypcap}% it must be loaded after hyperref.
% Set up styles of URL: it should be placed after hyperref.
\urlstyle{same}

\addto\captionsenglish{\renewcommand{\contentsname}{MISLAND System Admin}}

\usepackage{sphinxmessages}
\setcounter{tocdepth}{2}



\title{MISLAND DATA PREPARATION MANUAL}
\date{Feb 03, 2023}
\release{0.0.1}
\author{LocateIT Kenya Limited}
\newcommand{\sphinxlogo}{\vbox{}}
\renewcommand{\releasename}{Release}
\makeindex
\begin{document}

\ifdefined\shorthandoff
  \ifnum\catcode`\=\string=\active\shorthandoff{=}\fi
  \ifnum\catcode`\"=\active\shorthandoff{"}\fi
\fi

\pagestyle{empty}
\sphinxmaketitle
\pagestyle{plain}
\sphinxtableofcontents
\pagestyle{normal}
\phantomsection\label{\detokenize{index::doc}}


\begin{figure}[H]
\centering
\capstart

\noindent\sphinxincludegraphics[width=800\sphinxpxdimen,height=400\sphinxpxdimen]{{home}.png}
\caption{Misland Homepage}\label{\detokenize{index:id1}}\end{figure}

\sphinxAtStartPar
MISLAND Africa is an operational instrument relying on the international standards for reporting SDG 15.3.1 and technical approaches allowing the delivery of regular information on vegetation cover gain/loss to decision makers and environmental agencies at the first place.

\sphinxAtStartPar
The core\sphinxhyphen{}service provides land degradation indicators for five African Countries at two levels:
\begin{itemize}
\item {} 
\sphinxAtStartPar
At the regional level where low and medium resolution EO are used.

\item {} 
\sphinxAtStartPar
At the pilot site level, where(customized indicators) can be developed, using medium resoultion data(landsat time series imagery and derived vegetation indices, combined with different satellite\sphinxhyphen{}derived climate data)

\end{itemize}

\begin{sphinxadmonition}{note}{Note:}
\sphinxAtStartPar
You can download the \sphinxhref{https://mislanddata.readthedocs.io/\_/downloads/en/latest/pdf/}{PDF Version of this doucument} here.
\end{sphinxadmonition}

\sphinxstepscope


\chapter{MISLAND System Admin Panel}
\label{\detokenize{Introduction/admin:misland-system-admin-panel}}\label{\detokenize{Introduction/admin::doc}}
\sphinxAtStartPar
The following document is a system administrators manual that gives information on how to configure and manage the system. The backend of the system is built on django and therefore features the general characteristics of Django\sphinxhyphen{}admin settings. From the admin panel, the system administrators can manage dataset, users, and system resources thus ensuring that the computing resources are well utilized while still maintaining a high level of data security and integrity. The components of the system are described in details in the sections that follow.


\section{System Setting}
\label{\detokenize{Introduction/admin:system-setting}}
\sphinxAtStartPar
System settings can be accessed by going to \sphinxstyleemphasis{System Settings} and then click the item under \sphinxstyleemphasis{Email Host}. From here you can set several variables that are key to system resources management. One of the variables you can configure from here is the maximum size of a polygon that can be processed by the system on the fly.

\begin{figure}[H]
\centering
\capstart

\noindent\sphinxincludegraphics[width=750\sphinxpxdimen,height=577\sphinxpxdimen]{{admin1}.png}
\caption{Vegetation Qulity Index model outputs}\label{\detokenize{Introduction/admin:id1}}\end{figure}


\subsection{Guest user polygon size limit}
\label{\detokenize{Introduction/admin:guest-user-polygon-size-limit}}
\sphinxAtStartPar
Use this function to limit the size of the area for which unregistered users (Guest) can process data on the fly. The admin of the system can choose to turn off this option and therefore allow guest users to process large amounts of datasets on the fly.

\begin{figure}[H]
\centering
\capstart

\noindent\sphinxincludegraphics[width=744\sphinxpxdimen,height=584\sphinxpxdimen]{{admin3}.png}
\caption{Guest Polygon Size limit}\label{\detokenize{Introduction/admin:id2}}\end{figure}


\subsection{Signed up user polygon size limit}
\label{\detokenize{Introduction/admin:signed-up-user-polygon-size-limit}}
\sphinxAtStartPar
This limits the maximum size of an area a registered user can compute on the fly. Registered users have the option of scheduling processing tasks that use datasets that cover areas larger than the set polygon size limit. Tasks scheduling can be enabled/disabled by checking/unchecking the Enable tasks scheduling option.

\begin{figure}[H]
\centering
\capstart

\noindent\sphinxincludegraphics[width=744\sphinxpxdimen,height=584\sphinxpxdimen]{{admin3}.png}
\caption{Guest Polygon Size limit}\label{\detokenize{Introduction/admin:id3}}\end{figure}


\subsection{Enable user account activation}
\label{\detokenize{Introduction/admin:enable-user-account-activation}}
\sphinxAtStartPar
When this option is turned on, a verification email is sent to every user who registers with the system. This ensures that users only register with emails that they have access to, and are therefore able to get notifications related to their activities within the system.


\subsection{Email, URLs and host email host port setup}
\label{\detokenize{Introduction/admin:email-urls-and-host-email-host-port-setup}}
\sphinxAtStartPar
These options are crucial for connecting the backend to the frontend and are most likely to be setup only once during the initial system setup. Here you can set up the different emails required to ensure smooth working of the system. In addition to setting up the email urls, you can configure the raster clipping algorithm that is used when performing computation under Raster clipping algorithm option.

\begin{figure}[H]
\centering
\capstart

\noindent\sphinxincludegraphics[width=667\sphinxpxdimen,height=585\sphinxpxdimen]{{admin4a}.png}
\caption{Email, URLs and host email host port setup}\label{\detokenize{Introduction/admin:id4}}\end{figure}


\subsection{Enable Cache Limit}
\label{\detokenize{Introduction/admin:enable-cache-limit}}
\sphinxAtStartPar
The system features a cache that enables the system to store pre\sphinxhyphen{}computed outputs of the system. The Cache limit is set in seconds, which indicates how long the results should be stored in the system once they are computed.

\begin{figure}[H]
\centering
\capstart

\noindent\sphinxincludegraphics[width=663\sphinxpxdimen,height=287\sphinxpxdimen]{{admin4}.png}
\caption{Enable Cashe Limit options}\label{\detokenize{Introduction/admin:id5}}\end{figure}


\subsection{Backend port}
\label{\detokenize{Introduction/admin:backend-port}}
\sphinxAtStartPar
The system is designed to automatically generate a backend port, but it can be set manually by checking the override backend port option.
When all configuration for the system settings are done, click the Save button to apply all the changes made to system settings.

\begin{figure}[H]
\centering
\capstart

\noindent\sphinxincludegraphics[width=663\sphinxpxdimen,height=287\sphinxpxdimen]{{admin4}.png}
\caption{Enable Cashe Limit options}\label{\detokenize{Introduction/admin:id6}}\end{figure}


\section{Scheduled Tasks}
\label{\detokenize{Introduction/admin:scheduled-tasks}}
\sphinxAtStartPar
As mentioned before, the system enables users to schedule tasks that involve computation of relatively large dataset. To view all the scheduled tasks, select the Scheduled Tasks option in the admin panel. This option lists all scheduled tasks with information related to the tasks, including who scheduled them, when they were scheduled and the current status of those tasks.

\begin{figure}[H]
\centering
\capstart

\noindent\sphinxincludegraphics[width=762\sphinxpxdimen,height=629\sphinxpxdimen]{{admin5}.png}
\caption{Schedule task}\label{\detokenize{Introduction/admin:id7}}\end{figure}


\section{Uploading and Manipulating data in the system}
\label{\detokenize{Introduction/admin:uploading-and-manipulating-data-in-the-system}}
\sphinxAtStartPar
The Raster option enables system admin to upload, filter, delete and edit the raster datasets.

\begin{figure}[H]
\centering
\capstart

\noindent\sphinxincludegraphics[width=718\sphinxpxdimen,height=599\sphinxpxdimen]{{admin8}.png}
\caption{Uploading rasters}\label{\detokenize{Introduction/admin:id8}}\end{figure}


\subsection{Uploading data set}
\label{\detokenize{Introduction/admin:uploading-data-set}}
\sphinxAtStartPar
To upload the dataset, click on the Add Raster button. This will open a data upload panel that will guide you through the data upload process.

\begin{figure}[H]
\centering
\capstart

\noindent\sphinxincludegraphics[width=813\sphinxpxdimen,height=591\sphinxpxdimen]{{admin9}.png}
\caption{Add raster form}\label{\detokenize{Introduction/admin:id9}}\end{figure}


\subsection{Uploading Medalus data}
\label{\detokenize{Introduction/admin:uploading-medalus-data}}
\sphinxAtStartPar
Different types of datasets have different fields based on their roles in the system. For example, all medalus datasets must be associated with Aspect so that the system associates them with the computation modules for mebalus, see figure 7.


\subsection{Uploading Landsat data}
\label{\detokenize{Introduction/admin:uploading-landsat-data}}
\sphinxAtStartPar
Landsat rasters are relatively heavier in size per unit area, and should therefore be uploaded at country level. For this reason, in addition to providing the basic information such as name and year, they must be associated with their corresponding administrative level, e.g. admin level 0, see figure 8. Some of the datasets that are derived from landsat satellite images include MSAVI, SAVI and NDVI.
Uploading the other datasets
Uploading the other dataset is very similar to uploading Medalus dataset. LULC, Modis derived Vegetation indices, Carbon stock (SOC) and ecological units dataset must be uploaded at the regional scale, i.e. one file for the whole of North African States.


\section{Adding/editing Question}
\label{\detokenize{Introduction/admin:adding-editing-question}}
\sphinxAtStartPar
The Frontend of Misland provide a web page through which Frequently Asked Questions (FAQ) are displayed.  The Question option of the admin panel provides a user friendly  tool  through which FAQs can be managed. From here, new FAQ can be added, outdated ones deleted or deactivated, and existing ones edited.

\begin{figure}[H]
\centering
\capstart

\noindent\sphinxincludegraphics[width=663\sphinxpxdimen,height=287\sphinxpxdimen]{{admin4}.png}
\caption{Questions Section}\label{\detokenize{Introduction/admin:id10}}\end{figure}

\sphinxAtStartPar
Note that you can have several FAQs in the system but only display a few of them by activating and deactivating them.

\sphinxAtStartPar
Figure 11.


\section{Modifying Gallery Items}
\label{\detokenize{Introduction/admin:modifying-gallery-items}}
\sphinxAtStartPar
The Gallery option on the admin panel is used to Upload new images on the homepage of Misland. The system is designed in such a way that you do not need to delete old images, all you need to activate (check/uncheck is published) the image you want displayed, and deactivate outdated images.

\begin{figure}[H]
\centering
\capstart

\noindent\sphinxincludegraphics[width=741\sphinxpxdimen,height=586\sphinxpxdimen]{{admin12}.png}
\caption{Publishing Gallery Intems}\label{\detokenize{Introduction/admin:id11}}\end{figure}


\section{Managing Custom shapefile}
\label{\detokenize{Introduction/admin:managing-custom-shapefile}}
\sphinxAtStartPar
This option enables the system administrator to view and manage all the custom shapefiles (Shapefiles uploaded by individual users).


\section{Computation Threshold}
\label{\detokenize{Introduction/admin:computation-threshold}}
\sphinxAtStartPar
Computation thresholds are used to limit the amount of area that can be computed in real time within the system. This is an important feature of the system as it enables the system administrator to manage the computing resources of the system.  There are two main thresholds that are set here;  The Modis threshold, which applies to all dataset with spatial resolution greater than 30m, and Landsat threshold, which apply to dataset with spatial resolution of 30m.


\section{Managing Users}
\label{\detokenize{Introduction/admin:managing-users}}
\sphinxAtStartPar
The admin panel provides tools for managing registered system users. Using this functionality, the system admin can create new user, activate/deactivate existing user and assign different privileges to different them.

\begin{figure}[H]
\centering
\capstart

\noindent\sphinxincludegraphics[width=610\sphinxpxdimen,height=234\sphinxpxdimen]{{admin14}.png}
\caption{Gallery}\label{\detokenize{Introduction/admin:id12}}\end{figure}

\sphinxAtStartPar
To modify the access rights of a particular system user, just click that user’s name on the system panel and implement the desired changes. After the modification, just click the save button and the changes will take effect.


\section{User Feedback}
\label{\detokenize{Introduction/admin:user-feedback}}
\sphinxAtStartPar
User feedback is sent to a github account for which the system admin can login and take appropriate action. User feedback template is able to submit both text and images sent by the users of the system.


\section{Google analytics}
\label{\detokenize{Introduction/admin:google-analytics}}
\sphinxAtStartPar
The system uses google analytics to track user visits to this online service. Some of the information google analytics is able to provide include the number of visitors who are currently active on the system, the number of visits in a particular period, the countries from which online traffic is coming from among other information. Figure 16 shows sample information provided by google analytics.
\sphinxhref{https://analytics.google.com/analytics/web/?authuser=2\#/report-home/a184032602w254258877p233620264}{Google Analytics Link}

\sphinxstepscope


\chapter{Data and Data sources}
\label{\detokenize{Introduction/data:data-and-data-sources}}\label{\detokenize{Introduction/data::doc}}
\sphinxAtStartPar
MISLAND Africa draws on a number of data sources. The data sets listed below are
owned/made available by the following organizations and individuals under
separate terms as indicated in their respective metadata.


\section{NDVI}
\label{\detokenize{Introduction/data:ndvi}}

\begin{savenotes}\sphinxattablestart
\sphinxthistablewithglobalstyle
\centering
\begin{tabulary}{\linewidth}[t]{|T|T|T|T|T|}
\sphinxtoprule
\sphinxstyletheadfamily 
\sphinxAtStartPar
Sensor/Dataset
&\sphinxstyletheadfamily 
\sphinxAtStartPar
Temporal
&\sphinxstyletheadfamily 
\sphinxAtStartPar
Spatial
&\sphinxstyletheadfamily 
\sphinxAtStartPar
Extent
&\sphinxstyletheadfamily 
\sphinxAtStartPar
License
\\
\sphinxmidrule
\sphinxtableatstartofbodyhook
\sphinxAtStartPar
\sphinxhref{https://developers.google.com/earth-engine/datasets/catalog/landsat}{LANDSAT7}
&
\sphinxAtStartPar
2001\sphinxhyphen{}2020
&
\sphinxAtStartPar
30 m
&
\sphinxAtStartPar
Global
&
\sphinxAtStartPar
\sphinxhref{https://creativecommons.org/publicdomain/zero/1.0}{Public Domain}
\\
\sphinxhline
\sphinxAtStartPar
\sphinxhref{https://lpdaac.usgs.gov/dataset\_discovery/modis/modis\_products\_table/mod13q1\_v006}{MOD13Q1\sphinxhyphen{}coll6}
&
\sphinxAtStartPar
2001\sphinxhyphen{}2016
&
\sphinxAtStartPar
250 m
&
\sphinxAtStartPar
Global
&
\sphinxAtStartPar
\sphinxhref{https://creativecommons.org/publicdomain/zero/1.0}{Public Domain}
\\
\sphinxbottomrule
\end{tabulary}
\sphinxtableafterendhook\par
\sphinxattableend\end{savenotes}


\section{Soil moisture}
\label{\detokenize{Introduction/data:soil-moisture}}

\begin{savenotes}\sphinxattablestart
\sphinxthistablewithglobalstyle
\centering
\begin{tabulary}{\linewidth}[t]{|T|T|T|T|T|}
\sphinxtoprule
\sphinxstyletheadfamily 
\sphinxAtStartPar
Sensor/Dataset
&\sphinxstyletheadfamily 
\sphinxAtStartPar
Temporal
&\sphinxstyletheadfamily 
\sphinxAtStartPar
Spatial
&\sphinxstyletheadfamily 
\sphinxAtStartPar
Extent
&\sphinxstyletheadfamily 
\sphinxAtStartPar
License
\\
\sphinxmidrule
\sphinxtableatstartofbodyhook
\sphinxAtStartPar
\sphinxhref{https://gmao.gsfc.nasa.gov/reanalysis/MERRA-Land}{MERRA 2}
&
\sphinxAtStartPar
1980\sphinxhyphen{}2016
&
\sphinxAtStartPar
0.5° x 0.625°
&
\sphinxAtStartPar
Global
&
\sphinxAtStartPar
\sphinxhref{https://creativecommons.org/publicdomain/zero/1.0}{Public Domain}
\\
\sphinxhline
\sphinxAtStartPar
\sphinxhref{https://www.ecmwf.int/en/forecasts/datasets/reanalysis-datasets/era-interim-land}{ERA I}
&
\sphinxAtStartPar
1979\sphinxhyphen{}2016
&
\sphinxAtStartPar
0.75° x 0.75°
&
\sphinxAtStartPar
Global
&
\sphinxAtStartPar
\sphinxhref{https://creativecommons.org/publicdomain/zero/1.0}{Public Domain}
\\
\sphinxbottomrule
\end{tabulary}
\sphinxtableafterendhook\par
\sphinxattableend\end{savenotes}


\section{Precipitation}
\label{\detokenize{Introduction/data:precipitation}}

\begin{savenotes}\sphinxattablestart
\sphinxthistablewithglobalstyle
\centering
\begin{tabulary}{\linewidth}[t]{|T|T|T|T|T|}
\sphinxtoprule
\sphinxstyletheadfamily 
\sphinxAtStartPar
Sensor/Dataset
&\sphinxstyletheadfamily 
\sphinxAtStartPar
Temporal
&\sphinxstyletheadfamily 
\sphinxAtStartPar
Spatial
&\sphinxstyletheadfamily 
\sphinxAtStartPar
Extent
&\sphinxstyletheadfamily 
\sphinxAtStartPar
License
\\
\sphinxmidrule
\sphinxtableatstartofbodyhook
\sphinxAtStartPar
\sphinxhref{https://www.esrl.noaa.gov/psd/data/gridded/data.gpcp.html}{GPCP v2.3 1 month}
&
\sphinxAtStartPar
1979\sphinxhyphen{}2019
&
\sphinxAtStartPar
2.5° x 2.5°
&
\sphinxAtStartPar
Global
&
\sphinxAtStartPar
\sphinxhref{https://creativecommons.org/publicdomain/zero/1.0}{Public Domain}
\\
\sphinxhline
\sphinxAtStartPar
\sphinxhref{https://www.esrl.noaa.gov/psd/data/gridded/data.gpcc.html}{GPCC V6}
&
\sphinxAtStartPar
1891\sphinxhyphen{}2019
&
\sphinxAtStartPar
1° x 1°
&
\sphinxAtStartPar
Global
&
\sphinxAtStartPar
\sphinxhref{https://creativecommons.org/publicdomain/zero/1.0}{Public Domain}
\\
\sphinxhline
\sphinxAtStartPar
\sphinxhref{http://chg.geog.ucsb.edu/data/chirps}{CHIRPS}
&
\sphinxAtStartPar
1981\sphinxhyphen{}2016
&
\sphinxAtStartPar
5 km
&
\sphinxAtStartPar
50N\sphinxhyphen{}50S
&
\sphinxAtStartPar
\sphinxhref{https://creativecommons.org/publicdomain/zero/1.0}{Public Domain}
\\
\sphinxhline
\sphinxAtStartPar
\sphinxhref{http://chrsdata.eng.uci.edu}{PERSIANN\sphinxhyphen{}CDR}
&
\sphinxAtStartPar
1983\sphinxhyphen{}2015
&
\sphinxAtStartPar
25 km
&
\sphinxAtStartPar
60N\sphinxhyphen{}60S
&
\sphinxAtStartPar
\sphinxhref{https://creativecommons.org/publicdomain/zero/1.0}{Public Domain}
\\
\sphinxbottomrule
\end{tabulary}
\sphinxtableafterendhook\par
\sphinxattableend\end{savenotes}


\section{Evapotranspiration}
\label{\detokenize{Introduction/data:evapotranspiration}}

\begin{savenotes}\sphinxattablestart
\sphinxthistablewithglobalstyle
\centering
\begin{tabulary}{\linewidth}[t]{|T|T|T|T|T|}
\sphinxtoprule
\sphinxstyletheadfamily 
\sphinxAtStartPar
Sensor/Dataset
&\sphinxstyletheadfamily 
\sphinxAtStartPar
Temporal
&\sphinxstyletheadfamily 
\sphinxAtStartPar
Spatial
&\sphinxstyletheadfamily 
\sphinxAtStartPar
Extent
&\sphinxstyletheadfamily 
\sphinxAtStartPar
License
\\
\sphinxmidrule
\sphinxtableatstartofbodyhook
\sphinxAtStartPar
\sphinxhref{https://lpdaac.usgs.gov/dataset\_discovery/modis/modis\_products\_table/mod16a2\_v006}{MOD16A2}
&
\sphinxAtStartPar
2000\sphinxhyphen{}2014
&
\sphinxAtStartPar
1 km
&
\sphinxAtStartPar
Global
&
\sphinxAtStartPar
\sphinxhref{https://creativecommons.org/publicdomain/zero/1.0}{Public Domain}
\\
\sphinxbottomrule
\end{tabulary}
\sphinxtableafterendhook\par
\sphinxattableend\end{savenotes}


\section{Land cover}
\label{\detokenize{Introduction/data:land-cover}}

\begin{savenotes}\sphinxattablestart
\sphinxthistablewithglobalstyle
\centering
\begin{tabulary}{\linewidth}[t]{|T|T|T|T|T|}
\sphinxtoprule
\sphinxstyletheadfamily 
\sphinxAtStartPar
Sensor/Dataset
&\sphinxstyletheadfamily 
\sphinxAtStartPar
Temporal
&\sphinxstyletheadfamily 
\sphinxAtStartPar
Spatial
&\sphinxstyletheadfamily 
\sphinxAtStartPar
Extent
&\sphinxstyletheadfamily 
\sphinxAtStartPar
License
\\
\sphinxmidrule
\sphinxtableatstartofbodyhook
\sphinxAtStartPar
\sphinxhref{https://www.esa-landcover-cci.org}{ESA CCI Land Cover}
&
\sphinxAtStartPar
1992\sphinxhyphen{}2018
&
\sphinxAtStartPar
300 m
&
\sphinxAtStartPar
Global
&
\sphinxAtStartPar
\sphinxhref{https://creativecommons.org/licenses/by-sa/3.0/igo}{CC by\sphinxhyphen{}SA 3.0}
\\
\sphinxbottomrule
\end{tabulary}
\sphinxtableafterendhook\par
\sphinxattableend\end{savenotes}


\section{Soil carbon}
\label{\detokenize{Introduction/data:soil-carbon}}

\begin{savenotes}\sphinxattablestart
\sphinxthistablewithglobalstyle
\centering
\begin{tabulary}{\linewidth}[t]{|T|T|T|T|T|}
\sphinxtoprule
\sphinxstyletheadfamily 
\sphinxAtStartPar
Sensor/Dataset
&\sphinxstyletheadfamily 
\sphinxAtStartPar
Temporal
&\sphinxstyletheadfamily 
\sphinxAtStartPar
Spatial
&\sphinxstyletheadfamily 
\sphinxAtStartPar
Extent
&\sphinxstyletheadfamily 
\sphinxAtStartPar
License
\\
\sphinxmidrule
\sphinxtableatstartofbodyhook
\sphinxAtStartPar
\sphinxhref{https://www.soilgrids.org/}{Soil Grids (ISRIC)}
&
\sphinxAtStartPar
Present
&
\sphinxAtStartPar
250 m
&
\sphinxAtStartPar
Global
&
\sphinxAtStartPar
\sphinxhref{https://creativecommons.org/licenses/by-sa/4.0}{CC by\sphinxhyphen{}SA 4.0}
\\
\sphinxbottomrule
\end{tabulary}
\sphinxtableafterendhook\par
\sphinxattableend\end{savenotes}


\section{Agroecological Zones}
\label{\detokenize{Introduction/data:agroecological-zones}}

\begin{savenotes}\sphinxattablestart
\sphinxthistablewithglobalstyle
\centering
\begin{tabulary}{\linewidth}[t]{|T|T|T|T|T|}
\sphinxtoprule
\sphinxstyletheadfamily 
\sphinxAtStartPar
Sensor/Dataset
&\sphinxstyletheadfamily 
\sphinxAtStartPar
Temporal
&\sphinxstyletheadfamily 
\sphinxAtStartPar
Spatial
&\sphinxstyletheadfamily 
\sphinxAtStartPar
Extent
&\sphinxstyletheadfamily 
\sphinxAtStartPar
License
\\
\sphinxmidrule
\sphinxtableatstartofbodyhook
\sphinxAtStartPar
\sphinxhref{http://www.fao.org/nr/gaez/en}{FAO \sphinxhyphen{} IIASA Global Agroecological Zones (GAEZ)}
&
\sphinxAtStartPar
2000
&
\sphinxAtStartPar
8 km
&
\sphinxAtStartPar
Global
&
\sphinxAtStartPar
\sphinxhref{https://creativecommons.org/publicdomain/zero/1.0}{Public Domain}
\\
\sphinxbottomrule
\end{tabulary}
\sphinxtableafterendhook\par
\sphinxattableend\end{savenotes}


\section{Soil Quality}
\label{\detokenize{Introduction/data:soil-quality}}

\begin{savenotes}\sphinxattablestart
\sphinxthistablewithglobalstyle
\centering
\begin{tabulary}{\linewidth}[t]{|T|T|T|T|T|}
\sphinxtoprule
\sphinxstyletheadfamily 
\sphinxAtStartPar
Sensor/Dataset
&\sphinxstyletheadfamily 
\sphinxAtStartPar
Temporal
&\sphinxstyletheadfamily 
\sphinxAtStartPar
Spatial
&\sphinxstyletheadfamily 
\sphinxAtStartPar
Extent
&\sphinxstyletheadfamily 
\sphinxAtStartPar
License
\\
\sphinxmidrule
\sphinxtableatstartofbodyhook
\sphinxAtStartPar
\sphinxhref{https://cmr.earthdata.nasa.gov/search/concepts/C1000000240-LPDAAC\_ECS.html}{Soil Texture and Depth}
&
\sphinxAtStartPar
Present
&
\sphinxAtStartPar
250 m
&
\sphinxAtStartPar
Global
&
\sphinxAtStartPar
\sphinxhref{https://creativecommons.org/licenses/by-sa/4.0}{CC by\sphinxhyphen{}SA 4.0}
\\
\sphinxhline
\sphinxAtStartPar
\sphinxhref{https://doi.pangaea.de/10.1594/PANGAEA.788537}{Parent Material}
&
\sphinxAtStartPar
Present
&
\sphinxAtStartPar
N/A
&
\sphinxAtStartPar
Global
&
\sphinxAtStartPar
\sphinxhref{https://creativecommons.org/licenses/by-sa/4.0}{CC by\sphinxhyphen{}SA 4.0}
\\
\sphinxhline
\sphinxAtStartPar
\sphinxhref{https://developers.google.com/earth-engine/datasets/catalog/OpenLandMap\_SOL\_SOL\_TEXTURE-CLASS\_USDA-TT\_M\_v02}{Slope}
&
\sphinxAtStartPar
Present
&
\sphinxAtStartPar
30 m
&\sphinxstartmulticolumn{2}%
\begin{varwidth}[t]{\sphinxcolwidth{2}{5}}
\sphinxAtStartPar
Global  | \sphinxhref{https://www.jpl.nasa.gov/imagepolicy/}{JPL public}
\par
\vskip-\baselineskip\vbox{\hbox{\strut}}\end{varwidth}%
\sphinxstopmulticolumn
\\
\sphinxbottomrule
\end{tabulary}
\sphinxtableafterendhook\par
\sphinxattableend\end{savenotes}


\section{Climate}
\label{\detokenize{Introduction/data:climate}}

\begin{savenotes}\sphinxattablestart
\sphinxthistablewithglobalstyle
\centering
\begin{tabulary}{\linewidth}[t]{|T|T|T|T|T|}
\sphinxtoprule
\sphinxstyletheadfamily 
\sphinxAtStartPar
Sensor/Dataset
&\sphinxstyletheadfamily 
\sphinxAtStartPar
Temporal
&\sphinxstyletheadfamily 
\sphinxAtStartPar
Spatial
&\sphinxstyletheadfamily 
\sphinxAtStartPar
Extent
&\sphinxstyletheadfamily 
\sphinxAtStartPar
License
\\
\sphinxmidrule
\sphinxtableatstartofbodyhook
\sphinxAtStartPar
\sphinxhref{https://developers.google.com/earth-engine/datasets/catalog/IDAHO\_EPSCOR\_TERRACLIMATE\#description}{Terra Climate}
&
\sphinxAtStartPar
1985\sphinxhyphen{}2019
&
\sphinxAtStartPar
30 m
&
\sphinxAtStartPar
Global
&
\sphinxAtStartPar
\sphinxhref{https://creativecommons.org/publicdomain/zero/1.0}{Public Domain}
\\
\sphinxbottomrule
\end{tabulary}
\sphinxtableafterendhook\par
\sphinxattableend\end{savenotes}


\section{Administrative Boundaries}
\label{\detokenize{Introduction/data:administrative-boundaries}}

\begin{savenotes}\sphinxattablestart
\sphinxthistablewithglobalstyle
\centering
\begin{tabulary}{\linewidth}[t]{|T|T|T|T|T|}
\sphinxtoprule
\sphinxstyletheadfamily 
\sphinxAtStartPar
Sensor/Dataset
&\sphinxstyletheadfamily 
\sphinxAtStartPar
Temporal
&\sphinxstyletheadfamily 
\sphinxAtStartPar
Spatial
&\sphinxstyletheadfamily 
\sphinxAtStartPar
Extent
&\sphinxstyletheadfamily 
\sphinxAtStartPar
License
\\
\sphinxmidrule
\sphinxtableatstartofbodyhook
\sphinxAtStartPar
\sphinxhref{http://www.naturalearthdata.com}{Natural Earth Administrative Boundaries}
&
\sphinxAtStartPar
Present
&
\sphinxAtStartPar
10/50m
&
\sphinxAtStartPar
Global
&
\sphinxAtStartPar
\sphinxhref{https://creativecommons.org/publicdomain/zero/1.0}{Public Domain}
\\
\sphinxbottomrule
\end{tabulary}
\sphinxtableafterendhook\par
\sphinxattableend\end{savenotes}

\begin{sphinxadmonition}{note}{Note:}
\sphinxAtStartPar
The \sphinxhref{http://www.naturalearthdata.com}{Natural Earth Administrative Boundaries} provided in MISLAND\sphinxhyphen{}North Africa
are in the \sphinxhref{https://creativecommons.org/publicdomain/zero/1.0}{public domain}. The boundaries and names used, and the
designations used, in MISLAND\sphinxhyphen{}North Africa do not imply official endorsement or
acceptance by Conservation International Foundation, or by its partner
organizations and contributors.

\sphinxAtStartPar
If using MISLAND\sphinxhyphen{}North Africa for official purposes, it is recommended that users
choose an official boundary provided by the designated office of their
country.
\end{sphinxadmonition}

\sphinxstepscope


\chapter{Data Preparation Models}
\label{\detokenize{Introduction/models:data-preparation-models}}\label{\detokenize{Introduction/models::doc}}

\section{The graphical modeler}
\label{\detokenize{Introduction/models:the-graphical-modeler}}
\sphinxAtStartPar
The \sphinxstyleemphasis{graphical modeler} allows you to create complex models using
a simple and easy\sphinxhyphen{}to\sphinxhyphen{}use interface.
When working with a GIS, most analysis operations are not
isolated, rather part of a chain of operations.
Using the graphical modeler, that chain of operations can be wrapped
into a single process, making it convenient to execute later with a
different set of inputs.
No matter how many steps and different algorithms it involves, a
model is executed as a single algorithm, saving time and effort.

\sphinxAtStartPar
The graphical modeler can be opened from the Processing menu
(\sphinxmenuselection{Processing \(\rightarrow\) Graphical Modeler}).

\sphinxAtStartPar
The modeler has a working canvas where the structure of the model and
the workflow it represents are shown.
The left part of the window is a section with five panels that can be used
to add new elements to the model:
\begin{enumerate}
\sphinxsetlistlabels{\arabic}{enumi}{enumii}{}{.}%
\item {} 
\sphinxAtStartPar
\sphinxcode{\sphinxupquote{Model Properties}}: you can specify the name of the model and the group that
will contain it

\item {} 
\sphinxAtStartPar
\sphinxcode{\sphinxupquote{Inputs}}: all the inputs that will shape your model

\item {} 
\sphinxAtStartPar
\sphinxcode{\sphinxupquote{Algorithms}}: the Processing algorithms available

\item {} 
\sphinxAtStartPar
\sphinxcode{\sphinxupquote{Variables}}: you can also define variables that will only be available in
the Processing Modeler

\item {} 
\sphinxAtStartPar
\sphinxcode{\sphinxupquote{Undo History}}: this panel will register everything that happens in the
modeler, making it easy to cancel things you did wrong.

\end{enumerate}

\begin{figure}[H]
\centering
\capstart

\noindent\sphinxincludegraphics{{canvas}.png}
\caption{Modeler}\label{\detokenize{Introduction/models:id1}}\label{\detokenize{Introduction/models:figure-modeler}}\end{figure}

\sphinxAtStartPar
Creating a model involves two basic steps:
\begin{enumerate}
\sphinxsetlistlabels{\arabic}{enumi}{enumii}{}{.}%
\item {} 
\sphinxAtStartPar
\sphinxstyleemphasis{Definition of necessary inputs}.
These inputs will be added to the parameters window, so the user
can set their values when executing the model.
The model itself is an algorithm, so the parameters window is
generated automatically as for all algorithms
available in the Processing framework.

\item {} 
\sphinxAtStartPar
\sphinxstyleemphasis{Definition of the workflow}.
Using the input data of the model, the workflow is defined by
adding algorithms and selecting how they use the defined inputs
or the outputs generated by other algorithms in the model.

\end{enumerate}


\subsection{Definition of inputs}
\label{\detokenize{Introduction/models:definition-of-inputs}}
\sphinxAtStartPar
The first step is to define the inputs for the model.
The following elements are found in the \sphinxguilabel{Inputs} panel on
the left side of the modeler window:
\begin{itemize}
\item {} 
\sphinxAtStartPar
Authentication Configuration

\item {} 
\sphinxAtStartPar
Boolean

\item {} 
\sphinxAtStartPar
Color

\item {} 
\sphinxAtStartPar
Connection Name

\item {} 
\sphinxAtStartPar
Coordinate Operation

\item {} 
\sphinxAtStartPar
CRS

\item {} 
\sphinxAtStartPar
Database Schema

\item {} 
\sphinxAtStartPar
Database Table

\item {} 
\sphinxAtStartPar
Datetime

\item {} 
\sphinxAtStartPar
Distance

\item {} 
\sphinxAtStartPar
DXF Layers

\item {} 
\sphinxAtStartPar
Enum

\item {} 
\sphinxAtStartPar
Expression

\item {} 
\sphinxAtStartPar
Extent

\item {} 
\sphinxAtStartPar
Field Aggregates

\item {} 
\sphinxAtStartPar
Fields Mapper

\item {} 
\sphinxAtStartPar
File/Folder

\item {} 
\sphinxAtStartPar
Geometry

\item {} 
\sphinxAtStartPar
Map Layer

\item {} 
\sphinxAtStartPar
Map Theme

\item {} 
\sphinxAtStartPar
Matrix

\item {} 
\sphinxAtStartPar
Mesh Dataset Groups

\item {} 
\sphinxAtStartPar
Mesh Dataset Time

\item {} 
\sphinxAtStartPar
Mesh Layer

\item {} 
\sphinxAtStartPar
Multiple Input

\item {} 
\sphinxAtStartPar
Number

\item {} 
\sphinxAtStartPar
Point

\item {} 
\sphinxAtStartPar
Print Layout

\item {} 
\sphinxAtStartPar
Print Layout Item

\item {} 
\sphinxAtStartPar
Range

\item {} 
\sphinxAtStartPar
Raster Band

\item {} 
\sphinxAtStartPar
Raster Layer

\item {} 
\sphinxAtStartPar
Scale

\item {} 
\sphinxAtStartPar
String

\item {} 
\sphinxAtStartPar
TIN Creation Layers

\item {} 
\sphinxAtStartPar
Vector Features

\item {} 
\sphinxAtStartPar
Vector Field

\item {} 
\sphinxAtStartPar
Vector Layer

\item {} 
\sphinxAtStartPar
Vector Tile Writer Layers

\end{itemize}

\begin{sphinxadmonition}{note}{Note:}
\sphinxAtStartPar
Hovering with the mouse over the inputs will show a tooltip with
additional information.
\end{sphinxadmonition}

\sphinxAtStartPar
When double\sphinxhyphen{}clicking on an element, a dialog is shown that lets
you define its characteristics.
Depending on the parameter, the dialog will contain at least one
element (the description, which is what the user will see when
executing the model).
For example, when adding a raster data, as can be seen in the next figure,
in addition to the description of the parameter.

\begin{figure}[H]
\centering
\capstart

\noindent\sphinxincludegraphics{{models_parameters}.png}
\caption{Model Parameters Definition}\label{\detokenize{Introduction/models:id2}}\label{\detokenize{Introduction/models:figure-model-parameter}}\end{figure}

\sphinxAtStartPar
You can define your input as mandatory for your model by checking the
\sphinxincludegraphics[width=1.3em]{{checkbox}.png} \sphinxcode{\sphinxupquote{Mandatory}} option and by checking the \sphinxincludegraphics[width=1.3em]{{checkbox_unchecked}.png} \sphinxcode{\sphinxupquote{Advanced}}
checkbox you can set the input to be within the \sphinxcode{\sphinxupquote{Advanced}} section. This is
particularly useful when the model has many parameters and some of them are not
trivial, but you still want to choose them.

\sphinxAtStartPar
The \sphinxcode{\sphinxupquote{Comments}} tab allows you to tag the input with more information,
to better describe
the parameter. Comments are visible only in the modeler canvas and not in the
final algorithm dialog.

\sphinxAtStartPar
For each added input, a new element is added to the modeler canvas.

\begin{figure}[H]
\centering
\capstart

\noindent\sphinxincludegraphics{{models_parameters2}.png}
\caption{Model Parameters}\label{\detokenize{Introduction/models:id3}}\label{\detokenize{Introduction/models:figure-model-parameter-canvas}}\end{figure}

\sphinxAtStartPar
You can also add inputs by dragging the input type from the list and
dropping it at the position where you want it in the modeler canvas. If you want
to change a parameter of an existing input, just double click on it, and the
same dialog will pop up.


\subsection{Definition of the workflow}
\label{\detokenize{Introduction/models:definition-of-the-workflow}}
\sphinxAtStartPar
In the following example we will add two inputs and two algorithms. The aim of
the model is to copy the elevation values from a DEM raster layer to a line layer
using the \sphinxcode{\sphinxupquote{Drape}} algorithm,  and then calculate the total ascent of the line
layer using the \sphinxcode{\sphinxupquote{Climb Along Line}} algorithm.

\sphinxAtStartPar
In the \sphinxguilabel{Inputs} tab, choose the two inputs as \sphinxcode{\sphinxupquote{Vector Layer}} for the line and
\sphinxcode{\sphinxupquote{Raster Layer}} for the DEM.
We are now ready to add the algorithms to the workflow.

\sphinxAtStartPar
Algorithms can be found in the \sphinxguilabel{Algorithms} panel, grouped
much in the same way as they are in the Processing toolbox.

\begin{figure}[H]
\centering
\capstart

\noindent\sphinxincludegraphics{{models_parameters3}.png}
\caption{Model Inputs}\label{\detokenize{Introduction/models:id4}}\label{\detokenize{Introduction/models:figure-model-parameter-inputs}}\end{figure}

\sphinxAtStartPar
To add an algorithm to a model, double\sphinxhyphen{}click on its name or drag and
drop it, just like for inputs. As for the inputs you can change the description
of the algorithm and add a comment.
When adding an algorithm, an execution dialog will appear, with a content similar
to the one found in the execution panel that is shown when executing the
algorithm from the toolbox.
The following picture shows both the \sphinxcode{\sphinxupquote{Mean PET}} and the
\sphinxcode{\sphinxupquote{Aridity Score}} algorithm dialogs.

\begin{figure}[H]
\centering
\capstart

\noindent\sphinxincludegraphics{{models_parameters4}.png}
\caption{Model Algorithm parameters}\label{\detokenize{Introduction/models:id5}}\label{\detokenize{Introduction/models:figure-model-parameter-alg}}\end{figure}

\sphinxAtStartPar
As you can see there are some differences.

\sphinxAtStartPar
You have four choices to define the algorithm \sphinxstylestrong{inputs}:
\begin{itemize}
\item {} 
\sphinxAtStartPar
\sphinxincludegraphics[width=1.5em]{{mIconFieldInteger}.png} \sphinxcode{\sphinxupquote{Value}}: allows you to set the parameter from a loaded
layer in the QGIS project or to browse a layer from a folder

\item {} 
\sphinxAtStartPar
\sphinxincludegraphics[width=1.5em]{{mIconExpression}.png} \sphinxcode{\sphinxupquote{Pre\sphinxhyphen{}calculated Value}}: with this option you can open the
Expression Builder and define your own expression to fill the parameter. Model
inputs together with some other layer statistics are available as \sphinxstylestrong{variables}
and are listed at the top of the Search dialog of the Expression Builder

\item {} 
\sphinxAtStartPar
\sphinxincludegraphics[width=1.5em]{{processingModel}.png} \sphinxcode{\sphinxupquote{Model Input}}: choose this option if the
parameter comes from an input of the model you have defined. Once clicked, this
option will list all the suitable inputs for the parameter

\item {} 
\sphinxAtStartPar
\sphinxincludegraphics[width=1.5em]{{processingAlgorithm}.png} \sphinxcode{\sphinxupquote{Algorithm Output}}: is useful when the input
parameter of an algorithm is an output of another algorithm

\end{itemize}

\sphinxAtStartPar
Algorithm \sphinxstylestrong{outputs} have the addditional \sphinxincludegraphics[width=1.5em]{{mIconModelOutput}.png} \sphinxcode{\sphinxupquote{Model Output}}
option that makes the output of the algorithm available in the model.

\sphinxAtStartPar
If a layer generated by the algorithm is only to be used as input to another
algorithm,  don’t edit that text box.

\sphinxAtStartPar
In the following picture you can see the two input parameters defined as
\sphinxcode{\sphinxupquote{Model Input}} and the temporary output layer:

\begin{figure}[H]
\centering
\capstart

\noindent\sphinxincludegraphics{{models_parameters5}.png}
\caption{Algorithm Input and Output parameters}\label{\detokenize{Introduction/models:id6}}\end{figure}

\sphinxAtStartPar
In all cases, you will find an additional parameter named
\sphinxstyleemphasis{Dependencies} that is not available when calling the algorithm
from the toolbox.
This parameter allows you to define the order in which algorithms are
executed, by explicitly defining one algorithm as a \sphinxstyleemphasis{parent} of the current
one.
This will force the \sphinxstyleemphasis{parent} algorithm to be executed before the current one.

\sphinxAtStartPar
When you use the output of a previous algorithm as the input of your
algorithm, that implicitly sets the previous algorithm as parent of the
current one (and places the corresponding arrow in the modeler canvas).
However, in some cases an algorithm might depend on another one even if
it does not use any output object from it (for instance, an algorithm
that executes a SQL sentence on a PostGIS database and another one that
imports a layer into that same database).
In that case, just select the previous algorithm in the
\sphinxstyleemphasis{Dependencies} parameter and they will be executed in the correct
order.

\sphinxAtStartPar
Once all the parameters have been assigned valid values, click on
\sphinxguilabel{OK} and the algorithm will be added to the canvas.
It will be linked to the elements in the canvas (algorithms or inputs)
that provide objects that are used as inputs for the algorithm.

\sphinxAtStartPar
Elements can be dragged to a different position on the canvas.
This is useful to make the structure of the model more clear and
intuitive.
You can also resize elements.
This is particularly useful if the description of the input or algorithm is long.

\sphinxAtStartPar
Links between elements are updated automatically and you can see a plus button
at the top and at the bottom of each algorithm. Clicking the button will list
all the inputs and outputs of the algorithm so you can have a quick overview.

\sphinxAtStartPar
You can zoom in and out by using the mouse wheel.

\begin{figure}[H]
\centering
\capstart

\noindent\sphinxincludegraphics{{models_model}.png}
\caption{A complete model}\label{\detokenize{Introduction/models:id7}}\label{\detokenize{Introduction/models:figure-model-model}}\end{figure}

\sphinxAtStartPar
You can run your algorithm any time by clicking on the \sphinxincludegraphics[width=1.5em]{{mActionStart}.png} button.
In order to use the algorithm from the toolbox, it has to be saved
and the modeler dialog closed, to allow the toolbox to refresh its
contents.


\subsection{Interacting with the canvas and elements}
\label{\detokenize{Introduction/models:interacting-with-the-canvas-and-elements}}
\sphinxAtStartPar
You can use the \sphinxincludegraphics[width=1.5em]{{mActionZoomIn}.png}, \sphinxincludegraphics[width=1.5em]{{processingModel}.png}, \sphinxincludegraphics[width=1.5em]{{mActionZoomActual}.png} and \sphinxincludegraphics[width=1.5em]{{mActionZoomFullExtent}.png} buttons
to zoom the modeler canvas. The behavior of the buttons is basically the same
of the main QGIS toolbar.

\sphinxAtStartPar
The \sphinxcode{\sphinxupquote{Undo History}} panel together with the \sphinxincludegraphics[width=1.5em]{{mActionUndo}.png} and \sphinxincludegraphics[width=1.5em]{{mActionRedo}.png} buttons are
extremely useful to quickly rollback to a previous situation. The \sphinxcode{\sphinxupquote{Undo History}}
panel lists everything you have done when creating the workflow.

\sphinxAtStartPar
You can move or resize many elements at the same time by first selecting them,
dragging the mouse.

\sphinxAtStartPar
If you want to snap the elements while moving them in the canvas you can choose
\sphinxmenuselection{View \(\rightarrow\) Enable Snapping}.

\sphinxAtStartPar
The \sphinxmenuselection{Edit} menu contains some very useful options to interact with
your model elements:
\begin{itemize}
\item {} 
\sphinxAtStartPar
\sphinxincludegraphics[width=1.5em]{{mActionSelectAll}.png}$^{\text{Select All}}$: select all elements of the model

\item {} 
\sphinxAtStartPar
\sphinxcode{\sphinxupquote{Snap Selected Components to Grid}}: snap and align the elements into a
grid

\item {} 
\sphinxAtStartPar
\sphinxincludegraphics[width=1.5em]{{mActionUndo}.png}$^{\text{Undo}}$: undo the last action

\item {} 
\sphinxAtStartPar
\sphinxincludegraphics[width=1.5em]{{mActionRedo}.png}$^{\text{Redo}}$: redo the last action

\item {} 
\sphinxAtStartPar
\sphinxincludegraphics[width=1.5em]{{mActionEditCut}.png}$^{\text{Cut}}$: cut the selected elements

\item {} 
\sphinxAtStartPar
\sphinxincludegraphics[width=1.5em]{{mActionEditCopy}.png}$^{\text{Copy}}$: copy the selected elements

\item {} 
\sphinxAtStartPar
\sphinxincludegraphics[width=1.5em]{{mActionEditPaste}.png}$^{\text{Paste}}$: paste the elements

\item {} 
\sphinxAtStartPar
\sphinxincludegraphics[width=1.5em]{{mActionDeleteSelected}.png}$^{\text{Delete Selected Components}}$: delete all the selected
elements from the model

\item {} 
\sphinxAtStartPar
\sphinxcode{\sphinxupquote{Add Group Box}}: add a draggable \sphinxstyleemphasis{box} to the canvas. This feature is very
useful in big models to group elements in the modeler canvas and to keep the
workflow clean. For example we might group together all the inputs of the
example:
\begin{quote}

\begin{figure}[H]
\centering
\capstart

\noindent\sphinxincludegraphics{{model_group_box}.png}
\caption{Model Group Box}\label{\detokenize{Introduction/models:id8}}\end{figure}
\end{quote}

\end{itemize}

\sphinxAtStartPar
You can change the name and the color of the boxes.
Group boxes are very useful when used together with
\sphinxmenuselection{View \(\rightarrow\) Zoom To}.
This allows you to zoom to a specific part of the model.

\sphinxAtStartPar
You might want to change the order of the inputs and how they are listed in the
main model dialog. At the bottom of the \sphinxcode{\sphinxupquote{Input}} panel you will find the
\sphinxcode{\sphinxupquote{Reorder Model Inputs...}} button and by clicking on it a new dialog pops up
allowing you to change the order of the inputs:

\begin{figure}[H]
\centering
\capstart

\noindent\sphinxincludegraphics{{model_reorder_inputs}.png}
\caption{Reorder Model Inputs}\label{\detokenize{Introduction/models:id9}}\end{figure}


\subsection{Saving and loading models}
\label{\detokenize{Introduction/models:saving-and-loading-models}}
\sphinxAtStartPar
Use the \sphinxincludegraphics[width=1.5em]{{mActionFileSave}.png}$^{\text{Save model}}$ button to save the current model and the
\sphinxincludegraphics[width=1.5em]{{mActionFileOpen}.png}$^{\text{Open Model}}$ button to open a previously saved model.
Models are saved with the \sphinxcode{\sphinxupquote{.model3}} extension.
If the model has already been saved from the modeler window,
you will not be prompted for a filename.
Since there is already a file associated with the model, that file
will be used for subsequent saves.

\sphinxAtStartPar
Before saving a model, you have to enter a name and a group for it
in the text boxes in the upper part of the window.

\sphinxAtStartPar
Models saved in the \sphinxcode{\sphinxupquote{models}} folder (the default folder when you
are prompted for a filename to save the model) will appear in the
toolbox in the corresponding branch.
When the toolbox is invoked, it searches the \sphinxcode{\sphinxupquote{models}} folder for
files with the \sphinxcode{\sphinxupquote{.model3}} extension and loads the models they
contain.
Since a model is itself an algorithm, it can be added to the toolbox
just like any other algorithm.

\sphinxAtStartPar
Models can also be saved within the project file using the
\sphinxincludegraphics[width=1.5em]{{mAddToProject}.png}$^{\text{Save model in project}}$ button.
Models saved using this method won’t be written as \sphinxcode{\sphinxupquote{.model3}} files
on the disk but will be embedded in the project file.

\sphinxAtStartPar
Project models are available in the
\sphinxincludegraphics[width=1.5em]{{mIconQgsProjectFile}.png}\sphinxguilabel{Project models} menu of the toolbox.

\sphinxAtStartPar
The models folder can be set from the Processing configuration dialog,
under the \sphinxguilabel{Modeler} group.

\sphinxAtStartPar
Models loaded from the \sphinxcode{\sphinxupquote{models}} folder appear not only in the
toolbox, but also in the algorithms tree in the \sphinxguilabel{Algorithms}
tab of the modeler window.
That means that you can incorporate a model as a part of a bigger model,
just like other algorithms.

\sphinxAtStartPar
Models will show up in the \DUrole{xref,std,std-ref}{Browser} panel and can be run
from there.


\subsubsection{Exporting a model as an image, PDF or SVG}
\label{\detokenize{Introduction/models:exporting-a-model-as-an-image-pdf-or-svg}}
\sphinxAtStartPar
A model can also be exported as an image, SVG or PDF (for illustration
purposes) by clicking \sphinxincludegraphics[width=1.5em]{{mActionSaveMapAsImage}.png}$^{\text{Export as image}}$,
\sphinxincludegraphics[width=1.5em]{{mActionSaveAsPDF}.png}$^{\text{Export as PDF}}$ or \sphinxincludegraphics[width=1.5em]{{mActionSaveAsSVG}.png}$^{\text{Export as SVG}}$.


\subsection{Editing a model}
\label{\detokenize{Introduction/models:editing-a-model}}
\sphinxAtStartPar
You can edit the model you are currently creating, redefining the
workflow and the relationships between the algorithms and inputs that
define the model.

\sphinxAtStartPar
If you right\sphinxhyphen{}click on an algorithm in the canvas, you will see a context
menu like the one shown next:

\begin{figure}[H]
\centering
\capstart

\noindent\sphinxincludegraphics{{modeler_right_click}.png}
\caption{Modeler Right Click}\label{\detokenize{Introduction/models:id10}}\label{\detokenize{Introduction/models:figure-model-right-click}}\end{figure}

\sphinxAtStartPar
Selecting the \sphinxguilabel{Remove} option will cause the selected
algorithm to be removed.
An algorithm can be removed only if there are no other algorithms
depending on it.
That is, if no output from the algorithm is used in a different one as
input.
If you try to remove an algorithm that has others depending on it, a
warning message like the one you can see below will be shown:

\begin{figure}[H]
\centering
\capstart

\noindent\sphinxincludegraphics{{cannot_delete_alg}.png}
\caption{Cannot Delete Algorithm}\label{\detokenize{Introduction/models:id11}}\label{\detokenize{Introduction/models:figure-cannot-delete-alg}}\end{figure}

\sphinxAtStartPar
Selecting the \sphinxguilabel{Edit…} option will show the parameter dialog
of the algorithm, so you can change the inputs and parameter values.
Not all input elements available in the model will appear as
available inputs.
Layers or values generated at a more advanced step in the workflow
defined by the model will not be available if they cause circular
dependencies.

\sphinxAtStartPar
Select the new values and click on the \sphinxguilabel{OK} button as usual.
The connections between the model elements will change in the modeler
canvas accordingly.

\sphinxAtStartPar
The \sphinxguilabel{Add comment…} allows you to add a comment to the algorithm to
better describe the behavior.

\sphinxAtStartPar
A model can be run partially by deactivating some of its algorithms.
To do it, select the \sphinxguilabel{Deactivate} option in the context menu
that appears when right\sphinxhyphen{}clicking on an algorithm element.
The selected algorithm, and all the ones in the model that depend on it
will be displayed in grey and will not be executed as part of the model.

\begin{figure}[H]
\centering
\capstart

\noindent\sphinxincludegraphics{{deactivated}.png}
\caption{Model With Deactivated Algorithms}\label{\detokenize{Introduction/models:id12}}\label{\detokenize{Introduction/models:figure-cannot-model-deactivate}}\end{figure}

\sphinxAtStartPar
When right\sphinxhyphen{}clicking on an algorithm that is not active, you will
see a \sphinxguilabel{Activate} menu option that you can use to reactivate
it.


\subsection{Editing model help files and meta\sphinxhyphen{}information}
\label{\detokenize{Introduction/models:editing-model-help-files-and-meta-information}}
\sphinxAtStartPar
You can document your models from the modeler itself.
Click on the \sphinxincludegraphics[width=1.5em]{{mActionEditHelpContent}.png}$^{\text{Edit model help}}$ button, and a
dialog like the one shown next will appear.

\begin{figure}[H]
\centering
\capstart

\noindent\sphinxincludegraphics{{help_edition}.png}
\caption{Editing Help}\label{\detokenize{Introduction/models:id13}}\label{\detokenize{Introduction/models:figure-help-edition}}\end{figure}

\sphinxAtStartPar
On the right\sphinxhyphen{}hand side, you will see a simple HTML page, created using
the description of the input parameters and outputs of the algorithm,
along with some additional items like a general description of the
model or its author.
The first time you open the help editor, all these descriptions are
empty, but you can edit them using the elements on the left\sphinxhyphen{}hand side
of the dialog.
Select an element on the upper part and then write its description in
the text box below.

\sphinxAtStartPar
Model help is saved as part of the model itself.


\subsection{Exporting a model as a Python script}
\label{\detokenize{Introduction/models:exporting-a-model-as-a-python-script}}
\sphinxAtStartPar
As we will see in a later chapter, Processing algorithms can be called
from the QGIS Python console, and new Processing algorithms can be
created using Python.
A quick way to create such a Python script is to create a model and
then export it as a Python file.

\sphinxAtStartPar
To do so, click on the \sphinxincludegraphics[width=1.5em]{{mActionSaveAsPython}.png}$^{\text{Export as Script Algorithm…}}$
in the modeler canvas or right click on the name of the model in the Processing
Toolbox and choose \sphinxincludegraphics[width=1.5em]{{mActionSaveAsPython}.png}$^{\text{Export Model as Python Algorithm…}}$.

\sphinxstepscope


\chapter{Data sources}
\label{\detokenize{Introduction/Introduction:data-sources}}\label{\detokenize{Introduction/Introduction::doc}}
\sphinxAtStartPar
MISLAND Africa draws on a number of data sources. The data sets listed below are
owned/made available by the following organizations and individuals under
separate terms as indicated in their respective metadata.


\section{NDVI}
\label{\detokenize{Introduction/Introduction:ndvi}}

\begin{savenotes}\sphinxattablestart
\sphinxthistablewithglobalstyle
\centering
\begin{tabulary}{\linewidth}[t]{|T|T|T|T|T|}
\sphinxtoprule
\sphinxstyletheadfamily 
\sphinxAtStartPar
Sensor/Dataset
&\sphinxstyletheadfamily 
\sphinxAtStartPar
Temporal
&\sphinxstyletheadfamily 
\sphinxAtStartPar
Spatial
&\sphinxstyletheadfamily 
\sphinxAtStartPar
Extent
&\sphinxstyletheadfamily 
\sphinxAtStartPar
License
\\
\sphinxmidrule
\sphinxtableatstartofbodyhook
\sphinxAtStartPar
\sphinxhref{https://developers.google.com/earth-engine/datasets/catalog/landsat}{LANDSAT7}
&
\sphinxAtStartPar
2001\sphinxhyphen{}2020
&
\sphinxAtStartPar
30 m
&
\sphinxAtStartPar
Global
&
\sphinxAtStartPar
\sphinxhref{https://creativecommons.org/publicdomain/zero/1.0}{Public Domain}
\\
\sphinxhline
\sphinxAtStartPar
\sphinxhref{https://glam1.gsfc.nasa.gov}{AVHRR/GIMMS}
&
\sphinxAtStartPar
1982\sphinxhyphen{}2015
&
\sphinxAtStartPar
8 km
&
\sphinxAtStartPar
Global
&
\sphinxAtStartPar
\sphinxhref{https://creativecommons.org/publicdomain/zero/1.0}{Public Domain}
\\
\sphinxhline
\sphinxAtStartPar
\sphinxhref{https://lpdaac.usgs.gov/dataset\_discovery/modis/modis\_products\_table/mod13q1\_v006}{MOD13Q1\sphinxhyphen{}coll6}
&
\sphinxAtStartPar
2001\sphinxhyphen{}2016
&
\sphinxAtStartPar
250 m
&
\sphinxAtStartPar
Global
&
\sphinxAtStartPar
\sphinxhref{https://creativecommons.org/publicdomain/zero/1.0}{Public Domain}
\\
\sphinxbottomrule
\end{tabulary}
\sphinxtableafterendhook\par
\sphinxattableend\end{savenotes}


\section{Soil moisture}
\label{\detokenize{Introduction/Introduction:soil-moisture}}

\begin{savenotes}\sphinxattablestart
\sphinxthistablewithglobalstyle
\centering
\begin{tabulary}{\linewidth}[t]{|T|T|T|T|T|}
\sphinxtoprule
\sphinxstyletheadfamily 
\sphinxAtStartPar
Sensor/Dataset
&\sphinxstyletheadfamily 
\sphinxAtStartPar
Temporal
&\sphinxstyletheadfamily 
\sphinxAtStartPar
Spatial
&\sphinxstyletheadfamily 
\sphinxAtStartPar
Extent
&\sphinxstyletheadfamily 
\sphinxAtStartPar
License
\\
\sphinxmidrule
\sphinxtableatstartofbodyhook
\sphinxAtStartPar
\sphinxhref{https://gmao.gsfc.nasa.gov/reanalysis/MERRA-Land}{MERRA 2}
&
\sphinxAtStartPar
1980\sphinxhyphen{}2016
&
\sphinxAtStartPar
0.5° x 0.625°
&
\sphinxAtStartPar
Global
&
\sphinxAtStartPar
\sphinxhref{https://creativecommons.org/publicdomain/zero/1.0}{Public Domain}
\\
\sphinxhline
\sphinxAtStartPar
\sphinxhref{https://www.ecmwf.int/en/forecasts/datasets/reanalysis-datasets/era-interim-land}{ERA I}
&
\sphinxAtStartPar
1979\sphinxhyphen{}2016
&
\sphinxAtStartPar
0.75° x 0.75°
&
\sphinxAtStartPar
Global
&
\sphinxAtStartPar
\sphinxhref{https://creativecommons.org/publicdomain/zero/1.0}{Public Domain}
\\
\sphinxbottomrule
\end{tabulary}
\sphinxtableafterendhook\par
\sphinxattableend\end{savenotes}


\section{Precipitation}
\label{\detokenize{Introduction/Introduction:precipitation}}

\begin{savenotes}\sphinxattablestart
\sphinxthistablewithglobalstyle
\centering
\begin{tabulary}{\linewidth}[t]{|T|T|T|T|T|}
\sphinxtoprule
\sphinxstyletheadfamily 
\sphinxAtStartPar
Sensor/Dataset
&\sphinxstyletheadfamily 
\sphinxAtStartPar
Temporal
&\sphinxstyletheadfamily 
\sphinxAtStartPar
Spatial
&\sphinxstyletheadfamily 
\sphinxAtStartPar
Extent
&\sphinxstyletheadfamily 
\sphinxAtStartPar
License
\\
\sphinxmidrule
\sphinxtableatstartofbodyhook
\sphinxAtStartPar
\sphinxhref{https://www.esrl.noaa.gov/psd/data/gridded/data.gpcp.html}{GPCP v2.3 1 month}
&
\sphinxAtStartPar
1979\sphinxhyphen{}2019
&
\sphinxAtStartPar
2.5° x 2.5°
&
\sphinxAtStartPar
Global
&
\sphinxAtStartPar
\sphinxhref{https://creativecommons.org/publicdomain/zero/1.0}{Public Domain}
\\
\sphinxhline
\sphinxAtStartPar
\sphinxhref{https://www.esrl.noaa.gov/psd/data/gridded/data.gpcc.html}{GPCC V6}
&
\sphinxAtStartPar
1891\sphinxhyphen{}2019
&
\sphinxAtStartPar
1° x 1°
&
\sphinxAtStartPar
Global
&
\sphinxAtStartPar
\sphinxhref{https://creativecommons.org/publicdomain/zero/1.0}{Public Domain}
\\
\sphinxhline
\sphinxAtStartPar
\sphinxhref{http://chg.geog.ucsb.edu/data/chirps}{CHIRPS}
&
\sphinxAtStartPar
1981\sphinxhyphen{}2016
&
\sphinxAtStartPar
5 km
&
\sphinxAtStartPar
50N\sphinxhyphen{}50S
&
\sphinxAtStartPar
\sphinxhref{https://creativecommons.org/publicdomain/zero/1.0}{Public Domain}
\\
\sphinxhline
\sphinxAtStartPar
\sphinxhref{http://chrsdata.eng.uci.edu}{PERSIANN\sphinxhyphen{}CDR}
&
\sphinxAtStartPar
1983\sphinxhyphen{}2015
&
\sphinxAtStartPar
25 km
&
\sphinxAtStartPar
60N\sphinxhyphen{}60S
&
\sphinxAtStartPar
\sphinxhref{https://creativecommons.org/publicdomain/zero/1.0}{Public Domain}
\\
\sphinxbottomrule
\end{tabulary}
\sphinxtableafterendhook\par
\sphinxattableend\end{savenotes}


\section{Evapotranspiration}
\label{\detokenize{Introduction/Introduction:evapotranspiration}}

\begin{savenotes}\sphinxattablestart
\sphinxthistablewithglobalstyle
\centering
\begin{tabulary}{\linewidth}[t]{|T|T|T|T|T|}
\sphinxtoprule
\sphinxstyletheadfamily 
\sphinxAtStartPar
Sensor/Dataset
&\sphinxstyletheadfamily 
\sphinxAtStartPar
Temporal
&\sphinxstyletheadfamily 
\sphinxAtStartPar
Spatial
&\sphinxstyletheadfamily 
\sphinxAtStartPar
Extent
&\sphinxstyletheadfamily 
\sphinxAtStartPar
License
\\
\sphinxmidrule
\sphinxtableatstartofbodyhook
\sphinxAtStartPar
\sphinxhref{https://lpdaac.usgs.gov/dataset\_discovery/modis/modis\_products\_table/mod16a2\_v006}{MOD16A2}
&
\sphinxAtStartPar
2000\sphinxhyphen{}2014
&
\sphinxAtStartPar
1 km
&
\sphinxAtStartPar
Global
&
\sphinxAtStartPar
\sphinxhref{https://creativecommons.org/publicdomain/zero/1.0}{Public Domain}
\\
\sphinxbottomrule
\end{tabulary}
\sphinxtableafterendhook\par
\sphinxattableend\end{savenotes}


\section{Land cover}
\label{\detokenize{Introduction/Introduction:land-cover}}

\begin{savenotes}\sphinxattablestart
\sphinxthistablewithglobalstyle
\centering
\begin{tabulary}{\linewidth}[t]{|T|T|T|T|T|}
\sphinxtoprule
\sphinxstyletheadfamily 
\sphinxAtStartPar
Sensor/Dataset
&\sphinxstyletheadfamily 
\sphinxAtStartPar
Temporal
&\sphinxstyletheadfamily 
\sphinxAtStartPar
Spatial
&\sphinxstyletheadfamily 
\sphinxAtStartPar
Extent
&\sphinxstyletheadfamily 
\sphinxAtStartPar
License
\\
\sphinxmidrule
\sphinxtableatstartofbodyhook
\sphinxAtStartPar
\sphinxhref{https://www.esa-landcover-cci.org}{ESA CCI Land Cover}
&
\sphinxAtStartPar
1992\sphinxhyphen{}2018
&
\sphinxAtStartPar
300 m
&
\sphinxAtStartPar
Global
&
\sphinxAtStartPar
\sphinxhref{https://creativecommons.org/licenses/by-sa/3.0/igo}{CC by\sphinxhyphen{}SA 3.0}
\\
\sphinxbottomrule
\end{tabulary}
\sphinxtableafterendhook\par
\sphinxattableend\end{savenotes}


\section{Soil carbon}
\label{\detokenize{Introduction/Introduction:soil-carbon}}

\begin{savenotes}\sphinxattablestart
\sphinxthistablewithglobalstyle
\centering
\begin{tabulary}{\linewidth}[t]{|T|T|T|T|T|}
\sphinxtoprule
\sphinxstyletheadfamily 
\sphinxAtStartPar
Sensor/Dataset
&\sphinxstyletheadfamily 
\sphinxAtStartPar
Temporal
&\sphinxstyletheadfamily 
\sphinxAtStartPar
Spatial
&\sphinxstyletheadfamily 
\sphinxAtStartPar
Extent
&\sphinxstyletheadfamily 
\sphinxAtStartPar
License
\\
\sphinxmidrule
\sphinxtableatstartofbodyhook
\sphinxAtStartPar
\sphinxhref{https://www.soilgrids.org/}{Soil Grids (ISRIC)}
&
\sphinxAtStartPar
Present
&
\sphinxAtStartPar
250 m
&
\sphinxAtStartPar
Global
&
\sphinxAtStartPar
\sphinxhref{https://creativecommons.org/licenses/by-sa/4.0}{CC by\sphinxhyphen{}SA 4.0}
\\
\sphinxbottomrule
\end{tabulary}
\sphinxtableafterendhook\par
\sphinxattableend\end{savenotes}


\section{Agroecological Zones}
\label{\detokenize{Introduction/Introduction:agroecological-zones}}

\begin{savenotes}\sphinxattablestart
\sphinxthistablewithglobalstyle
\centering
\begin{tabulary}{\linewidth}[t]{|T|T|T|T|T|}
\sphinxtoprule
\sphinxstyletheadfamily 
\sphinxAtStartPar
Sensor/Dataset
&\sphinxstyletheadfamily 
\sphinxAtStartPar
Temporal
&\sphinxstyletheadfamily 
\sphinxAtStartPar
Spatial
&\sphinxstyletheadfamily 
\sphinxAtStartPar
Extent
&\sphinxstyletheadfamily 
\sphinxAtStartPar
License
\\
\sphinxmidrule
\sphinxtableatstartofbodyhook
\sphinxAtStartPar
\sphinxhref{http://www.fao.org/nr/gaez/en}{FAO \sphinxhyphen{} IIASA Global Agroecological Zones (GAEZ)}
&
\sphinxAtStartPar
2000
&
\sphinxAtStartPar
8 km
&
\sphinxAtStartPar
Global
&
\sphinxAtStartPar
\sphinxhref{https://creativecommons.org/publicdomain/zero/1.0}{Public Domain}
\\
\sphinxbottomrule
\end{tabulary}
\sphinxtableafterendhook\par
\sphinxattableend\end{savenotes}


\section{Soil Quality}
\label{\detokenize{Introduction/Introduction:soil-quality}}

\begin{savenotes}\sphinxattablestart
\sphinxthistablewithglobalstyle
\centering
\begin{tabulary}{\linewidth}[t]{|T|T|T|T|T|}
\sphinxtoprule
\sphinxstyletheadfamily 
\sphinxAtStartPar
Sensor/Dataset
&\sphinxstyletheadfamily 
\sphinxAtStartPar
Temporal
&\sphinxstyletheadfamily 
\sphinxAtStartPar
Spatial
&\sphinxstyletheadfamily 
\sphinxAtStartPar
Extent
&\sphinxstyletheadfamily 
\sphinxAtStartPar
License
\\
\sphinxmidrule
\sphinxtableatstartofbodyhook
\sphinxAtStartPar
\sphinxhref{https://cmr.earthdata.nasa.gov/search/concepts/C1000000240-LPDAAC\_ECS.html}{Soil Texture and Depth}
&
\sphinxAtStartPar
Present
&
\sphinxAtStartPar
250 m
&
\sphinxAtStartPar
Global
&
\sphinxAtStartPar
\sphinxhref{https://creativecommons.org/licenses/by-sa/4.0}{CC by\sphinxhyphen{}SA 4.0}
\\
\sphinxhline
\sphinxAtStartPar
\sphinxhref{https://doi.pangaea.de/10.1594/PANGAEA.788537}{Parent Material}
&
\sphinxAtStartPar
Present
&
\sphinxAtStartPar
N/A
&
\sphinxAtStartPar
Global
&
\sphinxAtStartPar
\sphinxhref{https://creativecommons.org/licenses/by-sa/4.0}{CC by\sphinxhyphen{}SA 4.0}
\\
\sphinxhline
\sphinxAtStartPar
\sphinxhref{https://developers.google.com/earth-engine/datasets/catalog/OpenLandMap\_SOL\_SOL\_TEXTURE-CLASS\_USDA-TT\_M\_v02}{Slope}
&
\sphinxAtStartPar
Present
&
\sphinxAtStartPar
30 m
&\sphinxstartmulticolumn{2}%
\begin{varwidth}[t]{\sphinxcolwidth{2}{5}}
\sphinxAtStartPar
Global  | \sphinxhref{https://www.jpl.nasa.gov/imagepolicy/}{JPL public}
\par
\vskip-\baselineskip\vbox{\hbox{\strut}}\end{varwidth}%
\sphinxstopmulticolumn
\\
\sphinxbottomrule
\end{tabulary}
\sphinxtableafterendhook\par
\sphinxattableend\end{savenotes}


\section{Climate}
\label{\detokenize{Introduction/Introduction:climate}}

\begin{savenotes}\sphinxattablestart
\sphinxthistablewithglobalstyle
\centering
\begin{tabulary}{\linewidth}[t]{|T|T|T|T|T|}
\sphinxtoprule
\sphinxstyletheadfamily 
\sphinxAtStartPar
Sensor/Dataset
&\sphinxstyletheadfamily 
\sphinxAtStartPar
Temporal
&\sphinxstyletheadfamily 
\sphinxAtStartPar
Spatial
&\sphinxstyletheadfamily 
\sphinxAtStartPar
Extent
&\sphinxstyletheadfamily 
\sphinxAtStartPar
License
\\
\sphinxmidrule
\sphinxtableatstartofbodyhook
\sphinxAtStartPar
\sphinxhref{https://developers.google.com/earth-engine/datasets/catalog/IDAHO\_EPSCOR\_TERRACLIMATE\#description}{Terra Climate}
&
\sphinxAtStartPar
1985\sphinxhyphen{}2019
&
\sphinxAtStartPar
30 m
&
\sphinxAtStartPar
Global
&
\sphinxAtStartPar
\sphinxhref{https://creativecommons.org/publicdomain/zero/1.0}{Public Domain}
\\
\sphinxbottomrule
\end{tabulary}
\sphinxtableafterendhook\par
\sphinxattableend\end{savenotes}


\section{Administrative Boundaries}
\label{\detokenize{Introduction/Introduction:administrative-boundaries}}

\begin{savenotes}\sphinxattablestart
\sphinxthistablewithglobalstyle
\centering
\begin{tabulary}{\linewidth}[t]{|T|T|T|T|T|}
\sphinxtoprule
\sphinxstyletheadfamily 
\sphinxAtStartPar
Sensor/Dataset
&\sphinxstyletheadfamily 
\sphinxAtStartPar
Temporal
&\sphinxstyletheadfamily 
\sphinxAtStartPar
Spatial
&\sphinxstyletheadfamily 
\sphinxAtStartPar
Extent
&\sphinxstyletheadfamily 
\sphinxAtStartPar
License
\\
\sphinxmidrule
\sphinxtableatstartofbodyhook
\sphinxAtStartPar
\sphinxhref{http://www.naturalearthdata.com}{Natural Earth Administrative Boundaries}
&
\sphinxAtStartPar
Present
&
\sphinxAtStartPar
10/50m
&
\sphinxAtStartPar
Global
&
\sphinxAtStartPar
\sphinxhref{https://creativecommons.org/publicdomain/zero/1.0}{Public Domain}
\\
\sphinxbottomrule
\end{tabulary}
\sphinxtableafterendhook\par
\sphinxattableend\end{savenotes}

\begin{sphinxadmonition}{note}{Note:}
\sphinxAtStartPar
The \sphinxhref{http://www.naturalearthdata.com}{Natural Earth Administrative Boundaries} provided in MISLAND\sphinxhyphen{}North Africa
are in the \sphinxhref{https://creativecommons.org/publicdomain/zero/1.0}{public domain}. The boundaries and names used, and the
designations used, in MISLAND\sphinxhyphen{}North Africa do not imply official endorsement or
acceptance by Conservation International Foundation, or by its partner
organizations and contributors.

\sphinxAtStartPar
If using MISLAND\sphinxhyphen{}North Africa for official purposes, it is recommended that users
choose an official boundary provided by the designated office of their
country.
\end{sphinxadmonition}

\sphinxstepscope


\chapter{Vegetation Indices}
\label{\detokenize{Preprocessing/NDVI:vegetation-indices}}\label{\detokenize{Preprocessing/NDVI::doc}}

\section{MODIS NDVI}
\label{\detokenize{Preprocessing/NDVI:modis-ndvi}}

\subsection{Data download from Google Earth Engine}
\label{\detokenize{Preprocessing/NDVI:data-download-from-google-earth-engine}}
\sphinxAtStartPar
The MOD13Q1 V6 product provides a Vegetation Index (VI) value at a per pixel basis. There are two primary vegetation layers. The first is the Normalized Difference Vegetation Index (NDVI) which is referred to as the continuity index to the existing National Oceanic and Atmospheric Administration\sphinxhyphen{}Advanced Very High Resolution Radiometer (NOAA\sphinxhyphen{}AVHRR) derived NDVI. The second vegetation layer is the Enhanced Vegetation Index (EVI) that minimizes canopy background variations and maintains sensitivity over dense vegetation conditions. The MODIS NDVI and EVI products are computed from atmospherically corrected bi\sphinxhyphen{}directional surface reflectances that have been masked for water, clouds, heavy aerosols, and cloud shadows.

\sphinxAtStartPar
To compute and download the mean anual MODIS NDVI data from google earth engine. Open the \sphinxhref{https://code.earthengine.google.com/}{Google Earth Engine Code} and paste the lines of code provided below

\begin{sphinxVerbatim}[commandchars=\\\{\},numbers=left,firstnumber=1,stepnumber=1]
\PYG{+w}{     }\PYG{c+cm}{/**** Start of imports. If edited, may not auto\PYGZhy{}convert in the playground. ****/}
\PYG{+w}{     }\PYG{k+kd}{var}\PYG{+w}{ }\PYG{n+nx}{table}\PYG{+w}{ }\PYG{o}{=}\PYG{+w}{ }\PYG{n+nx}{ee}\PYG{p}{.}\PYG{n+nx}{FeatureCollection}\PYG{p}{(}\PYG{l+s+s2}{\PYGZdq{}users/derickongeri/NorthAfrica\PYGZdq{}}\PYG{p}{)}\PYG{p}{;}
\PYG{+w}{     }\PYG{c+cm}{/***** End of imports. If edited, may not auto\PYGZhy{}convert in the playground. *****/}
\PYG{+w}{     }\PYG{k+kd}{var}\PYG{+w}{ }\PYG{n+nx}{dataset}\PYG{+w}{ }\PYG{o}{=}\PYG{+w}{ }\PYG{n+nx}{ee}\PYG{p}{.}\PYG{n+nx}{ImageCollection}\PYG{p}{(}\PYG{l+s+s1}{\PYGZsq{}MODIS/006/MOD13Q1\PYGZsq{}}\PYG{p}{)}
\PYG{+w}{                       }\PYG{p}{.}\PYG{n+nx}{filter}\PYG{p}{(}\PYG{n+nx}{ee}\PYG{p}{.}\PYG{n+nx}{Filter}\PYG{p}{.}\PYG{n+nx}{date}\PYG{p}{(}\PYG{l+s+s1}{\PYGZsq{}2018\PYGZhy{}01\PYGZhy{}01\PYGZsq{}}\PYG{p}{,}\PYG{+w}{ }\PYG{l+s+s1}{\PYGZsq{}2019\PYGZhy{}01\PYGZhy{}01\PYGZsq{}}\PYG{p}{)}\PYG{p}{)}
\PYG{+w}{                       }\PYG{p}{.}\PYG{n+nx}{mean}\PYG{p}{(}\PYG{p}{)}
\PYG{+w}{                       }\PYG{p}{.}\PYG{n+nx}{clip}\PYG{p}{(}\PYG{n+nx}{table}\PYG{p}{)}\PYG{p}{;}

\PYG{+w}{     }\PYG{k+kd}{var}\PYG{+w}{ }\PYG{n+nx}{ndvi}\PYG{+w}{ }\PYG{o}{=}\PYG{+w}{ }\PYG{n+nx}{dataset}\PYG{p}{.}\PYG{n+nx}{select}\PYG{p}{(}\PYG{l+s+s1}{\PYGZsq{}NDVI\PYGZsq{}}\PYG{p}{)}\PYG{p}{;}
\PYG{+w}{     }\PYG{k+kd}{var}\PYG{+w}{ }\PYG{n+nx}{ndviVis}\PYG{+w}{ }\PYG{o}{=}\PYG{+w}{ }\PYG{p}{\PYGZob{}}
\PYG{+w}{       }\PYG{n+nx}{min}\PYG{o}{:}\PYG{+w}{ }\PYG{l+m+mf}{0.0}\PYG{p}{,}
\PYG{+w}{       }\PYG{n+nx}{max}\PYG{o}{:}\PYG{+w}{ }\PYG{l+m+mf}{8000.0}\PYG{p}{,}
\PYG{+w}{       }\PYG{n+nx}{palette}\PYG{o}{:}\PYG{+w}{ }\PYG{p}{[}
\PYG{+w}{         }\PYG{l+s+s1}{\PYGZsq{}FFFFFF\PYGZsq{}}\PYG{p}{,}\PYG{+w}{ }\PYG{l+s+s1}{\PYGZsq{}CE7E45\PYGZsq{}}\PYG{p}{,}\PYG{+w}{ }\PYG{l+s+s1}{\PYGZsq{}DF923D\PYGZsq{}}\PYG{p}{,}\PYG{+w}{ }\PYG{l+s+s1}{\PYGZsq{}F1B555\PYGZsq{}}\PYG{p}{,}\PYG{+w}{ }\PYG{l+s+s1}{\PYGZsq{}FCD163\PYGZsq{}}\PYG{p}{,}\PYG{+w}{ }\PYG{l+s+s1}{\PYGZsq{}99B718\PYGZsq{}}\PYG{p}{,}\PYG{+w}{ }\PYG{l+s+s1}{\PYGZsq{}74A901\PYGZsq{}}\PYG{p}{,}
\PYG{+w}{         }\PYG{l+s+s1}{\PYGZsq{}66A000\PYGZsq{}}\PYG{p}{,}\PYG{+w}{ }\PYG{l+s+s1}{\PYGZsq{}529400\PYGZsq{}}\PYG{p}{,}\PYG{+w}{ }\PYG{l+s+s1}{\PYGZsq{}3E8601\PYGZsq{}}\PYG{p}{,}\PYG{+w}{ }\PYG{l+s+s1}{\PYGZsq{}207401\PYGZsq{}}\PYG{p}{,}\PYG{+w}{ }\PYG{l+s+s1}{\PYGZsq{}056201\PYGZsq{}}\PYG{p}{,}\PYG{+w}{ }\PYG{l+s+s1}{\PYGZsq{}004C00\PYGZsq{}}\PYG{p}{,}\PYG{+w}{ }\PYG{l+s+s1}{\PYGZsq{}023B01\PYGZsq{}}\PYG{p}{,}
\PYG{+w}{         }\PYG{l+s+s1}{\PYGZsq{}012E01\PYGZsq{}}\PYG{p}{,}\PYG{+w}{ }\PYG{l+s+s1}{\PYGZsq{}011D01\PYGZsq{}}\PYG{p}{,}\PYG{+w}{ }\PYG{l+s+s1}{\PYGZsq{}011301\PYGZsq{}}
\PYG{+w}{       }\PYG{p}{]}\PYG{p}{,}
\PYG{+w}{     }\PYG{p}{\PYGZcb{}}\PYG{p}{;}
\PYG{+w}{     }\PYG{n+nb}{Map}\PYG{p}{.}\PYG{n+nx}{centerObject}\PYG{p}{(}\PYG{n+nx}{table}\PYG{p}{)}\PYG{p}{;}
\PYG{+w}{     }\PYG{n+nb}{Map}\PYG{p}{.}\PYG{n+nx}{addLayer}\PYG{p}{(}\PYG{n+nx}{ndvi}\PYG{p}{,}\PYG{+w}{ }\PYG{n+nx}{ndviVis}\PYG{p}{,}\PYG{+w}{ }\PYG{l+s+s1}{\PYGZsq{}NDVI\PYGZsq{}}\PYG{p}{)}\PYG{p}{;}

\PYG{+w}{     }\PYG{n+nx}{Export}\PYG{p}{.}\PYG{n+nx}{image}\PYG{p}{.}\PYG{n+nx}{toCloudStorage}\PYG{p}{(}\PYG{p}{\PYGZob{}}
\PYG{+w}{     }\PYG{n+nx}{image}\PYG{o}{:}\PYG{n+nx}{ndvi}\PYG{p}{,}
\PYG{+w}{     }\PYG{n+nx}{description}\PYG{o}{:}\PYG{+w}{ }\PYG{l+s+s1}{\PYGZsq{}NDVI\PYGZsq{}}\PYG{p}{,}
\PYG{+w}{     }\PYG{n+nx}{maxPixels}\PYG{o}{:}\PYG{l+m+mf}{1e13}\PYG{p}{,}
\PYG{+w}{     }\PYG{n+nx}{scale}\PYG{o}{:}\PYG{l+m+mf}{250}\PYG{p}{,}
\PYG{+w}{     }\PYG{n+nx}{bucket}\PYG{o}{:}\PYG{l+s+s1}{\PYGZsq{}oss\PYGZus{}ldms\PYGZus{}vi\PYGZsq{}}\PYG{p}{,}
\PYG{+w}{     }\PYG{n+nx}{region}\PYG{o}{:}\PYG{n+nx}{table}
\PYG{+w}{     }\PYG{p}{\PYGZcb{}}\PYG{p}{)}
\end{sphinxVerbatim}


\section{Landsat derived NDVI, MSAVI, SAVI}
\label{\detokenize{Preprocessing/NDVI:landsat-derived-ndvi-msavi-savi}}

\subsection{Data download from Google Earth Engine}
\label{\detokenize{Preprocessing/NDVI:id1}}
\sphinxAtStartPar
Several annual NDVI composite techniques have been discussed to overcome current Landsat 5 artifacts.
MISLAND uses annual NDVI products based on different percentiles in order to better qualify and quantify the Vegetation Loss Index.

\sphinxAtStartPar
To compute and download desired percentile composites from google earth engine. Open the \sphinxhref{https://code.earthengine.google.com/}{Google Earth Engine Code} and paste the lines of code provided below

\begin{sphinxVerbatim}[commandchars=\\\{\},numbers=left,firstnumber=1,stepnumber=1]
\PYG{+w}{ }\PYG{c+c1}{// Required Data Inputs}
\PYG{+w}{     }\PYG{c+c1}{// ===================}
\PYG{+w}{     }\PYG{c+c1}{// * USGS/NASA\PYGZsq{}s Landsat 4 surface reflectance tier 1 dataset (August 1982 \PYGZhy{} December 1993)}
\PYG{+w}{     }\PYG{c+c1}{// * USGS/NASA\PYGZsq{}s Landsat 5 surface reflectance tier 1 dataset (January 1, 1984 \PYGZhy{} May 5, 2012)}
\PYG{+w}{     }\PYG{c+c1}{// * USGS/NASA\PYGZsq{}s Landsat 7 surface reflectance tier 1 dataset (January 1, 1999 \PYGZhy{} December 31, 2019)}
\PYG{+w}{     }\PYG{c+c1}{// * USGS/NASA\PYGZsq{}s Landsat 8 surface reflectance tier 1 dataset (April 11, 2013 \PYGZhy{} December 31, 2019)}
\PYG{+w}{     }\PYG{c+c1}{// * Study Area Polygon}

\PYG{+w}{     }\PYG{k+kd}{var}\PYG{+w}{ }\PYG{n+nx}{countries}\PYG{+w}{ }\PYG{o}{=}\PYG{+w}{ }\PYG{n+nx}{ee}\PYG{p}{.}\PYG{n+nx}{FeatureCollection}\PYG{p}{(}\PYG{l+s+s2}{\PYGZdq{}USDOS/LSIB\PYGZus{}SIMPLE/2017\PYGZdq{}}\PYG{p}{)}\PYG{p}{;}

\PYG{+w}{     }\PYG{k+kd}{var}\PYG{+w}{ }\PYG{n+nx}{Year}\PYG{o}{=}\PYG{l+s+s1}{\PYGZsq{}2002\PYGZsq{}}


\PYG{+w}{     }\PYG{k+kd}{var}\PYG{+w}{ }\PYG{n+nx}{country}\PYG{+w}{ }\PYG{o}{=}\PYG{+w}{ }\PYG{l+s+s1}{\PYGZsq{}Libya\PYGZsq{}}\PYG{p}{;}
\PYG{+w}{     }\PYG{k+kd}{var}\PYG{+w}{ }\PYG{n+nx}{ALGO}\PYG{+w}{ }\PYG{o}{=}\PYG{+w}{ }\PYG{l+s+s2}{\PYGZdq{}PERCENTILE65\PYGZdq{}}\PYG{p}{;}\PYG{+w}{ }\PYG{c+c1}{// PERCENTILE75  // PERCENTILE65 // PERCENTILE60 // MEDIAN}
\PYG{+w}{     }\PYG{k+kd}{var}\PYG{+w}{ }\PYG{n+nx}{cloudCoveragePercentage}\PYG{+w}{ }\PYG{o}{=}\PYG{+w}{ }\PYG{l+m+mf}{80}\PYG{p}{;}


\PYG{+w}{     }\PYG{k+kd}{var}\PYG{+w}{ }\PYG{n+nx}{studyArea}\PYG{+w}{ }\PYG{o}{=}\PYG{+w}{ }\PYG{n+nx}{countries}\PYG{p}{.}\PYG{n+nx}{filter}\PYG{p}{(}\PYG{n+nx}{ee}\PYG{p}{.}\PYG{n+nx}{Filter}\PYG{p}{.}\PYG{n+nx}{eq}\PYG{p}{(}\PYG{l+s+s1}{\PYGZsq{}country\PYGZus{}na\PYGZsq{}}\PYG{p}{,}\PYG{n+nx}{country}\PYG{+w}{ }\PYG{p}{)}\PYG{p}{)}
\PYG{+w}{     }\PYG{c+c1}{//Map.addLayer(studyArea);}

\PYG{+w}{     }\PYG{c+cm}{/*}
\PYG{c+cm}{     borders are quite coarse}
\PYG{c+cm}{     var northAfrica = ee.FeatureCollection(\PYGZsq{}users/derickongeri/Admin\PYGZsq{})}
\PYG{c+cm}{     // Replace country name with EGYPT, LIBYA, ALGERIA, MAURITANIA, MOROCCO, TUNISIA}
\PYG{c+cm}{     var country = \PYGZsq{}TUNISIA\PYGZsq{};}
\PYG{c+cm}{     var studyArea = northAfrica.filter(ee.Filter.eq(\PYGZsq{}NAME\PYGZsq{}, country))}
\PYG{c+cm}{     Map.addLayer(studyArea);}
\PYG{c+cm}{     */}




\PYG{+w}{     }\PYG{k+kd}{var}\PYG{+w}{ }\PYG{n+nx}{start\PYGZus{}date}\PYG{+w}{ }\PYG{o}{=}\PYG{+w}{ }\PYG{n+nx}{Year}\PYG{o}{+}\PYG{+w}{ }\PYG{l+s+s1}{\PYGZsq{}\PYGZhy{}01\PYGZhy{}01\PYGZsq{}}\PYG{p}{;}
\PYG{+w}{     }\PYG{k+kd}{var}\PYG{+w}{ }\PYG{n+nx}{end\PYGZus{}date}\PYG{+w}{   }\PYG{o}{=}\PYG{+w}{ }\PYG{n+nx}{Year}\PYG{o}{+}\PYG{+w}{ }\PYG{l+s+s1}{\PYGZsq{}\PYGZhy{}12\PYGZhy{}31\PYGZsq{}}\PYG{p}{;}


\PYG{+w}{     }\PYG{c+c1}{//\PYGZhy{}\PYGZhy{}\PYGZhy{}\PYGZhy{}\PYGZhy{}\PYGZhy{}\PYGZhy{}\PYGZhy{}\PYGZhy{}\PYGZhy{}\PYGZhy{}\PYGZhy{}\PYGZhy{}\PYGZhy{}\PYGZhy{}\PYGZhy{}\PYGZhy{}\PYGZhy{}\PYGZhy{}\PYGZhy{}\PYGZhy{}\PYGZhy{}\PYGZhy{}\PYGZhy{}\PYGZhy{}\PYGZhy{}\PYGZhy{}\PYGZhy{}\PYGZhy{}\PYGZhy{}\PYGZhy{}\PYGZhy{}\PYGZhy{}\PYGZhy{}\PYGZhy{}\PYGZhy{}\PYGZhy{}\PYGZhy{}\PYGZhy{}\PYGZhy{}\PYGZhy{}\PYGZhy{}\PYGZhy{}\PYGZhy{}\PYGZhy{}\PYGZhy{}\PYGZhy{}\PYGZhy{}\PYGZhy{}\PYGZhy{}\PYGZhy{}\PYGZhy{}\PYGZhy{}\PYGZhy{}\PYGZhy{}\PYGZhy{}\PYGZhy{}\PYGZhy{}\PYGZhy{}\PYGZhy{}\PYGZhy{}\PYGZhy{}\PYGZhy{}\PYGZhy{}\PYGZhy{}\PYGZhy{}\PYGZhy{}\PYGZhy{}}
\PYG{+w}{     }\PYG{c+c1}{//       Landsat 4, 5, 7 cloudmask}
\PYG{+w}{     }\PYG{c+c1}{//\PYGZhy{}\PYGZhy{}\PYGZhy{}\PYGZhy{}\PYGZhy{}\PYGZhy{}\PYGZhy{}\PYGZhy{}\PYGZhy{}\PYGZhy{}\PYGZhy{}\PYGZhy{}\PYGZhy{}\PYGZhy{}\PYGZhy{}\PYGZhy{}\PYGZhy{}\PYGZhy{}\PYGZhy{}\PYGZhy{}\PYGZhy{}\PYGZhy{}\PYGZhy{}\PYGZhy{}\PYGZhy{}\PYGZhy{}\PYGZhy{}\PYGZhy{}\PYGZhy{}\PYGZhy{}\PYGZhy{}\PYGZhy{}\PYGZhy{}\PYGZhy{}\PYGZhy{}\PYGZhy{}\PYGZhy{}\PYGZhy{}\PYGZhy{}\PYGZhy{}\PYGZhy{}\PYGZhy{}\PYGZhy{}\PYGZhy{}\PYGZhy{}\PYGZhy{}\PYGZhy{}\PYGZhy{}\PYGZhy{}\PYGZhy{}\PYGZhy{}\PYGZhy{}\PYGZhy{}\PYGZhy{}\PYGZhy{}\PYGZhy{}\PYGZhy{}\PYGZhy{}\PYGZhy{}\PYGZhy{}\PYGZhy{}\PYGZhy{}\PYGZhy{}\PYGZhy{}\PYGZhy{}\PYGZhy{}\PYGZhy{}\PYGZhy{}}

\PYG{+w}{         }\PYG{c+c1}{// If the cloud bit (5) is set and the cloud confidence (7) is high}
\PYG{+w}{         }\PYG{c+c1}{// or the cloud shadow bit is set (3), then it\PYGZsq{}s a bad pixel.}
\PYG{+w}{     }\PYG{k+kd}{var}\PYG{+w}{ }\PYG{n+nx}{cloudMaskL7}\PYG{+w}{ }\PYG{o}{=}\PYG{+w}{ }\PYG{k+kd}{function}\PYG{p}{(}\PYG{n+nx}{image}\PYG{p}{)}\PYG{+w}{ }\PYG{p}{\PYGZob{}}
\PYG{+w}{       }\PYG{k+kd}{var}\PYG{+w}{ }\PYG{n+nx}{qa}\PYG{+w}{ }\PYG{o}{=}\PYG{+w}{ }\PYG{n+nx}{image}\PYG{p}{.}\PYG{n+nx}{select}\PYG{p}{(}\PYG{l+s+s1}{\PYGZsq{}pixel\PYGZus{}qa\PYGZsq{}}\PYG{p}{)}\PYG{p}{;}
\PYG{+w}{       }\PYG{k+kd}{var}\PYG{+w}{ }\PYG{n+nx}{cloud}\PYG{+w}{ }\PYG{o}{=}\PYG{+w}{ }\PYG{n+nx}{qa}\PYG{p}{.}\PYG{n+nx}{bitwiseAnd}\PYG{p}{(}\PYG{l+m+mf}{1}\PYG{+w}{ }\PYG{o}{\PYGZlt{}\PYGZlt{}}\PYG{+w}{ }\PYG{l+m+mf}{5}\PYG{p}{)}
\PYG{+w}{                       }\PYG{p}{.}\PYG{n+nx}{and}\PYG{p}{(}\PYG{n+nx}{qa}\PYG{p}{.}\PYG{n+nx}{bitwiseAnd}\PYG{p}{(}\PYG{l+m+mf}{1}\PYG{+w}{ }\PYG{o}{\PYGZlt{}\PYGZlt{}}\PYG{+w}{ }\PYG{l+m+mf}{7}\PYG{p}{)}\PYG{p}{)}
\PYG{+w}{                       }\PYG{p}{.}\PYG{n+nx}{or}\PYG{p}{(}\PYG{n+nx}{qa}\PYG{p}{.}\PYG{n+nx}{bitwiseAnd}\PYG{p}{(}\PYG{l+m+mf}{1}\PYG{+w}{ }\PYG{o}{\PYGZlt{}\PYGZlt{}}\PYG{+w}{ }\PYG{l+m+mf}{3}\PYG{p}{)}\PYG{p}{)}\PYG{p}{;}

\PYG{+w}{         }\PYG{c+c1}{// Remove edge pixels that don\PYGZsq{}t occur in all bands}
\PYG{+w}{       }\PYG{c+c1}{//var mask2 = image.mask().reduce(ee.Reducer.min())//.focal\PYGZus{}min(300,\PYGZsq{}square\PYGZsq{},\PYGZsq{}meters\PYGZsq{}).eq(0);}
\PYG{+w}{       }\PYG{c+c1}{//var mask2 = image.select(\PYGZsq{}B4\PYGZsq{}).reduce(ee.Reducer.min()).gt(0)//.focal\PYGZus{}min(500,\PYGZsq{}square\PYGZsq{},\PYGZsq{}meters\PYGZsq{});}
\PYG{+w}{       }\PYG{c+c1}{// Remove edge pixels that don\PYGZsq{}t occur in all bands}
\PYG{+w}{       }\PYG{k+kd}{var}\PYG{+w}{ }\PYG{n+nx}{mask3}\PYG{+w}{ }\PYG{o}{=}
\PYG{+w}{                   }\PYG{p}{(}\PYG{n+nx}{image}\PYG{p}{.}\PYG{n+nx}{select}\PYG{p}{(}\PYG{l+s+s1}{\PYGZsq{}B3\PYGZsq{}}\PYG{p}{)}\PYG{p}{.}\PYG{n+nx}{gt}\PYG{p}{(}\PYG{l+m+mf}{100}\PYG{p}{)}\PYG{p}{)}
\PYG{+w}{                   }\PYG{p}{.}\PYG{n+nx}{and}\PYG{p}{(}\PYG{n+nx}{image}\PYG{p}{.}\PYG{n+nx}{select}\PYG{p}{(}\PYG{l+s+s1}{\PYGZsq{}B4\PYGZsq{}}\PYG{p}{)}\PYG{p}{.}\PYG{n+nx}{gt}\PYG{p}{(}\PYG{l+m+mf}{100}\PYG{p}{)}\PYG{p}{)}


\PYG{+w}{                   }\PYG{p}{.}\PYG{n+nx}{and}\PYG{p}{(}\PYG{n+nx}{image}\PYG{p}{.}\PYG{n+nx}{select}\PYG{p}{(}\PYG{l+s+s1}{\PYGZsq{}B4\PYGZsq{}}\PYG{p}{)}\PYG{p}{.}\PYG{n+nx}{lt}\PYG{p}{(}\PYG{l+m+mf}{10000}\PYG{p}{)}\PYG{p}{)}
\PYG{+w}{                   }\PYG{p}{.}\PYG{n+nx}{and}\PYG{p}{(}\PYG{n+nx}{image}\PYG{p}{.}\PYG{n+nx}{select}\PYG{p}{(}\PYG{l+s+s1}{\PYGZsq{}B3\PYGZsq{}}\PYG{p}{)}\PYG{p}{.}\PYG{n+nx}{lt}\PYG{p}{(}\PYG{l+m+mf}{10000}\PYG{p}{)}\PYG{p}{)}



\PYG{+w}{       }\PYG{k}{return}\PYG{+w}{ }\PYG{n+nx}{image}\PYG{p}{.}\PYG{n+nx}{updateMask}\PYG{p}{(}\PYG{n+nx}{cloud}\PYG{p}{.}\PYG{n+nx}{not}\PYG{p}{(}\PYG{p}{)}\PYG{p}{)}\PYG{p}{.}\PYG{n+nx}{updateMask}\PYG{p}{(}\PYG{n+nx}{mask3}\PYG{p}{)}\PYG{c+c1}{//.updateMask(mask2)//.clip(image.geometry().buffer(\PYGZhy{}5000))//.or(mask3));}
\PYG{+w}{     }\PYG{p}{\PYGZcb{}}\PYG{p}{;}


\PYG{+w}{     }\PYG{k+kd}{var}\PYG{+w}{ }\PYG{n+nx}{cloudMaskL45}\PYG{+w}{ }\PYG{o}{=}\PYG{+w}{ }\PYG{k+kd}{function}\PYG{p}{(}\PYG{n+nx}{image}\PYG{p}{)}\PYG{+w}{ }\PYG{p}{\PYGZob{}}
\PYG{+w}{       }\PYG{k+kd}{var}\PYG{+w}{ }\PYG{n+nx}{qa}\PYG{+w}{ }\PYG{o}{=}\PYG{+w}{ }\PYG{n+nx}{image}\PYG{p}{.}\PYG{n+nx}{select}\PYG{p}{(}\PYG{l+s+s1}{\PYGZsq{}pixel\PYGZus{}qa\PYGZsq{}}\PYG{p}{)}\PYG{p}{;}
\PYG{+w}{       }\PYG{k+kd}{var}\PYG{+w}{ }\PYG{n+nx}{cloud}\PYG{+w}{ }\PYG{o}{=}\PYG{+w}{ }\PYG{n+nx}{qa}\PYG{p}{.}\PYG{n+nx}{bitwiseAnd}\PYG{p}{(}\PYG{l+m+mf}{1}\PYG{+w}{ }\PYG{o}{\PYGZlt{}\PYGZlt{}}\PYG{+w}{ }\PYG{l+m+mf}{5}\PYG{p}{)}
\PYG{+w}{                       }\PYG{p}{.}\PYG{n+nx}{and}\PYG{p}{(}\PYG{n+nx}{qa}\PYG{p}{.}\PYG{n+nx}{bitwiseAnd}\PYG{p}{(}\PYG{l+m+mf}{1}\PYG{+w}{ }\PYG{o}{\PYGZlt{}\PYGZlt{}}\PYG{+w}{ }\PYG{l+m+mf}{7}\PYG{p}{)}\PYG{p}{)}
\PYG{+w}{                       }\PYG{p}{.}\PYG{n+nx}{or}\PYG{p}{(}\PYG{n+nx}{qa}\PYG{p}{.}\PYG{n+nx}{bitwiseAnd}\PYG{p}{(}\PYG{l+m+mf}{1}\PYG{+w}{ }\PYG{o}{\PYGZlt{}\PYGZlt{}}\PYG{+w}{ }\PYG{l+m+mf}{3}\PYG{p}{)}\PYG{p}{)}\PYG{p}{;}

\PYG{+w}{       }\PYG{c+c1}{// Remove edge pixels that don\PYGZsq{}t occur in all bands}
\PYG{+w}{       }\PYG{c+c1}{//var mask2 = image.mask().reduce(ee.Reducer.min());}
\PYG{+w}{         }\PYG{k+kd}{var}\PYG{+w}{ }\PYG{n+nx}{mask2}\PYG{+w}{ }\PYG{o}{=}
\PYG{+w}{                   }\PYG{p}{(}\PYG{n+nx}{image}\PYG{p}{.}\PYG{n+nx}{select}\PYG{p}{(}\PYG{l+s+s1}{\PYGZsq{}B3\PYGZsq{}}\PYG{p}{)}\PYG{p}{.}\PYG{n+nx}{gt}\PYG{p}{(}\PYG{l+m+mf}{100}\PYG{p}{)}\PYG{p}{)}
\PYG{+w}{                   }\PYG{p}{.}\PYG{n+nx}{and}\PYG{p}{(}\PYG{n+nx}{image}\PYG{p}{.}\PYG{n+nx}{select}\PYG{p}{(}\PYG{l+s+s1}{\PYGZsq{}B4\PYGZsq{}}\PYG{p}{)}\PYG{p}{.}\PYG{n+nx}{gt}\PYG{p}{(}\PYG{l+m+mf}{100}\PYG{p}{)}\PYG{p}{)}


\PYG{+w}{                   }\PYG{p}{.}\PYG{n+nx}{and}\PYG{p}{(}\PYG{n+nx}{image}\PYG{p}{.}\PYG{n+nx}{select}\PYG{p}{(}\PYG{l+s+s1}{\PYGZsq{}B4\PYGZsq{}}\PYG{p}{)}\PYG{p}{.}\PYG{n+nx}{lt}\PYG{p}{(}\PYG{l+m+mf}{10000}\PYG{p}{)}\PYG{p}{)}
\PYG{+w}{                   }\PYG{p}{.}\PYG{n+nx}{and}\PYG{p}{(}\PYG{n+nx}{image}\PYG{p}{.}\PYG{n+nx}{select}\PYG{p}{(}\PYG{l+s+s1}{\PYGZsq{}B3\PYGZsq{}}\PYG{p}{)}\PYG{p}{.}\PYG{n+nx}{lt}\PYG{p}{(}\PYG{l+m+mf}{10000}\PYG{p}{)}\PYG{p}{)}

\PYG{+w}{       }\PYG{k}{return}\PYG{+w}{ }\PYG{p}{(}\PYG{n+nx}{image}\PYG{p}{.}\PYG{n+nx}{updateMask}\PYG{p}{(}\PYG{n+nx}{cloud}\PYG{p}{.}\PYG{n+nx}{not}\PYG{p}{(}\PYG{p}{)}\PYG{p}{)}\PYG{p}{.}\PYG{n+nx}{updateMask}\PYG{p}{(}\PYG{n+nx}{mask2}\PYG{p}{)}\PYG{p}{)}\PYG{c+c1}{//.clip(image.geometry().buffer(\PYGZhy{}5000))//.updateMask(mask2);}
\PYG{+w}{     }\PYG{p}{\PYGZcb{}}\PYG{p}{;}


\PYG{+w}{     }\PYG{c+c1}{//\PYGZhy{}\PYGZhy{}\PYGZhy{}\PYGZhy{}\PYGZhy{}\PYGZhy{}\PYGZhy{}\PYGZhy{}\PYGZhy{}\PYGZhy{}\PYGZhy{}\PYGZhy{}\PYGZhy{}\PYGZhy{}\PYGZhy{}\PYGZhy{}\PYGZhy{}\PYGZhy{}\PYGZhy{}\PYGZhy{}\PYGZhy{}\PYGZhy{}\PYGZhy{}\PYGZhy{}\PYGZhy{}\PYGZhy{}\PYGZhy{}\PYGZhy{}\PYGZhy{}\PYGZhy{}\PYGZhy{}\PYGZhy{}\PYGZhy{}\PYGZhy{}\PYGZhy{}\PYGZhy{}\PYGZhy{}\PYGZhy{}\PYGZhy{}\PYGZhy{}\PYGZhy{}\PYGZhy{}\PYGZhy{}\PYGZhy{}\PYGZhy{}\PYGZhy{}\PYGZhy{}\PYGZhy{}\PYGZhy{}\PYGZhy{}\PYGZhy{}\PYGZhy{}\PYGZhy{}\PYGZhy{}\PYGZhy{}\PYGZhy{}\PYGZhy{}\PYGZhy{}\PYGZhy{}\PYGZhy{}\PYGZhy{}\PYGZhy{}\PYGZhy{}\PYGZhy{}\PYGZhy{}\PYGZhy{}\PYGZhy{}\PYGZhy{}}
\PYG{+w}{     }\PYG{c+c1}{//         Landsat 8 cloudmask}
\PYG{+w}{     }\PYG{c+c1}{//\PYGZhy{}\PYGZhy{}\PYGZhy{}\PYGZhy{}\PYGZhy{}\PYGZhy{}\PYGZhy{}\PYGZhy{}\PYGZhy{}\PYGZhy{}\PYGZhy{}\PYGZhy{}\PYGZhy{}\PYGZhy{}\PYGZhy{}\PYGZhy{}\PYGZhy{}\PYGZhy{}\PYGZhy{}\PYGZhy{}\PYGZhy{}\PYGZhy{}\PYGZhy{}\PYGZhy{}\PYGZhy{}\PYGZhy{}\PYGZhy{}\PYGZhy{}\PYGZhy{}\PYGZhy{}\PYGZhy{}\PYGZhy{}\PYGZhy{}\PYGZhy{}\PYGZhy{}\PYGZhy{}\PYGZhy{}\PYGZhy{}\PYGZhy{}\PYGZhy{}\PYGZhy{}\PYGZhy{}\PYGZhy{}\PYGZhy{}\PYGZhy{}\PYGZhy{}\PYGZhy{}\PYGZhy{}\PYGZhy{}\PYGZhy{}\PYGZhy{}\PYGZhy{}\PYGZhy{}\PYGZhy{}\PYGZhy{}\PYGZhy{}\PYGZhy{}\PYGZhy{}\PYGZhy{}\PYGZhy{}\PYGZhy{}\PYGZhy{}\PYGZhy{}\PYGZhy{}\PYGZhy{}\PYGZhy{}\PYGZhy{}\PYGZhy{}}

\PYG{+w}{         }\PYG{c+c1}{// Bits 3 and 5 are cloud shadow and cloud, respectively.}
\PYG{+w}{     }\PYG{k+kd}{function}\PYG{+w}{ }\PYG{n+nx}{maskL8sr}\PYG{p}{(}\PYG{n+nx}{image}\PYG{p}{)}\PYG{+w}{ }\PYG{p}{\PYGZob{}}
\PYG{+w}{       }\PYG{k+kd}{var}\PYG{+w}{ }\PYG{n+nx}{cloudShadowBitMask}\PYG{+w}{ }\PYG{o}{=}\PYG{+w}{ }\PYG{p}{(}\PYG{l+m+mf}{1}\PYG{+w}{ }\PYG{o}{\PYGZlt{}\PYGZlt{}}\PYG{+w}{ }\PYG{l+m+mf}{3}\PYG{p}{)}\PYG{p}{;}
\PYG{+w}{       }\PYG{k+kd}{var}\PYG{+w}{ }\PYG{n+nx}{cloudsBitMask}\PYG{+w}{ }\PYG{o}{=}\PYG{+w}{ }\PYG{p}{(}\PYG{l+m+mf}{1}\PYG{+w}{ }\PYG{o}{\PYGZlt{}\PYGZlt{}}\PYG{+w}{ }\PYG{l+m+mf}{5}\PYG{p}{)}\PYG{p}{;}

\PYG{+w}{         }\PYG{c+c1}{// Get the pixel QA band.}
\PYG{+w}{       }\PYG{k+kd}{var}\PYG{+w}{ }\PYG{n+nx}{qa}\PYG{+w}{ }\PYG{o}{=}\PYG{+w}{ }\PYG{n+nx}{image}\PYG{p}{.}\PYG{n+nx}{select}\PYG{p}{(}\PYG{l+s+s1}{\PYGZsq{}pixel\PYGZus{}qa\PYGZsq{}}\PYG{p}{)}\PYG{p}{;}

\PYG{+w}{         }\PYG{c+c1}{// Both flags should be set to zero, indicating clear conditions.}
\PYG{+w}{       }\PYG{k+kd}{var}\PYG{+w}{ }\PYG{n+nx}{mask}\PYG{+w}{ }\PYG{o}{=}\PYG{+w}{ }\PYG{n+nx}{qa}\PYG{p}{.}\PYG{n+nx}{bitwiseAnd}\PYG{p}{(}\PYG{n+nx}{cloudShadowBitMask}\PYG{p}{)}\PYG{p}{.}\PYG{n+nx}{eq}\PYG{p}{(}\PYG{l+m+mf}{0}\PYG{p}{)}
\PYG{+w}{                      }\PYG{p}{.}\PYG{n+nx}{and}\PYG{p}{(}\PYG{n+nx}{qa}\PYG{p}{.}\PYG{n+nx}{bitwiseAnd}\PYG{p}{(}\PYG{n+nx}{cloudsBitMask}\PYG{p}{)}\PYG{p}{.}\PYG{n+nx}{eq}\PYG{p}{(}\PYG{l+m+mf}{0}\PYG{p}{)}\PYG{p}{)}\PYG{p}{;}
\PYG{+w}{       }\PYG{k+kd}{var}\PYG{+w}{ }\PYG{n+nx}{mask2}\PYG{+w}{ }\PYG{o}{=}

\PYG{+w}{                   }\PYG{p}{(}\PYG{n+nx}{image}\PYG{p}{.}\PYG{n+nx}{select}\PYG{p}{(}\PYG{l+s+s1}{\PYGZsq{}B5\PYGZsq{}}\PYG{p}{)}\PYG{p}{.}\PYG{n+nx}{gt}\PYG{p}{(}\PYG{l+m+mf}{100}\PYG{p}{)}\PYG{p}{)}
\PYG{+w}{                   }\PYG{p}{.}\PYG{n+nx}{and}\PYG{p}{(}\PYG{n+nx}{image}\PYG{p}{.}\PYG{n+nx}{select}\PYG{p}{(}\PYG{l+s+s1}{\PYGZsq{}B4\PYGZsq{}}\PYG{p}{)}\PYG{p}{.}\PYG{n+nx}{gt}\PYG{p}{(}\PYG{l+m+mf}{100}\PYG{p}{)}\PYG{p}{)}


\PYG{+w}{                   }\PYG{p}{.}\PYG{n+nx}{and}\PYG{p}{(}\PYG{n+nx}{image}\PYG{p}{.}\PYG{n+nx}{select}\PYG{p}{(}\PYG{l+s+s1}{\PYGZsq{}B5\PYGZsq{}}\PYG{p}{)}\PYG{p}{.}\PYG{n+nx}{lt}\PYG{p}{(}\PYG{l+m+mf}{10000}\PYG{p}{)}\PYG{p}{)}
\PYG{+w}{                   }\PYG{p}{.}\PYG{n+nx}{and}\PYG{p}{(}\PYG{n+nx}{image}\PYG{p}{.}\PYG{n+nx}{select}\PYG{p}{(}\PYG{l+s+s1}{\PYGZsq{}B4\PYGZsq{}}\PYG{p}{)}\PYG{p}{.}\PYG{n+nx}{lt}\PYG{p}{(}\PYG{l+m+mf}{10000}\PYG{p}{)}\PYG{p}{)}

\PYG{+w}{        }\PYG{c+c1}{//var mask2 = image.mask().reduce(ee.Reducer.min()).focal\PYGZus{}min(500,\PYGZsq{}square\PYGZsq{},\PYGZsq{}meters\PYGZsq{});}
\PYG{+w}{       }\PYG{c+c1}{//return image}
\PYG{+w}{       }\PYG{k}{return}\PYG{+w}{ }\PYG{n+nx}{image}\PYG{p}{.}\PYG{n+nx}{updateMask}\PYG{p}{(}\PYG{n+nx}{mask}\PYG{p}{)}\PYG{p}{.}\PYG{n+nx}{updateMask}\PYG{p}{(}\PYG{n+nx}{mask2}\PYG{p}{)}\PYG{c+c1}{//.clip(image.geometry().buffer(\PYGZhy{}5000));}
\PYG{+w}{     }\PYG{p}{\PYGZcb{}}




\PYG{+w}{         }\PYG{c+c1}{// Apply Cloudmask to L4.5.7}
\PYG{+w}{     }\PYG{k+kd}{var}\PYG{+w}{ }\PYG{n+nx}{L4}\PYG{+w}{ }\PYG{o}{=}\PYG{+w}{ }\PYG{n+nx}{ee}\PYG{p}{.}\PYG{n+nx}{ImageCollection}\PYG{p}{(}\PYG{l+s+s2}{\PYGZdq{}LANDSAT/LT04/C01/T1\PYGZus{}SR\PYGZdq{}}\PYG{p}{)}
\PYG{+w}{                       }\PYG{p}{.}\PYG{n+nx}{filterDate}\PYG{p}{(}\PYG{n+nx}{start\PYGZus{}date}\PYG{p}{,}\PYG{+w}{ }\PYG{n+nx}{end\PYGZus{}date}\PYG{p}{)}
\PYG{+w}{                       }\PYG{p}{.}\PYG{n+nx}{filter}\PYG{p}{(}\PYG{n+nx}{ee}\PYG{p}{.}\PYG{n+nx}{Filter}\PYG{p}{.}\PYG{n+nx}{lessThan}\PYG{p}{(}\PYG{l+s+s1}{\PYGZsq{}CLOUD\PYGZus{}COVER\PYGZus{}LAND\PYGZsq{}}\PYG{p}{,}\PYG{+w}{ }\PYG{n+nx}{cloudCoveragePercentage}\PYG{p}{)}\PYG{p}{)}
\PYG{+w}{                       }\PYG{p}{.}\PYG{n+nx}{filterBounds}\PYG{p}{(}\PYG{n+nx}{studyArea}\PYG{p}{)}
\PYG{+w}{                       }\PYG{p}{.}\PYG{n+nx}{map}\PYG{p}{(}\PYG{n+nx}{cloudMaskL45}\PYG{p}{)}
\PYG{+w}{                       }\PYG{p}{.}\PYG{n+nx}{select}\PYG{p}{(}\PYG{p}{[}\PYG{l+s+s1}{\PYGZsq{}B3\PYGZsq{}}\PYG{p}{,}\PYG{+w}{ }\PYG{l+s+s1}{\PYGZsq{}B4\PYGZsq{}}\PYG{p}{]}\PYG{p}{,}\PYG{+w}{ }\PYG{p}{[}\PYG{l+s+s1}{\PYGZsq{}RED\PYGZsq{}}\PYG{p}{,}\PYG{+w}{ }\PYG{l+s+s1}{\PYGZsq{}NIR\PYGZsq{}}\PYG{p}{]}\PYG{p}{)}\PYG{p}{;}\PYG{p}{;}

\PYG{+w}{     }\PYG{k+kd}{var}\PYG{+w}{ }\PYG{n+nx}{L5}\PYG{+w}{ }\PYG{o}{=}\PYG{+w}{ }\PYG{n+nx}{ee}\PYG{p}{.}\PYG{n+nx}{ImageCollection}\PYG{p}{(}\PYG{l+s+s1}{\PYGZsq{}LANDSAT/LT05/C01/T1\PYGZus{}SR\PYGZsq{}}\PYG{p}{)}
\PYG{+w}{                       }\PYG{p}{.}\PYG{n+nx}{filterDate}\PYG{p}{(}\PYG{n+nx}{start\PYGZus{}date}\PYG{p}{,}\PYG{+w}{ }\PYG{n+nx}{end\PYGZus{}date}\PYG{p}{)}
\PYG{+w}{                       }\PYG{p}{.}\PYG{n+nx}{filter}\PYG{p}{(}\PYG{n+nx}{ee}\PYG{p}{.}\PYG{n+nx}{Filter}\PYG{p}{.}\PYG{n+nx}{lessThan}\PYG{p}{(}\PYG{l+s+s1}{\PYGZsq{}CLOUD\PYGZus{}COVER\PYGZus{}LAND\PYGZsq{}}\PYG{p}{,}\PYG{+w}{ }\PYG{n+nx}{cloudCoveragePercentage}\PYG{p}{)}\PYG{p}{)}
\PYG{+w}{                       }\PYG{p}{.}\PYG{n+nx}{filterBounds}\PYG{p}{(}\PYG{n+nx}{studyArea}\PYG{p}{)}
\PYG{+w}{                       }\PYG{p}{.}\PYG{n+nx}{map}\PYG{p}{(}\PYG{n+nx}{cloudMaskL45}\PYG{p}{)}
\PYG{+w}{                       }\PYG{p}{.}\PYG{n+nx}{select}\PYG{p}{(}\PYG{p}{[}\PYG{l+s+s1}{\PYGZsq{}B3\PYGZsq{}}\PYG{p}{,}\PYG{+w}{ }\PYG{l+s+s1}{\PYGZsq{}B4\PYGZsq{}}\PYG{p}{]}\PYG{p}{,}\PYG{+w}{ }\PYG{p}{[}\PYG{l+s+s1}{\PYGZsq{}RED\PYGZsq{}}\PYG{p}{,}\PYG{+w}{ }\PYG{l+s+s1}{\PYGZsq{}NIR\PYGZsq{}}\PYG{p}{]}\PYG{p}{)}\PYG{p}{;}\PYG{p}{;}

\PYG{+w}{     }\PYG{k+kd}{var}\PYG{+w}{ }\PYG{n+nx}{L7a}\PYG{+w}{ }\PYG{o}{=}\PYG{+w}{ }\PYG{n+nx}{ee}\PYG{p}{.}\PYG{n+nx}{ImageCollection}\PYG{p}{(}\PYG{l+s+s1}{\PYGZsq{}LANDSAT/LE07/C01/T1\PYGZus{}SR\PYGZsq{}}\PYG{p}{)}
\PYG{+w}{                       }\PYG{p}{.}\PYG{n+nx}{filterDate}\PYG{p}{(}\PYG{l+s+s1}{\PYGZsq{}1999\PYGZhy{}01\PYGZhy{}01\PYGZsq{}}\PYG{p}{,}\PYG{+w}{ }\PYG{l+s+s1}{\PYGZsq{}2003\PYGZhy{}04\PYGZhy{}01\PYGZsq{}}\PYG{p}{)}
\PYG{+w}{                       }\PYG{p}{.}\PYG{n+nx}{filterDate}\PYG{p}{(}\PYG{n+nx}{start\PYGZus{}date}\PYG{p}{,}\PYG{+w}{ }\PYG{n+nx}{end\PYGZus{}date}\PYG{p}{)}
\PYG{+w}{                       }\PYG{p}{.}\PYG{n+nx}{filter}\PYG{p}{(}\PYG{n+nx}{ee}\PYG{p}{.}\PYG{n+nx}{Filter}\PYG{p}{.}\PYG{n+nx}{lessThan}\PYG{p}{(}\PYG{l+s+s1}{\PYGZsq{}CLOUD\PYGZus{}COVER\PYGZus{}LAND\PYGZsq{}}\PYG{p}{,}\PYG{+w}{ }\PYG{l+m+mf}{100}\PYG{p}{)}\PYG{p}{)}
\PYG{+w}{                       }\PYG{p}{.}\PYG{n+nx}{filterBounds}\PYG{p}{(}\PYG{n+nx}{studyArea}\PYG{p}{)}
\PYG{+w}{                       }\PYG{p}{.}\PYG{n+nx}{map}\PYG{p}{(}\PYG{n+nx}{cloudMaskL7}\PYG{p}{)}
\PYG{+w}{                       }\PYG{p}{.}\PYG{n+nx}{select}\PYG{p}{(}\PYG{p}{[}\PYG{l+s+s1}{\PYGZsq{}B3\PYGZsq{}}\PYG{p}{,}\PYG{+w}{ }\PYG{l+s+s1}{\PYGZsq{}B4\PYGZsq{}}\PYG{p}{]}\PYG{p}{,}\PYG{+w}{ }\PYG{p}{[}\PYG{l+s+s1}{\PYGZsq{}RED\PYGZsq{}}\PYG{p}{,}\PYG{+w}{ }\PYG{l+s+s1}{\PYGZsq{}NIR\PYGZsq{}}\PYG{p}{]}\PYG{p}{)}\PYG{p}{;}\PYG{p}{;}
\PYG{+w}{     }\PYG{k+kd}{var}\PYG{+w}{ }\PYG{n+nx}{L7b}\PYG{+w}{ }\PYG{o}{=}\PYG{+w}{ }\PYG{n+nx}{ee}\PYG{p}{.}\PYG{n+nx}{ImageCollection}\PYG{p}{(}\PYG{l+s+s1}{\PYGZsq{}LANDSAT/LE07/C01/T1\PYGZus{}SR\PYGZsq{}}\PYG{p}{)}
\PYG{+w}{                       }\PYG{p}{.}\PYG{n+nx}{filterDate}\PYG{p}{(}\PYG{l+s+s1}{\PYGZsq{}2012\PYGZhy{}01\PYGZhy{}01\PYGZsq{}}\PYG{p}{,}\PYG{+w}{ }\PYG{l+s+s1}{\PYGZsq{}2013\PYGZhy{}12\PYGZhy{}31\PYGZsq{}}\PYG{p}{)}
\PYG{+w}{                       }\PYG{p}{.}\PYG{n+nx}{filterDate}\PYG{p}{(}\PYG{n+nx}{start\PYGZus{}date}\PYG{p}{,}\PYG{+w}{ }\PYG{n+nx}{end\PYGZus{}date}\PYG{p}{)}
\PYG{+w}{                       }\PYG{p}{.}\PYG{n+nx}{filter}\PYG{p}{(}\PYG{n+nx}{ee}\PYG{p}{.}\PYG{n+nx}{Filter}\PYG{p}{.}\PYG{n+nx}{lessThan}\PYG{p}{(}\PYG{l+s+s1}{\PYGZsq{}CLOUD\PYGZus{}COVER\PYGZus{}LAND\PYGZsq{}}\PYG{p}{,}\PYG{+w}{ }\PYG{l+m+mf}{100}\PYG{p}{)}\PYG{p}{)}
\PYG{+w}{                       }\PYG{p}{.}\PYG{n+nx}{filterBounds}\PYG{p}{(}\PYG{n+nx}{studyArea}\PYG{p}{)}
\PYG{+w}{                       }\PYG{p}{.}\PYG{n+nx}{map}\PYG{p}{(}\PYG{n+nx}{cloudMaskL7}\PYG{p}{)}
\PYG{+w}{                       }\PYG{p}{.}\PYG{n+nx}{select}\PYG{p}{(}\PYG{p}{[}\PYG{l+s+s1}{\PYGZsq{}B3\PYGZsq{}}\PYG{p}{,}\PYG{+w}{ }\PYG{l+s+s1}{\PYGZsq{}B4\PYGZsq{}}\PYG{p}{]}\PYG{p}{,}\PYG{+w}{ }\PYG{p}{[}\PYG{l+s+s1}{\PYGZsq{}RED\PYGZsq{}}\PYG{p}{,}\PYG{+w}{ }\PYG{l+s+s1}{\PYGZsq{}NIR\PYGZsq{}}\PYG{p}{]}\PYG{p}{)}\PYG{p}{;}\PYG{p}{;}

\PYG{+w}{     }\PYG{k+kd}{var}\PYG{+w}{ }\PYG{n+nx}{L7}\PYG{+w}{ }\PYG{o}{=}\PYG{+w}{ }\PYG{n+nx}{L7a}\PYG{p}{.}\PYG{n+nx}{merge}\PYG{p}{(}\PYG{n+nx}{L7b}\PYG{p}{)}\PYG{p}{;}

\PYG{+w}{     }\PYG{k+kd}{var}\PYG{+w}{ }\PYG{n+nx}{L8}\PYG{+w}{ }\PYG{o}{=}\PYG{+w}{ }\PYG{n+nx}{ee}\PYG{p}{.}\PYG{n+nx}{ImageCollection}\PYG{p}{(}\PYG{l+s+s1}{\PYGZsq{}LANDSAT/LC08/C01/T1\PYGZus{}SR\PYGZsq{}}\PYG{p}{)}
\PYG{+w}{                       }\PYG{p}{.}\PYG{n+nx}{filterDate}\PYG{p}{(}\PYG{n+nx}{start\PYGZus{}date}\PYG{p}{,}\PYG{+w}{ }\PYG{n+nx}{end\PYGZus{}date}\PYG{p}{)}
\PYG{+w}{                       }\PYG{p}{.}\PYG{n+nx}{filter}\PYG{p}{(}\PYG{n+nx}{ee}\PYG{p}{.}\PYG{n+nx}{Filter}\PYG{p}{.}\PYG{n+nx}{lessThan}\PYG{p}{(}\PYG{l+s+s1}{\PYGZsq{}CLOUD\PYGZus{}COVER\PYGZsq{}}\PYG{p}{,}\PYG{+w}{ }\PYG{n+nx}{cloudCoveragePercentage}\PYG{p}{)}\PYG{p}{)}
\PYG{+w}{                       }\PYG{p}{.}\PYG{n+nx}{filterBounds}\PYG{p}{(}\PYG{n+nx}{studyArea}\PYG{p}{)}
\PYG{+w}{                       }\PYG{c+c1}{//.filterBounds(AOI)}
\PYG{+w}{                       }\PYG{p}{.}\PYG{n+nx}{map}\PYG{p}{(}\PYG{n+nx}{maskL8sr}\PYG{p}{)}
\PYG{+w}{                       }\PYG{p}{.}\PYG{n+nx}{select}\PYG{p}{(}\PYG{p}{[}\PYG{l+s+s1}{\PYGZsq{}B4\PYGZsq{}}\PYG{p}{,}\PYG{+w}{ }\PYG{l+s+s1}{\PYGZsq{}B5\PYGZsq{}}\PYG{p}{]}\PYG{p}{,}\PYG{+w}{ }\PYG{p}{[}\PYG{l+s+s1}{\PYGZsq{}RED\PYGZsq{}}\PYG{p}{,}\PYG{+w}{ }\PYG{l+s+s1}{\PYGZsq{}NIR\PYGZsq{}}\PYG{p}{]}\PYG{p}{)}\PYG{p}{;}






\PYG{+w}{         }\PYG{c+c1}{//Define collection}


\PYG{+w}{     }\PYG{c+c1}{//\PYGZhy{}\PYGZhy{}\PYGZhy{}\PYGZhy{}\PYGZhy{}\PYGZhy{}\PYGZhy{}\PYGZhy{}\PYGZhy{}\PYGZhy{}\PYGZhy{}\PYGZhy{}\PYGZhy{}\PYGZhy{}\PYGZhy{}\PYGZhy{}\PYGZhy{}\PYGZhy{}\PYGZhy{}\PYGZhy{}\PYGZhy{}\PYGZhy{}\PYGZhy{}\PYGZhy{}\PYGZhy{}\PYGZhy{}\PYGZhy{}\PYGZhy{}\PYGZhy{}\PYGZhy{}\PYGZhy{}\PYGZhy{}\PYGZhy{}\PYGZhy{}\PYGZhy{}\PYGZhy{}\PYGZhy{}\PYGZhy{}\PYGZhy{}\PYGZhy{}\PYGZhy{}\PYGZhy{}\PYGZhy{}\PYGZhy{}\PYGZhy{}\PYGZhy{}\PYGZhy{}\PYGZhy{}\PYGZhy{}\PYGZhy{}\PYGZhy{}\PYGZhy{}\PYGZhy{}\PYGZhy{}\PYGZhy{}\PYGZhy{}\PYGZhy{}\PYGZhy{}\PYGZhy{}\PYGZhy{}\PYGZhy{}\PYGZhy{}\PYGZhy{}\PYGZhy{}\PYGZhy{}\PYGZhy{}\PYGZhy{}\PYGZhy{}}
\PYG{+w}{     }\PYG{c+c1}{// Merge Landsat 4, 5, 8 imagery collections and filter all by date/place}
\PYG{+w}{     }\PYG{c+c1}{//\PYGZhy{}\PYGZhy{}\PYGZhy{}\PYGZhy{}\PYGZhy{}\PYGZhy{}\PYGZhy{}\PYGZhy{}\PYGZhy{}\PYGZhy{}\PYGZhy{}\PYGZhy{}\PYGZhy{}\PYGZhy{}\PYGZhy{}\PYGZhy{}\PYGZhy{}\PYGZhy{}\PYGZhy{}\PYGZhy{}\PYGZhy{}\PYGZhy{}\PYGZhy{}\PYGZhy{}\PYGZhy{}\PYGZhy{}\PYGZhy{}\PYGZhy{}\PYGZhy{}\PYGZhy{}\PYGZhy{}\PYGZhy{}\PYGZhy{}\PYGZhy{}\PYGZhy{}\PYGZhy{}\PYGZhy{}\PYGZhy{}\PYGZhy{}\PYGZhy{}\PYGZhy{}\PYGZhy{}\PYGZhy{}\PYGZhy{}\PYGZhy{}\PYGZhy{}\PYGZhy{}\PYGZhy{}\PYGZhy{}\PYGZhy{}\PYGZhy{}\PYGZhy{}\PYGZhy{}\PYGZhy{}\PYGZhy{}\PYGZhy{}\PYGZhy{}\PYGZhy{}\PYGZhy{}\PYGZhy{}\PYGZhy{}\PYGZhy{}\PYGZhy{}\PYGZhy{}\PYGZhy{}\PYGZhy{}\PYGZhy{}\PYGZhy{}}

\PYG{+w}{     }\PYG{c+c1}{//Merge Landsat 4, 5 , 7 \PYGZsq{}}

\PYG{+w}{     }\PYG{k+kd}{var}\PYG{+w}{ }\PYG{n+nx}{L4578}\PYG{+w}{ }\PYG{o}{=}\PYG{+w}{ }\PYG{n+nx}{L4}\PYG{p}{.}\PYG{n+nx}{merge}\PYG{p}{(}\PYG{n+nx}{L5}\PYG{p}{)}\PYG{p}{.}\PYG{n+nx}{merge}\PYG{p}{(}\PYG{n+nx}{L7}\PYG{p}{)}\PYG{p}{.}\PYG{n+nx}{merge}\PYG{p}{(}\PYG{n+nx}{L8}\PYG{p}{)}\PYG{p}{;}


\PYG{+w}{     }\PYG{c+c1}{//\PYGZhy{}\PYGZhy{}\PYGZhy{}\PYGZhy{}\PYGZhy{}\PYGZhy{}\PYGZhy{}\PYGZhy{}\PYGZhy{}\PYGZhy{}\PYGZhy{}\PYGZhy{}\PYGZhy{}\PYGZhy{}\PYGZhy{}\PYGZhy{}\PYGZhy{}\PYGZhy{}\PYGZhy{}\PYGZhy{}\PYGZhy{}\PYGZhy{}\PYGZhy{}\PYGZhy{}\PYGZhy{}\PYGZhy{}\PYGZhy{}\PYGZhy{}\PYGZhy{}\PYGZhy{}\PYGZhy{}\PYGZhy{}\PYGZhy{}\PYGZhy{}\PYGZhy{}\PYGZhy{}\PYGZhy{}\PYGZhy{}\PYGZhy{}\PYGZhy{}\PYGZhy{}\PYGZhy{}\PYGZhy{}\PYGZhy{}\PYGZhy{}\PYGZhy{}\PYGZhy{}\PYGZhy{}\PYGZhy{}\PYGZhy{}\PYGZhy{}\PYGZhy{}\PYGZhy{}\PYGZhy{}\PYGZhy{}\PYGZhy{}\PYGZhy{}\PYGZhy{}\PYGZhy{}\PYGZhy{}\PYGZhy{}\PYGZhy{}\PYGZhy{}\PYGZhy{}\PYGZhy{}\PYGZhy{}\PYGZhy{}\PYGZhy{}}
\PYG{+w}{     }\PYG{c+c1}{//                     Create NDVI Collection}
\PYG{+w}{     }\PYG{c+c1}{//\PYGZhy{}\PYGZhy{}\PYGZhy{}\PYGZhy{}\PYGZhy{}\PYGZhy{}\PYGZhy{}\PYGZhy{}\PYGZhy{}\PYGZhy{}\PYGZhy{}\PYGZhy{}\PYGZhy{}\PYGZhy{}\PYGZhy{}\PYGZhy{}\PYGZhy{}\PYGZhy{}\PYGZhy{}\PYGZhy{}\PYGZhy{}\PYGZhy{}\PYGZhy{}\PYGZhy{}\PYGZhy{}\PYGZhy{}\PYGZhy{}\PYGZhy{}\PYGZhy{}\PYGZhy{}\PYGZhy{}\PYGZhy{}\PYGZhy{}\PYGZhy{}\PYGZhy{}\PYGZhy{}\PYGZhy{}\PYGZhy{}\PYGZhy{}\PYGZhy{}\PYGZhy{}\PYGZhy{}\PYGZhy{}\PYGZhy{}\PYGZhy{}\PYGZhy{}\PYGZhy{}\PYGZhy{}\PYGZhy{}\PYGZhy{}\PYGZhy{}\PYGZhy{}\PYGZhy{}\PYGZhy{}\PYGZhy{}\PYGZhy{}\PYGZhy{}\PYGZhy{}\PYGZhy{}\PYGZhy{}\PYGZhy{}\PYGZhy{}\PYGZhy{}\PYGZhy{}\PYGZhy{}\PYGZhy{}\PYGZhy{}\PYGZhy{}}

\PYG{+w}{     }\PYG{k+kd}{var}\PYG{+w}{ }\PYG{n+nx}{NDVI}\PYG{+w}{ }\PYG{o}{=}\PYG{+w}{ }\PYG{k+kd}{function}\PYG{p}{(}\PYG{n+nx}{image}\PYG{p}{)}\PYG{+w}{ }\PYG{p}{\PYGZob{}}
\PYG{+w}{       }\PYG{k}{return}\PYG{+w}{ }\PYG{n+nx}{image}\PYG{p}{.}\PYG{n+nx}{normalizedDifference}\PYG{p}{(}\PYG{p}{[}\PYG{l+s+s1}{\PYGZsq{}NIR\PYGZsq{}}\PYG{p}{,}\PYG{+w}{ }\PYG{l+s+s1}{\PYGZsq{}RED\PYGZsq{}}\PYG{p}{]}\PYG{p}{)}\PYG{p}{.}\PYG{n+nx}{rename}\PYG{p}{(}\PYG{l+s+s1}{\PYGZsq{}NDVI\PYGZsq{}}\PYG{p}{)}\PYG{p}{;}
\PYG{+w}{       }\PYG{c+c1}{//return image.addBands(ndvi);}
\PYG{+w}{     }\PYG{p}{\PYGZcb{}}\PYG{p}{;}

\PYG{+w}{     }\PYG{k}{if}\PYG{+w}{ }\PYG{p}{(}\PYG{n+nx}{ALGO}\PYG{o}{==}\PYG{l+s+s1}{\PYGZsq{}MEDIAN\PYGZsq{}}\PYG{p}{)}\PYG{p}{\PYGZob{}}
\PYG{+w}{         }\PYG{k+kd}{var}\PYG{+w}{ }\PYG{n+nx}{suffix}\PYG{+w}{ }\PYG{o}{=}\PYG{+w}{ }\PYG{l+s+s1}{\PYGZsq{}median\PYGZsq{}}\PYG{p}{;}
\PYG{+w}{         }\PYG{k+kd}{var}\PYG{+w}{ }\PYG{n+nx}{annualNDVI}\PYG{+w}{ }\PYG{o}{=}\PYG{+w}{ }\PYG{n+nx}{L4578}\PYG{p}{.}\PYG{n+nx}{map}\PYG{p}{(}\PYG{n+nx}{NDVI}\PYG{p}{)}\PYG{p}{.}\PYG{n+nx}{median}\PYG{p}{(}\PYG{p}{)}\PYG{p}{.}\PYG{n+nx}{clip}\PYG{p}{(}\PYG{n+nx}{studyArea}\PYG{p}{)}\PYG{p}{;}
\PYG{+w}{     }\PYG{p}{\PYGZcb{}}

\PYG{+w}{     }\PYG{k}{if}\PYG{+w}{ }\PYG{p}{(}\PYG{n+nx}{ALGO}\PYG{o}{==}\PYG{l+s+s1}{\PYGZsq{}PERCENTILE75\PYGZsq{}}\PYG{p}{)}\PYG{p}{\PYGZob{}}
\PYG{+w}{         }\PYG{k+kd}{var}\PYG{+w}{ }\PYG{n+nx}{suffix}\PYG{+w}{ }\PYG{o}{=}\PYG{+w}{ }\PYG{l+s+s1}{\PYGZsq{}75pc\PYGZsq{}}\PYG{p}{;}
\PYG{+w}{         }\PYG{k+kd}{var}\PYG{+w}{ }\PYG{n+nx}{annualNDVI}\PYG{+w}{ }\PYG{o}{=}\PYG{+w}{ }\PYG{n+nx}{L4578}\PYG{p}{.}\PYG{n+nx}{map}\PYG{p}{(}\PYG{n+nx}{NDVI}\PYG{p}{)}\PYG{p}{.}\PYG{n+nx}{reduce}\PYG{p}{(}\PYG{n+nx}{ee}\PYG{p}{.}\PYG{n+nx}{Reducer}\PYG{p}{.}\PYG{n+nx}{percentile}\PYG{p}{(}\PYG{p}{[}\PYG{l+m+mf}{75}\PYG{p}{]}\PYG{p}{)}\PYG{p}{)}\PYG{p}{.}\PYG{n+nx}{clip}\PYG{p}{(}\PYG{n+nx}{studyArea}\PYG{p}{)}\PYG{p}{;}
\PYG{+w}{     }\PYG{p}{\PYGZcb{}}

\PYG{+w}{     }\PYG{k}{if}\PYG{+w}{ }\PYG{p}{(}\PYG{n+nx}{ALGO}\PYG{o}{==}\PYG{l+s+s1}{\PYGZsq{}PERCENTILE65\PYGZsq{}}\PYG{p}{)}\PYG{p}{\PYGZob{}}
\PYG{+w}{         }\PYG{k+kd}{var}\PYG{+w}{ }\PYG{n+nx}{suffix}\PYG{+w}{ }\PYG{o}{=}\PYG{+w}{ }\PYG{l+s+s1}{\PYGZsq{}65pc\PYGZsq{}}\PYG{p}{;}
\PYG{+w}{         }\PYG{k+kd}{var}\PYG{+w}{ }\PYG{n+nx}{annualNDVI}\PYG{+w}{ }\PYG{o}{=}\PYG{+w}{ }\PYG{n+nx}{L4578}\PYG{p}{.}\PYG{n+nx}{map}\PYG{p}{(}\PYG{n+nx}{NDVI}\PYG{p}{)}\PYG{p}{.}\PYG{n+nx}{reduce}\PYG{p}{(}\PYG{n+nx}{ee}\PYG{p}{.}\PYG{n+nx}{Reducer}\PYG{p}{.}\PYG{n+nx}{percentile}\PYG{p}{(}\PYG{p}{[}\PYG{l+m+mf}{65}\PYG{p}{]}\PYG{p}{)}\PYG{p}{)}\PYG{p}{.}\PYG{n+nx}{clip}\PYG{p}{(}\PYG{n+nx}{studyArea}\PYG{p}{)}\PYG{p}{;}
\PYG{+w}{     }\PYG{p}{\PYGZcb{}}


\PYG{+w}{     }\PYG{k+kd}{var}\PYG{+w}{ }\PYG{n+nx}{ndvi\PYGZus{}visualization}\PYG{+w}{ }\PYG{o}{=}\PYG{+w}{ }\PYG{p}{\PYGZob{}}
\PYG{+w}{       }\PYG{n+nx}{min}\PYG{o}{:}\PYG{+w}{ }\PYG{o}{\PYGZhy{}}\PYG{l+m+mf}{0.22789797020331423}\PYG{p}{,}
\PYG{+w}{       }\PYG{n+nx}{max}\PYG{o}{:}\PYG{+w}{ }\PYG{l+m+mf}{0.6575894075894075}\PYG{p}{,}
\PYG{+w}{       }\PYG{n+nx}{palette}\PYG{o}{:}\PYG{+w}{ }\PYG{l+s+s1}{\PYGZsq{}FFFFFF, CE7E45, DF923D, F1B555, FCD163, 99B718, 74A901, 66A000, 529400,\PYGZsq{}}\PYG{+w}{ }\PYG{o}{+}
\PYG{+w}{         }\PYG{l+s+s1}{\PYGZsq{}3E8601, 207401, 056201, 004C00, 023B01, 012E01, 011D01, 011301\PYGZsq{}}
\PYG{+w}{     }\PYG{p}{\PYGZcb{}}\PYG{p}{;}
\PYG{+w}{     }\PYG{n+nb}{Map}\PYG{p}{.}\PYG{n+nx}{addLayer}\PYG{p}{(}\PYG{n+nx}{annualNDVI}\PYG{p}{,}\PYG{+w}{ }\PYG{n+nx}{ndvi\PYGZus{}visualization}\PYG{p}{,}\PYG{+w}{ }\PYG{l+s+s1}{\PYGZsq{}NDVI\PYGZsq{}}\PYG{p}{)}\PYG{p}{;}

\PYG{+w}{     }\PYG{c+c1}{//\PYGZhy{}\PYGZhy{}\PYGZhy{}\PYGZhy{}\PYGZhy{}\PYGZhy{}\PYGZhy{}\PYGZhy{}\PYGZhy{}\PYGZhy{}\PYGZhy{}\PYGZhy{}\PYGZhy{}\PYGZhy{}\PYGZhy{}\PYGZhy{}\PYGZhy{}\PYGZhy{}\PYGZhy{}\PYGZhy{}\PYGZhy{}\PYGZhy{}\PYGZhy{}\PYGZhy{}\PYGZhy{}\PYGZhy{}\PYGZhy{}\PYGZhy{}\PYGZhy{}\PYGZhy{}\PYGZhy{}\PYGZhy{}\PYGZhy{}\PYGZhy{}\PYGZhy{}\PYGZhy{}\PYGZhy{}\PYGZhy{}\PYGZhy{}\PYGZhy{}\PYGZhy{}\PYGZhy{}\PYGZhy{}\PYGZhy{}\PYGZhy{}\PYGZhy{}\PYGZhy{}\PYGZhy{}\PYGZhy{}\PYGZhy{}\PYGZhy{}\PYGZhy{}\PYGZhy{}\PYGZhy{}\PYGZhy{}\PYGZhy{}\PYGZhy{}\PYGZhy{}\PYGZhy{}\PYGZhy{}\PYGZhy{}\PYGZhy{}\PYGZhy{}\PYGZhy{}\PYGZhy{}\PYGZhy{}\PYGZhy{}\PYGZhy{}}
\PYG{+w}{     }\PYG{c+c1}{//       Export as GeoTIFF}
\PYG{+w}{     }\PYG{c+c1}{//\PYGZhy{}\PYGZhy{}\PYGZhy{}\PYGZhy{}\PYGZhy{}\PYGZhy{}\PYGZhy{}\PYGZhy{}\PYGZhy{}\PYGZhy{}\PYGZhy{}\PYGZhy{}\PYGZhy{}\PYGZhy{}\PYGZhy{}\PYGZhy{}\PYGZhy{}\PYGZhy{}\PYGZhy{}\PYGZhy{}\PYGZhy{}\PYGZhy{}\PYGZhy{}\PYGZhy{}\PYGZhy{}\PYGZhy{}\PYGZhy{}\PYGZhy{}\PYGZhy{}\PYGZhy{}\PYGZhy{}\PYGZhy{}\PYGZhy{}\PYGZhy{}\PYGZhy{}\PYGZhy{}\PYGZhy{}\PYGZhy{}\PYGZhy{}\PYGZhy{}\PYGZhy{}\PYGZhy{}\PYGZhy{}\PYGZhy{}\PYGZhy{}\PYGZhy{}\PYGZhy{}\PYGZhy{}\PYGZhy{}\PYGZhy{}\PYGZhy{}\PYGZhy{}\PYGZhy{}\PYGZhy{}\PYGZhy{}\PYGZhy{}\PYGZhy{}\PYGZhy{}\PYGZhy{}\PYGZhy{}\PYGZhy{}\PYGZhy{}\PYGZhy{}\PYGZhy{}\PYGZhy{}\PYGZhy{}\PYGZhy{}\PYGZhy{}}


\PYG{+w}{     }\PYG{n+nx}{Export}\PYG{p}{.}\PYG{n+nx}{image}\PYG{p}{.}\PYG{n+nx}{toDrive}\PYG{p}{(}\PYG{p}{\PYGZob{}}
\PYG{+w}{      }\PYG{n+nx}{image}\PYG{o}{:}\PYG{+w}{ }\PYG{n+nx}{annualNDVI}\PYG{p}{,}
\PYG{+w}{      }\PYG{n+nx}{description}\PYG{o}{:}\PYG{+w}{ }\PYG{n+nx}{country}\PYG{+w}{ }\PYG{o}{+}\PYG{+w}{ }\PYG{l+s+s1}{\PYGZsq{}\PYGZus{}NDVI\PYGZus{}\PYGZsq{}}\PYG{+w}{ }\PYG{o}{+}\PYG{+w}{ }\PYG{n+nx}{suffix}\PYG{+w}{ }\PYG{o}{+}\PYG{+w}{ }\PYG{l+s+s1}{\PYGZsq{}\PYGZus{}\PYGZsq{}}\PYG{+w}{ }\PYG{o}{+}\PYG{+w}{ }\PYG{n+nx}{Year}\PYG{p}{,}
\PYG{+w}{      }\PYG{n+nx}{scale}\PYG{o}{:}\PYG{+w}{ }\PYG{l+m+mf}{30}\PYG{p}{,}
\PYG{+w}{      }\PYG{n+nx}{region}\PYG{o}{:}\PYG{+w}{ }\PYG{n+nx}{studyArea}\PYG{p}{,}
\PYG{+w}{      }\PYG{n+nx}{maxPixels}\PYG{o}{:}\PYG{+w}{  }\PYG{l+m+mf}{1e13}\PYG{p}{,}
\PYG{+w}{      }\PYG{n+nx}{fileFormat}\PYG{o}{:}\PYG{+w}{ }\PYG{l+s+s1}{\PYGZsq{}GeoTIFF\PYGZsq{}}\PYG{p}{,}
\PYG{+w}{      }\PYG{n+nx}{folder}\PYG{o}{:}\PYG{l+s+s1}{\PYGZsq{}GEE\PYGZus{}classification\PYGZsq{}}\PYG{p}{,}
\PYG{+w}{      }\PYG{n+nx}{formatOptions}\PYG{o}{:}\PYG{+w}{ }\PYG{p}{\PYGZob{}}
\PYG{+w}{        }\PYG{n+nx}{cloudOptimized}\PYG{o}{:}\PYG{+w}{ }\PYG{k+kc}{true}
\PYG{+w}{          }\PYG{p}{\PYGZcb{}}\PYG{p}{,}
\PYG{+w}{       }\PYG{n+nx}{skipEmptyTiles}\PYG{o}{:}\PYG{+w}{ }\PYG{k+kc}{true}
\PYG{+w}{       }\PYG{p}{\PYGZcb{}}\PYG{p}{)}\PYG{p}{;}

\PYG{+w}{       }\PYG{c+c1}{//Map.addLayer(L7.first(), \PYGZob{}\PYGZcb{}, \PYGZsq{}L7\PYGZsq{});}
\end{sphinxVerbatim}

\sphinxstepscope


\chapter{Landcover Data Preparation}
\label{\detokenize{Preprocessing/Landcover:landcover-data-preparation}}\label{\detokenize{Preprocessing/Landcover::doc}}
\sphinxAtStartPar
MISLAND North Africa usese the European Space Agency (ESA) Climate Change Initiative (CCI) land cover dataset. This dataset provides global maps describing the land surface into 22 classes, which have been defined using the United Nations Food and Agriculture Organization’s (UN FAO) Land Cover Classification System (LCCS). In addition to the land cover (LC) maps, four quality flags are produced to document the reliability of the classification and change detection.
In order to ensure continuity, these land cover maps are consistent with the series of global annual LC maps from the 1990s to 2015 produced by the European Space Agency (ESA) Climate Change Initiative (CCI), which are also available on the ESA CCI LC viewer.

\sphinxAtStartPar
The dataset can be downloaded here from the \sphinxhref{https://developers.google.com/earth-engine/datasets/catalog/landsat}{ESA C3S archives}


\section{Data Preprocessing Steps}
\label{\detokenize{Preprocessing/Landcover:data-preprocessing-steps}}\begin{enumerate}
\sphinxsetlistlabels{\arabic}{enumi}{enumii}{}{.}%
\item {} 
\sphinxAtStartPar
Unzip the C3S\sphinxhyphen{}LC\sphinxhyphen{}L4\sphinxhyphen{}LCCS\sphinxhyphen{}Map\sphinxhyphen{}300m\sphinxhyphen{}P1Y\sphinxhyphen{}yyyy\sphinxhyphen{}v2.1.1 data from the compressed format that it comes in after downloading. Closely examine the contents of this folder as it has various files.

\end{enumerate}

\begin{figure}[H]
\centering
\capstart

\noindent\sphinxincludegraphics[width=800\sphinxpxdimen,height=400\sphinxpxdimen]{{LC1}.png}
\caption{Rasters option on admin panel}\label{\detokenize{Preprocessing/Landcover:id3}}\end{figure}
\begin{enumerate}
\sphinxsetlistlabels{\arabic}{enumi}{enumii}{}{.}%
\setcounter{enumi}{1}
\item {} 
\sphinxAtStartPar
Load the unzipped NetCDF4 raster onto qgis and select the lccs\_class on the “Select Raster Layers to Add” dialog.

\end{enumerate}

\begin{figure}[H]
\centering
\capstart

\noindent\sphinxincludegraphics[width=800\sphinxpxdimen,height=600\sphinxpxdimen]{{LC2}.png}
\caption{Extraction by mask}\label{\detokenize{Preprocessing/Landcover:id4}}\end{figure}
\begin{enumerate}
\sphinxsetlistlabels{\arabic}{enumi}{enumii}{}{.}%
\setcounter{enumi}{2}
\item {} 
\sphinxAtStartPar
Once the raster layer is loaded add the vector layer of the OSS region and navigate to Raster \textgreater{} Extraction \textgreater{} Clip Raster by Mask Layer

\end{enumerate}

\begin{figure}[H]
\centering
\capstart

\noindent\sphinxincludegraphics[width=625\sphinxpxdimen,height=287\sphinxpxdimen]{{LC3}.png}
\caption{Finding the r.reclass processing tool}\label{\detokenize{Preprocessing/Landcover:id5}}\end{figure}
\begin{enumerate}
\sphinxsetlistlabels{\arabic}{enumi}{enumii}{}{.}%
\setcounter{enumi}{3}
\item {} 
\sphinxAtStartPar
Once the raster layer has been clipped to the area of interest, open the processing toolbox and search for r.reclass and select the GRASS\textgreater{}Raster\textgreater{}r.reclass  tool.

\end{enumerate}

\begin{figure}[H]
\centering
\capstart

\noindent\sphinxincludegraphics[width=863\sphinxpxdimen,height=709\sphinxpxdimen]{{LC4}.png}
\caption{Finding the r.reclass processing tool}\label{\detokenize{Preprocessing/Landcover:id6}}\end{figure}
\begin{enumerate}
\sphinxsetlistlabels{\arabic}{enumi}{enumii}{}{.}%
\setcounter{enumi}{4}
\item {} 
\sphinxAtStartPar
On the \sphinxstyleemphasis{r.reclass} dialog that pops up, select the clipped land cover data as the input layer and paste the following reclassification rules into the “Reclass rules text box” and save the output in a desired location.

\end{enumerate}

\begin{figure}[H]
\centering
\capstart

\noindent\sphinxincludegraphics[width=800\sphinxpxdimen,height=700\sphinxpxdimen]{{LC5}.png}
\caption{Finding the r.reclass processing tool}\label{\detokenize{Preprocessing/Landcover:id7}}\end{figure}
\begin{enumerate}
\sphinxsetlistlabels{\arabic}{enumi}{enumii}{}{.}%
\setcounter{enumi}{5}
\item {} 
\sphinxAtStartPar
Once the data is reclassified you can upload the QML style layer to visualize and validate the reclassified land cover data.

\end{enumerate}

\begin{figure}[H]
\centering
\capstart

\noindent\sphinxincludegraphics[width=800\sphinxpxdimen,height=400\sphinxpxdimen]{{LC6}.png}
\caption{Finding the r.reclass processing tool}\label{\detokenize{Preprocessing/Landcover:id8}}\end{figure}


\section{Data Upload to MISLAND service}
\label{\detokenize{Preprocessing/Landcover:data-upload-to-misland-service}}
\sphinxAtStartPar
To upload the Land cover dataset to the admin panel. Follow these simple steps
\begin{enumerate}
\sphinxsetlistlabels{\arabic}{enumi}{enumii}{}{.}%
\item {} 
\sphinxAtStartPar
Select the \sphinxstylestrong{Rasters} option from the list of options on the admin panel

\end{enumerate}

\begin{figure}[H]
\centering
\capstart

\noindent\sphinxincludegraphics[width=652\sphinxpxdimen,height=598\sphinxpxdimen]{{LC7}.png}
\caption{Rasters option on admin panel}\label{\detokenize{Preprocessing/Landcover:id9}}\end{figure}

\sphinxAtStartPar
2  From the \sphinxstyleemphasis{FILTER} options, \sphinxstyleemphasis{By raster type} select \sphinxstylestrong{LULC: Land use/land Cover} option to view the list of Land cover datasets that are already availabel on the database

\begin{figure}[H]
\centering
\capstart

\noindent\sphinxincludegraphics[width=550\sphinxpxdimen,height=435\sphinxpxdimen]{{LC8}.png}
\caption{Selecting Land cover option from the list of filters}\label{\detokenize{Preprocessing/Landcover:id10}}\end{figure}
\begin{enumerate}
\sphinxsetlistlabels{\arabic}{enumi}{enumii}{}{.}%
\setcounter{enumi}{2}
\item {} 
\sphinxAtStartPar
Once you have confirmed that the raster you wish to add is not in the database. Select the \sphinxstylestrong{ADD RASTER} option from the top\sphinxhyphen{}right conner of the admin panel.

\end{enumerate}

\begin{figure}[H]
\centering
\capstart

\noindent\sphinxincludegraphics[width=425\sphinxpxdimen,height=301\sphinxpxdimen]{{LC9}.png}
\caption{Selecting ‘ADD RASTER’ option}\label{\detokenize{Preprocessing/Landcover:id11}}\end{figure}
\begin{enumerate}
\sphinxsetlistlabels{\arabic}{enumi}{enumii}{}{.}%
\setcounter{enumi}{3}
\item {} 
\sphinxAtStartPar
On the add raster form that opens up, fill in the \sphinxstyleemphasis{Name} of the Land cover raster you with to add, then select the \sphinxstyleemphasis{Raster Year} and the \sphinxstyleemphasis{Raster Type} as shown below:

\end{enumerate}

\begin{figure}[H]
\centering
\capstart

\noindent\sphinxincludegraphics[width=738\sphinxpxdimen,height=588\sphinxpxdimen]{{LC10}.png}
\caption{Filling the ADD RASTER form for land cover data upload}\label{\detokenize{Preprocessing/Landcover:id12}}\end{figure}

\begin{sphinxadmonition}{note}{Note:}
\sphinxAtStartPar
It is recomended that you include the Year of the raster in the \sphinxstyleemphasis{Name} field as shown and that you associate the Land cover raster with the \sphinxstylestrong{LULC: Land use/land cover} {\color{red}\bfseries{}*}Raster Type*for the system to work properly and point to the right raster dataset.
\end{sphinxadmonition}

\sphinxstepscope


\chapter{Forest Loss}
\label{\detokenize{Preprocessing/Forestloss:forest-loss}}\label{\detokenize{Preprocessing/Forestloss::doc}}
\sphinxAtStartPar
The High\sphinxhyphen{}resolution global forset change results from time\sphinxhyphen{}series analysis of Landsat images in characterizing global forest extent and change.

\sphinxAtStartPar
The ‘first’ and ‘last’ bands are reference multispectral imagery from the first and last available years for Landsat spectral bands 3, 4, 5, and 7. Reference composite imagery represents median observations from a set of quality\sphinxhyphen{}assessed growing\sphinxhyphen{}season observations for each of these bands.

\sphinxAtStartPar
MISLAND North Africal uses the Hansen Global Forest Change v1.8 to asses forest loss in the North Africa action zone. For more details of the Hansen Global Forest Change see the \sphinxhref{https://storage.googleapis.com/earthenginepartners-hansen/GFC-2020-v1.8/download.html}{user notes}


\section{Data Preprocessing and Download on GEE}
\label{\detokenize{Preprocessing/Forestloss:data-preprocessing-and-download-on-gee}}\begin{enumerate}
\sphinxsetlistlabels{\arabic}{enumi}{enumii}{}{.}%
\item {} 
\sphinxAtStartPar
Open the \phantomsection\label{\detokenize{Preprocessing/Forestloss:google-earth-engine-code}}Google Earth Engine Code and paste the following code to import the \sphinxstylestrong{Hansen Global forest change v1.8}, \sphinxstylestrong{OSS North Africa Action zone**(study area) and **Geometry} for forest areas in North Africa.

\end{enumerate}

\begin{sphinxVerbatim}[commandchars=\\\{\},numbers=left,firstnumber=1,stepnumber=1]
\PYG{k+kd}{var}\PYG{+w}{ }\PYG{n+nx}{image}\PYG{+w}{ }\PYG{o}{=}\PYG{+w}{ }\PYG{n+nx}{ee}\PYG{p}{.}\PYG{n+nx}{Image}\PYG{p}{(}\PYG{l+s+s2}{\PYGZdq{}UMD/hansen/global\PYGZus{}forest\PYGZus{}change\PYGZus{}2020\PYGZus{}v1\PYGZus{}8\PYGZdq{}}\PYG{p}{)}\PYG{p}{,}
\PYG{+w}{ }\PYG{n+nx}{table}\PYG{+w}{ }\PYG{o}{=}\PYG{+w}{ }\PYG{n+nx}{ee}\PYG{p}{.}\PYG{n+nx}{FeatureCollection}\PYG{p}{(}\PYG{l+s+s2}{\PYGZdq{}users/derickongeri/NorthAfrica\PYGZdq{}}\PYG{p}{)}\PYG{p}{,}
\PYG{+w}{ }\PYG{n+nx}{geometry}\PYG{+w}{ }\PYG{o}{=}
\PYG{+w}{ }\PYG{c+cm}{/* color: \PYGZsh{}d63000 */}
\PYG{+w}{ }\PYG{c+cm}{/* shown: false */}
\PYG{+w}{ }\PYG{c+cm}{/* displayProperties: [}
\PYG{c+cm}{   \PYGZob{}}
\PYG{c+cm}{     \PYGZdq{}type\PYGZdq{}: \PYGZdq{}rectangle\PYGZdq{}}
\PYG{c+cm}{   \PYGZcb{}}
\PYG{c+cm}{ ] */}
\PYG{+w}{ }\PYG{n+nx}{ee}\PYG{p}{.}\PYG{n+nx}{Geometry}\PYG{p}{.}\PYG{n+nx}{Polygon}\PYG{p}{(}
\PYG{+w}{     }\PYG{p}{[}\PYG{p}{[}\PYG{p}{[}\PYG{o}{\PYGZhy{}}\PYG{l+m+mf}{10.330249200933336}\PYG{p}{,}\PYG{+w}{ }\PYG{l+m+mf}{37.60222178276082}\PYG{p}{]}\PYG{p}{,}
\PYG{+w}{       }\PYG{p}{[}\PYG{o}{\PYGZhy{}}\PYG{l+m+mf}{10.330249200933336}\PYG{p}{,}\PYG{+w}{ }\PYG{l+m+mf}{31.415726509598066}\PYG{p}{]}\PYG{p}{,}
\PYG{+w}{       }\PYG{p}{[}\PYG{l+m+mf}{24.562328924066662}\PYG{p}{,}\PYG{+w}{ }\PYG{l+m+mf}{31.415726509598066}\PYG{p}{]}\PYG{p}{,}
\PYG{+w}{       }\PYG{p}{[}\PYG{l+m+mf}{24.562328924066662}\PYG{p}{,}\PYG{+w}{ }\PYG{l+m+mf}{37.60222178276082}\PYG{p}{]}\PYG{p}{]}\PYG{p}{]}\PYG{p}{,}\PYG{+w}{ }\PYG{k+kc}{null}\PYG{p}{,}\PYG{+w}{ }\PYG{k+kc}{false}\PYG{p}{)}\PYG{p}{;}
\end{sphinxVerbatim}
\begin{enumerate}
\sphinxsetlistlabels{\arabic}{enumi}{enumii}{}{.}%
\setcounter{enumi}{1}
\item {} 
\sphinxAtStartPar
To select the forestLoss by year band and export the product paste the following code.

\end{enumerate}

\begin{sphinxVerbatim}[commandchars=\\\{\},numbers=left,firstnumber=1,stepnumber=1]
\PYG{k+kd}{var}\PYG{+w}{ }\PYG{n+nx}{dataset}\PYG{+w}{ }\PYG{o}{=}\PYG{+w}{ }\PYG{n+nx}{image}\PYG{p}{.}\PYG{n+nx}{clip}\PYG{p}{(}\PYG{n+nx}{table}\PYG{p}{)}\PYG{p}{;}

\PYG{k+kd}{var}\PYG{+w}{ }\PYG{n+nx}{forestLoss}\PYG{+w}{ }\PYG{o}{=}\PYG{+w}{ }\PYG{n+nx}{dataset}\PYG{p}{.}\PYG{n+nx}{select}\PYG{p}{(}\PYG{l+s+s1}{\PYGZsq{}lossyear\PYGZsq{}}\PYG{p}{)}\PYG{p}{;}

\PYG{n+nb}{Map}\PYG{p}{.}\PYG{n+nx}{addLayer}\PYG{p}{(}\PYG{n+nx}{forestLoss}\PYG{p}{)}\PYG{p}{;}

\PYG{n+nx}{Export}\PYG{p}{.}\PYG{n+nx}{image}\PYG{p}{.}\PYG{n+nx}{toDrive}\PYG{p}{(}\PYG{p}{\PYGZob{}}
\PYG{+w}{  }\PYG{n+nx}{image}\PYG{o}{:}\PYG{+w}{ }\PYG{n+nx}{forestLoss}\PYG{p}{,}
\PYG{+w}{  }\PYG{n+nx}{description}\PYG{o}{:}\PYG{+w}{ }\PYG{l+s+s1}{\PYGZsq{}ForestLoss\PYGZsq{}}\PYG{p}{,}
\PYG{+w}{  }\PYG{n+nx}{scale}\PYG{o}{:}\PYG{+w}{ }\PYG{l+m+mf}{30}\PYG{p}{,}
\PYG{+w}{  }\PYG{n+nx}{region}\PYG{o}{:}\PYG{+w}{ }\PYG{n+nx}{geometry}\PYG{p}{,}
\PYG{+w}{  }\PYG{n+nx}{maxPixels}\PYG{o}{:}\PYG{+w}{  }\PYG{l+m+mf}{1e13}\PYG{p}{,}
\PYG{+w}{  }\PYG{n+nx}{fileFormat}\PYG{o}{:}\PYG{+w}{ }\PYG{l+s+s1}{\PYGZsq{}GeoTIFF\PYGZsq{}}\PYG{p}{,}
\PYG{+w}{  }\PYG{n+nx}{folder}\PYG{o}{:}\PYG{l+s+s1}{\PYGZsq{}GEE\PYGZus{}classification\PYGZsq{}}\PYG{p}{,}
\PYG{+w}{  }\PYG{n+nx}{formatOptions}\PYG{o}{:}\PYG{+w}{ }\PYG{p}{\PYGZob{}}
\PYG{+w}{    }\PYG{n+nx}{cloudOptimized}\PYG{o}{:}\PYG{+w}{ }\PYG{k+kc}{true}
\PYG{+w}{  }\PYG{p}{\PYGZcb{}}\PYG{p}{,}
\PYG{+w}{    }\PYG{n+nx}{skipEmptyTiles}\PYG{o}{:}\PYG{+w}{ }\PYG{k+kc}{true}
\PYG{+w}{  }\PYG{p}{\PYGZcb{}}\PYG{p}{)}\PYG{p}{;}
\end{sphinxVerbatim}
\begin{enumerate}
\sphinxsetlistlabels{\arabic}{enumi}{enumii}{}{.}%
\setcounter{enumi}{2}
\item {} 
\sphinxAtStartPar
Save the code and run it to export the forestloss by year data.

\end{enumerate}


\section{Data Preprocessing in Qgis}
\label{\detokenize{Preprocessing/Forestloss:data-preprocessing-in-qgis}}\begin{enumerate}
\sphinxsetlistlabels{\arabic}{enumi}{enumii}{}{.}%
\item {} 
\sphinxAtStartPar
Download the data form Google drive to your desired location on your local machine and load it to Qgis to view the outputs.

\end{enumerate}

\begin{figure}[H]
\centering
\capstart

\noindent\sphinxincludegraphics[width=735\sphinxpxdimen,height=415\sphinxpxdimen]{{fr1}.png}
\caption{Opening the Forestloss year data on Qgis}\label{\detokenize{Preprocessing/Forestloss:id1}}\end{figure}
\begin{enumerate}
\sphinxsetlistlabels{\arabic}{enumi}{enumii}{}{.}%
\setcounter{enumi}{1}
\item {} 
\sphinxAtStartPar
On the Qgis menu\sphinxhyphen{}bar navigate to \sphinxstyleemphasis{Raster}\textgreater{}*Miscellaneous*\textgreater{}*Merge* to merge the downloaded tiles

\end{enumerate}

\begin{figure}[H]
\centering
\capstart

\noindent\sphinxincludegraphics[width=735\sphinxpxdimen,height=415\sphinxpxdimen]{{fr2}.png}
\caption{Merging the Forestloss year data}\label{\detokenize{Preprocessing/Forestloss:id2}}\end{figure}
\begin{enumerate}
\sphinxsetlistlabels{\arabic}{enumi}{enumii}{}{.}%
\item {} 
\sphinxAtStartPar
On the Merge dialog, \sphinxstylestrong{Input layers} option choose the \sphinxincludegraphics[width=1.5em]{{mActionSelectAll}.png}$^{\text{Select All}}$ option and click on \sphinxguilabel{OK}.

\end{enumerate}

\begin{figure}[H]
\centering
\capstart

\noindent\sphinxincludegraphics[width=742\sphinxpxdimen,height=642\sphinxpxdimen]{{fr3}.png}
\caption{Selecting All layers to merge}\label{\detokenize{Preprocessing/Forestloss:id3}}\end{figure}

\sphinxAtStartPar
save the output to a temporary layer so as to export it with desired properties in the next step.
\begin{enumerate}
\sphinxsetlistlabels{\arabic}{enumi}{enumii}{}{.}%
\setcounter{enumi}{3}
\item {} 
\sphinxAtStartPar
Right click on the layer and navigate to \sphinxstyleemphasis{Export}\textgreater{}*Save as* option and save the layer to your desired location with the appropriate name.

\end{enumerate}

\begin{sphinxadmonition}{note}{Note:}
\sphinxAtStartPar
The forest loss by year raster has values ranging from 1\sphinxhyphen{}20. The values represent the loss year form 2001 to 2020 hence to set the “nodata” value to 0 on the \sphinxstyleemphasis{Save Raster Layer as} dialog, check the \sphinxstylestrong{No data values} and input the values as shown in the figure below:
\begin{quote}

\begin{figure}[H]
\centering
\capstart

\noindent\sphinxincludegraphics[width=691\sphinxpxdimen,height=608\sphinxpxdimen]{{fr4}.png}
\caption{Saving the Forest Loss year data}\label{\detokenize{Preprocessing/Forestloss:id4}}\end{figure}
\end{quote}
\end{sphinxadmonition}


\section{Upload to MISLAND Service}
\label{\detokenize{Preprocessing/Forestloss:upload-to-misland-service}}
\begin{figure}[H]
\centering
\capstart

\noindent\sphinxincludegraphics[width=642\sphinxpxdimen,height=597\sphinxpxdimen]{{fr5}.png}
\caption{Uploading the data to MISLAND service}\label{\detokenize{Preprocessing/Forestloss:id5}}\end{figure}

\sphinxstepscope


\chapter{Ecological Units Layer}
\label{\detokenize{Preprocessing/Ecologicalunits:ecological-units-layer}}\label{\detokenize{Preprocessing/Ecologicalunits::doc}}

\section{QGIS Data pre\sphinxhyphen{}processing steps}
\label{\detokenize{Preprocessing/Ecologicalunits:qgis-data-pre-processing-steps}}\begin{enumerate}
\sphinxsetlistlabels{\arabic}{enumi}{enumii}{}{.}%
\item {} 
\sphinxAtStartPar
Open the land\sphinxhyphen{}cover and soil grids data on qgis

\end{enumerate}

\begin{figure}[H]
\centering
\capstart

\noindent\sphinxincludegraphics[width=800\sphinxpxdimen,height=635\sphinxpxdimen]{{ecounit1}.png}
\caption{Loading the Landcover and Soil Grids on Qgis}\label{\detokenize{Preprocessing/Ecologicalunits:id1}}\end{figure}
\begin{enumerate}
\sphinxsetlistlabels{\arabic}{enumi}{enumii}{}{.}%
\setcounter{enumi}{1}
\item {} 
\sphinxAtStartPar
On the processing tool box, search for the \sphinxstyleemphasis{EcologicalUnitsModel} which will appear under \sphinxstylestrong{Models}\textgreater{}**Land Productivity Models**

\end{enumerate}

\begin{figure}[H]
\centering
\capstart

\noindent\sphinxincludegraphics[width=361\sphinxpxdimen,height=279\sphinxpxdimen]{{ecounit2a}.png}
\caption{Finding the Ecological Unit Model}\label{\detokenize{Preprocessing/Ecologicalunits:id2}}\end{figure}
\begin{enumerate}
\sphinxsetlistlabels{\arabic}{enumi}{enumii}{}{.}%
\setcounter{enumi}{2}
\item {} 
\sphinxAtStartPar
On the dialog that pops up, select the Landcover and the soil grid data as the input data and run the model

\end{enumerate}

\begin{figure}[H]
\centering
\capstart

\noindent\sphinxincludegraphics[width=709\sphinxpxdimen,height=580\sphinxpxdimen]{{ecounit2}.png}
\caption{Ecological Unit Model dialog}\label{\detokenize{Preprocessing/Ecologicalunits:id3}}\end{figure}

\begin{sphinxadmonition}{note}{Note:}\begin{quote}

\begin{figure}[H]
\centering
\capstart

\noindent\sphinxincludegraphics[width=743\sphinxpxdimen,height=242\sphinxpxdimen]{{ecounit3}.png}
\caption{Ecological Unit Model summary}\label{\detokenize{Preprocessing/Ecologicalunits:id4}}\end{figure}
\end{quote}

\sphinxAtStartPar
It is important that the dimenssions of the Landcover data and the soil grids data are matching. This can be checked by right clicking on the layer and navigating to \sphinxstyleemphasis{properties}\textgreater{}*Information*.

\begin{figure}[H]
\centering
\capstart

\noindent\sphinxincludegraphics[width=754\sphinxpxdimen,height=592\sphinxpxdimen]{{ecounit4}.png}
\caption{Checking the dimenssions of the datasets}\label{\detokenize{Preprocessing/Ecologicalunits:id5}}\end{figure}

\sphinxAtStartPar
To match the dimensions of the two layers you can resample the soil grids to match the resolution of the Landcover data.
\end{sphinxadmonition}
\begin{enumerate}
\sphinxsetlistlabels{\arabic}{enumi}{enumii}{}{.}%
\setcounter{enumi}{3}
\item {} 
\sphinxAtStartPar
On runing the model successfully the units layer will be loaded onto Qgis as shown below

\end{enumerate}

\begin{figure}[H]
\centering
\capstart

\noindent\sphinxincludegraphics[width=750\sphinxpxdimen,height=551\sphinxpxdimen]{{ecounit3b}.png}
\caption{Ecological Unit Model output}\label{\detokenize{Preprocessing/Ecologicalunits:id6}}\end{figure}

\sphinxAtStartPar
You can save the output from the Model to your desired location with the appropriate name

\begin{figure}[H]
\centering
\capstart

\noindent\sphinxincludegraphics[width=732\sphinxpxdimen,height=612\sphinxpxdimen]{{ecounit5}.png}
\caption{Saving the output}\label{\detokenize{Preprocessing/Ecologicalunits:id7}}\end{figure}


\section{Uploading the Ecological Units to MISLAND Service}
\label{\detokenize{Preprocessing/Ecologicalunits:uploading-the-ecological-units-to-misland-service}}
\begin{figure}[H]
\centering
\capstart

\noindent\sphinxincludegraphics[width=716\sphinxpxdimen,height=608\sphinxpxdimen]{{ecounit7}.png}
\caption{Data Upload Form}\label{\detokenize{Preprocessing/Ecologicalunits:id8}}\end{figure}

\sphinxstepscope


\chapter{Climate Quality Index}
\label{\detokenize{Preprocessing/cqi:climate-quality-index}}\label{\detokenize{Preprocessing/cqi::doc}}
\sphinxAtStartPar
Climate quality is assessed on the basis of how it influences water availability to the plants. Consideration has been given to the amount of rainfall, air temperature and aridity. Climate layers and relative scores are reported in Table 3. In particular the selected layers are: Annual precipitation (a crucial parameter in plant growth); Bagnouls\sphinxhyphen{}Gaussen aridity index (a synthesis of precipitation, evapotranspiration and run\sphinxhyphen{}off information); Slope aspect (affects microclimatic conditions and erosion).


\section{Data Preprocessing in Qgis}
\label{\detokenize{Preprocessing/cqi:data-preprocessing-in-qgis}}\begin{enumerate}
\sphinxsetlistlabels{\arabic}{enumi}{enumii}{}{.}%
\item {} 
\sphinxAtStartPar
Open the downloaded Potential Evapotranspiration and Precipitation Datasets on Qgis.

\end{enumerate}

\begin{sphinxadmonition}{note}{Note:}
\sphinxAtStartPar
The two datasets are in \sphinxstylestrong{NetCDF} (Network Common Data Form) \sphinxhyphen{} a file format for storing multidimensional scientific data (variables) such as temperature, humidity, pressure, wind speed, and direction. Each of these variables can be displayed through a dimension (such as time) by making a layer or table view from the netCDF file.
When loading the precipitation data select the \sphinxstylestrong{ppt} option on the \sphinxstyleemphasis{Select Raster Layers to Add} dialog that pops up.

\begin{figure}[H]
\centering
\capstart

\noindent\sphinxincludegraphics[width=588\sphinxpxdimen,height=453\sphinxpxdimen]{{cqi2a}.png}
\caption{Selecting the Precipitation data to add to Qgis}\label{\detokenize{Preprocessing/cqi:id1}}\end{figure}
\end{sphinxadmonition}

\sphinxAtStartPar
The loaded layers, alongside the OSS North Africa action zone shapefile should appear on the layers panel on Qgis as shown below

\begin{figure}[H]
\centering
\capstart

\noindent\sphinxincludegraphics[width=700\sphinxpxdimen,height=450\sphinxpxdimen]{{cqi2}.png}
\caption{Loading the PPT and PET NetCDF files to Qgis}\label{\detokenize{Preprocessing/cqi:id2}}\end{figure}
\begin{enumerate}
\sphinxsetlistlabels{\arabic}{enumi}{enumii}{}{.}%
\setcounter{enumi}{1}
\item {} 
\sphinxAtStartPar
Clip the layers to the desires study area by navigating to \sphinxstyleemphasis{Raster}\textgreater{}*Extraction*\textgreater{}*Clip Raster by Mask Layer* option on the Qgis Menu bar.

\end{enumerate}

\begin{figure}[H]
\centering
\capstart

\noindent\sphinxincludegraphics[width=526\sphinxpxdimen,height=338\sphinxpxdimen]{{cqi1a}.png}
\caption{Clipping by Mask Layer}\label{\detokenize{Preprocessing/cqi:id3}}\end{figure}

\sphinxAtStartPar
On the dialoge that pops up select the inputs as shown and specify the \sphinxstylestrong{Source CRS} and \sphinxstylestrong{target CRS} options as shown below. Save the outputs to your desired location before running the tool.

\begin{figure}[H]
\centering
\capstart

\noindent\sphinxincludegraphics[width=782\sphinxpxdimen,height=650\sphinxpxdimen]{{cqi1}.png}
\caption{Clipping by Mask Layer}\label{\detokenize{Preprocessing/cqi:id4}}\end{figure}
\begin{enumerate}
\sphinxsetlistlabels{\arabic}{enumi}{enumii}{}{.}%
\setcounter{enumi}{2}
\item {} 
\sphinxAtStartPar
Once the layers are successfully cliped and saved, open the Processing toolbox and type “Climate Quality Index” and select the \sphinxstyleemphasis{Climate Quality Index} moded under \sphinxstylestrong{Models}

\end{enumerate}

\begin{figure}[H]
\centering
\capstart

\noindent\sphinxincludegraphics[width=409\sphinxpxdimen,height=281\sphinxpxdimen]{{cqi3}.png}
\caption{Climate Quality Index model}\label{\detokenize{Preprocessing/cqi:id5}}\end{figure}

\sphinxAtStartPar
Select the proper inputs for the \sphinxstyleemphasis{Potential Evapotranspiration} and the \sphinxstyleemphasis{Precipitation}

\begin{figure}[H]
\centering
\capstart

\noindent\sphinxincludegraphics[width=742\sphinxpxdimen,height=644\sphinxpxdimen]{{cqi4}.png}
\caption{Defining Inputs for the CQI model}\label{\detokenize{Preprocessing/cqi:id6}}\end{figure}

\begin{sphinxadmonition}{note}{Note:}
\sphinxAtStartPar
The CQI Model yeilds the Precipitation Index by computing the total annual precipitation and reclassifying the values. The Aridity Index is computed as the ration of the Mean annual precipitation and the potential evapotranspiration using the simple Penman\sphinxhyphen{}Monteith formulae. The process is as summarized in the graphical representation of the model shown below.
\begin{quote}

\begin{figure}[H]
\centering
\capstart

\noindent\sphinxincludegraphics[width=800\sphinxpxdimen,height=300\sphinxpxdimen]{{cqi5}.png}
\caption{CQI Model graphical model}\label{\detokenize{Preprocessing/cqi:id7}}\end{figure}

\sphinxAtStartPar
The classes for the precipitation and aridity index are obtained from: Ferrara, A*., Kosmas, C., Salvati, L., Padula, A., Mancino, G., \& Nolè, A. (2020). Updating the MEDALUS‐ESA Framework for Worldwide Land Degradation and Desertification Assessment. \sphinxstyleemphasis{Land Degradation \& Development}, 31(12), 1593\sphinxhyphen{}1607.

\begin{figure}[H]
\centering
\capstart

\noindent\sphinxincludegraphics[width=544\sphinxpxdimen,height=208\sphinxpxdimen]{{cqi9}.png}
\caption{Aridity Index and Precipitaiton index scores}\label{\detokenize{Preprocessing/cqi:id8}}\end{figure}
\end{quote}
\end{sphinxadmonition}
\begin{enumerate}
\sphinxsetlistlabels{\arabic}{enumi}{enumii}{}{.}%
\setcounter{enumi}{3}
\item {} 
\sphinxAtStartPar
Once the model has been executed successfully, The ouputs will be loaded onto Qgis. Right click on the temporary layer and navigate to the \sphinxstyleemphasis{Save as} option to to export the layers with the desired Name, CRS and dimensions as shown below;

\end{enumerate}

\begin{figure}[H]
\centering
\capstart

\noindent\sphinxincludegraphics[width=800\sphinxpxdimen,height=681\sphinxpxdimen]{{cqi6}.png}
\caption{CQI Model graphical model outputs}\label{\detokenize{Preprocessing/cqi:id9}}\end{figure}

\begin{figure}[H]
\centering
\capstart

\noindent\sphinxincludegraphics[width=777\sphinxpxdimen,height=375\sphinxpxdimen]{{cqi7}.png}
\caption{Saving the outputs}\label{\detokenize{Preprocessing/cqi:id10}}\end{figure}

\begin{sphinxadmonition}{note}{Note:}
\sphinxAtStartPar
On the horizontal and vertical resolution setings in the \sphinxstyleemphasis{Save as} dialog paset the value 0.0027778 for each of the outpus to give the layers a resolution 0f 300m to match other variables for computing the desertification indicators

\begin{figure}[H]
\centering
\capstart

\noindent\sphinxincludegraphics[width=695\sphinxpxdimen,height=609\sphinxpxdimen]{{cqi10}.png}
\caption{Setting the resolution of the output layers layers}\label{\detokenize{Preprocessing/cqi:id11}}\end{figure}
\end{sphinxadmonition}


\section{Uploading the Arididy Index to MISLAND Service}
\label{\detokenize{Preprocessing/cqi:uploading-the-arididy-index-to-misland-service}}
\begin{sphinxadmonition}{note}{Note:}
\sphinxAtStartPar
It is important that you give the Aridity Index layer a descriptive name and assocciate it with the correct year and raster category as shown below.
\end{sphinxadmonition}

\begin{figure}[H]
\centering
\capstart

\noindent\sphinxincludegraphics[width=663\sphinxpxdimen,height=544\sphinxpxdimen]{{cqi8}.png}
\caption{Aridity index data upload}\label{\detokenize{Preprocessing/cqi:id12}}\end{figure}


\section{Uploading the Rainfall Data to MISLAND Service}
\label{\detokenize{Preprocessing/cqi:uploading-the-rainfall-data-to-misland-service}}
\sphinxstepscope


\chapter{Vegetation Qulality Index}
\label{\detokenize{Preprocessing/vqi:vegetation-qulality-index}}\label{\detokenize{Preprocessing/vqi::doc}}
\sphinxAtStartPar
Vegetation plays a key role in preventing desertification by providing shelter against wind and water erosion. Plant cover promotes infiltration of water and reduces runoff, residual plant materials from senescent vegetation enriches the soil with organic matter which improves its structure and cohesion . The preliminary classification and the assigned scores of the vegetation characteristics is as summarized in the table below:


\begin{savenotes}\sphinxattablestart
\sphinxthistablewithglobalstyle
\centering
\begin{tabulary}{\linewidth}[t]{|T|T|T|T|T|}
\sphinxtoprule
\sphinxstyletheadfamily 
\sphinxAtStartPar
Sensor/Dataset
&\sphinxstyletheadfamily 
\sphinxAtStartPar
Temporal
&\sphinxstyletheadfamily 
\sphinxAtStartPar
Spatial
&\sphinxstyletheadfamily 
\sphinxAtStartPar
Extent
&\sphinxstyletheadfamily 
\sphinxAtStartPar
License
\\
\sphinxmidrule
\sphinxtableatstartofbodyhook
\sphinxAtStartPar
\sphinxhref{https://www.esa-landcover-cci.org}{ESA CCI Land Cover}
&
\sphinxAtStartPar
1992\sphinxhyphen{}2018
&
\sphinxAtStartPar
300 m
&
\sphinxAtStartPar
Global
&
\sphinxAtStartPar
\sphinxhref{https://creativecommons.org/licenses/by-sa/3.0/igo}{CC by\sphinxhyphen{}SA 3.0}
\\
\sphinxhline
\sphinxAtStartPar
\sphinxhref{https://developers.google.com/earth-engine/datasets/catalog/VITO\_PROBAV\_C1\_S1\_TOC\_100M}{PROBAV NDVI}
&
\sphinxAtStartPar
2013\sphinxhyphen{}present
&
\sphinxAtStartPar
100 m
&
\sphinxAtStartPar
Global
&
\sphinxAtStartPar
\sphinxtitleref{Public Domain}
\\
\sphinxbottomrule
\end{tabulary}
\sphinxtableafterendhook\par
\sphinxattableend\end{savenotes}


\section{Compute and download PROBAV maxNDVI form Google Earth Engine}
\label{\detokenize{Preprocessing/vqi:compute-and-download-probav-maxndvi-form-google-earth-engine}}
\sphinxAtStartPar
To compute and download PROBAV maximum NDVI composite from google earth engine. Open the \sphinxhref{https://code.earthengine.google.com/}{Google Earth Engine Code} and paste the lines of code provided below

\begin{sphinxVerbatim}[commandchars=\\\{\},numbers=left,firstnumber=1,stepnumber=1]
\PYG{+w}{ }\PYG{c+c1}{//Import the Proba\PYGZhy{}V data and North Africa region geometry}

\PYG{+w}{ }\PYG{k+kd}{var}\PYG{+w}{ }\PYG{n+nx}{imageCollection}\PYG{+w}{ }\PYG{o}{=}\PYG{+w}{ }\PYG{n+nx}{ee}\PYG{p}{.}\PYG{n+nx}{ImageCollection}\PYG{p}{(}\PYG{l+s+s2}{\PYGZdq{}VITO/PROBAV/C1/S1\PYGZus{}TOC\PYGZus{}100M\PYGZdq{}}\PYG{p}{)}\PYG{p}{,}
\PYG{+w}{ }\PYG{n+nx}{table2}\PYG{+w}{ }\PYG{o}{=}\PYG{+w}{ }\PYG{n+nx}{ee}\PYG{p}{.}\PYG{n+nx}{FeatureCollection}\PYG{p}{(}\PYG{l+s+s2}{\PYGZdq{}users/derickongeri/NorthAfrica\PYGZdq{}}\PYG{p}{)}\PYG{p}{;}

\PYG{+w}{ }\PYG{c+c1}{//filet the image collection by year, compute the maximum NDVI, and clip to the study area}
\PYG{+w}{ }\PYG{k+kd}{var}\PYG{+w}{ }\PYG{n+nx}{filteredCollection}\PYG{+w}{ }\PYG{o}{=}\PYG{+w}{ }\PYG{n+nx}{imageCollection}\PYG{p}{.}\PYG{n+nx}{filterDate}\PYG{p}{(}\PYG{l+s+s1}{\PYGZsq{}2018\PYGZhy{}01\PYGZhy{}01\PYGZsq{}}\PYG{p}{,}\PYG{l+s+s1}{\PYGZsq{}2019\PYGZhy{}01\PYGZhy{}01\PYGZsq{}}\PYG{p}{)}
\PYG{+w}{                                         }\PYG{p}{.}\PYG{n+nx}{max}\PYG{p}{(}\PYG{p}{)}
\PYG{+w}{                                         }\PYG{p}{.}\PYG{n+nx}{clip}\PYG{p}{(}\PYG{n+nx}{table2}\PYG{p}{)}\PYG{p}{;}

\PYG{+w}{ }\PYG{k+kd}{var}\PYG{+w}{ }\PYG{n+nx}{ndviImage}\PYG{+w}{ }\PYG{o}{=}\PYG{+w}{ }\PYG{n+nx}{filteredCollection}\PYG{p}{.}\PYG{n+nx}{select}\PYG{p}{(}\PYG{l+s+s1}{\PYGZsq{}NDVI\PYGZsq{}}\PYG{p}{)}\PYG{p}{;}

\PYG{+w}{ }\PYG{n+nb}{Map}\PYG{p}{.}\PYG{n+nx}{addLayer}\PYG{p}{(}\PYG{n+nx}{ndviImage}\PYG{p}{.}\PYG{n+nx}{randomVisualizer}\PYG{p}{(}\PYG{p}{)}\PYG{p}{)}\PYG{p}{;}

\PYG{+w}{ }\PYG{c+c1}{//Export the data to drive}
\PYG{+w}{ }\PYG{n+nx}{Export}\PYG{p}{.}\PYG{n+nx}{image}\PYG{p}{.}\PYG{n+nx}{toDrive}\PYG{p}{(}\PYG{p}{\PYGZob{}}
\PYG{+w}{   }\PYG{n+nx}{image}\PYG{o}{:}\PYG{+w}{ }\PYG{n+nx}{ndviImage}\PYG{p}{,}
\PYG{+w}{   }\PYG{n+nx}{description}\PYG{o}{:}\PYG{+w}{ }\PYG{l+s+s1}{\PYGZsq{}NDVI\PYGZus{}Max\PYGZus{}2018\PYGZsq{}}\PYG{p}{,}
\PYG{+w}{   }\PYG{n+nx}{maxPixels}\PYG{o}{:}\PYG{+w}{ }\PYG{l+m+mf}{1e13}\PYG{p}{,}
\PYG{+w}{   }\PYG{n+nx}{scale}\PYG{o}{:}\PYG{+w}{ }\PYG{l+m+mf}{100}\PYG{p}{,}
\PYG{+w}{   }\PYG{n+nx}{region}\PYG{o}{:}\PYG{+w}{ }\PYG{n+nx}{table2}
\PYG{+w}{ }\PYG{p}{\PYGZcb{}}\PYG{p}{)}
\end{sphinxVerbatim}


\section{Data Preprocessing in Qgis}
\label{\detokenize{Preprocessing/vqi:data-preprocessing-in-qgis}}\begin{enumerate}
\sphinxsetlistlabels{\arabic}{enumi}{enumii}{}{.}%
\item {} 
\sphinxAtStartPar
Open the cliped landcover data(all 36 classes) and the ProbaV maximum NDVI composite data on Qgis

\end{enumerate}

\begin{figure}[H]
\centering
\capstart

\noindent\sphinxincludegraphics[width=650\sphinxpxdimen,height=430\sphinxpxdimen]{{vqi3}.png}
\caption{Landcover data and NDVI data loaded to Qgis}\label{\detokenize{Preprocessing/vqi:id1}}\end{figure}
\begin{enumerate}
\sphinxsetlistlabels{\arabic}{enumi}{enumii}{}{.}%
\setcounter{enumi}{1}
\item {} 
\sphinxAtStartPar
Once the layers are loaded on to Qgis, open the processing toolbox and search for ‘Vegetation Quality Index’ in the search bar. The vegetatioon quality index model should show up under the \sphinxstylestrong{Models} section as shown. Click on the Model to open it.

\end{enumerate}

\begin{figure}[H]
\centering
\capstart

\noindent\sphinxincludegraphics[width=334\sphinxpxdimen,height=221\sphinxpxdimen]{{vqi3a}.png}
\caption{Vegetation Quality Index processing Model}\label{\detokenize{Preprocessing/vqi:id2}}\end{figure}
\begin{enumerate}
\sphinxsetlistlabels{\arabic}{enumi}{enumii}{}{.}%
\setcounter{enumi}{2}
\item {} 
\sphinxAtStartPar
On the Vegetation Quality Index dialog that pops up, select the loaded landcover and NDVI data as inputs and run the model

\end{enumerate}

\begin{figure}[H]
\centering
\capstart

\noindent\sphinxincludegraphics[width=800\sphinxpxdimen,height=600\sphinxpxdimen]{{vqi4}.png}
\caption{Vegetation Qulity Index inputs}\label{\detokenize{Preprocessing/vqi:id3}}\end{figure}

\begin{sphinxadmonition}{note}{Note:}
\sphinxAtStartPar
The vegetation Quality Index model Reclassifies the landcover and assigns scores to the landcover groups for the Fire Risk, Erosion Protection and the Drought resistance. The plant cover is derived from the Maximum NDVI composite as summarized in the graphical model below.

\begin{figure}[H]
\centering
\capstart

\noindent\sphinxincludegraphics[width=700\sphinxpxdimen,height=400\sphinxpxdimen]{{vqi4a}.png}
\caption{Vegetation Qulity Index inputs}\label{\detokenize{Preprocessing/vqi:id4}}\end{figure}

\sphinxAtStartPar
The scores to the reclassified landcover outputs and plant cover scores are assigned according to the table below: \sphinxstyleemphasis{Ferrara}, \sphinxstyleemphasis{Agostino}, \sphinxstyleemphasis{et al}. “\sphinxstyleemphasis{Updating the MEDALUS‐ESA Framework for Worldwide Land Degradation and Desertification Assessment}.” \sphinxstyleemphasis{Land Degradation \& Development} 31.12 (2020): 1593\sphinxhyphen{}1607.

\begin{figure}[H]
\centering
\capstart

\noindent\sphinxincludegraphics[width=800\sphinxpxdimen,height=600\sphinxpxdimen]{{vqi1}.png}
\caption{Vegetation Qulity Index inputs}\label{\detokenize{Preprocessing/vqi:id5}}\end{figure}

\begin{figure}[H]
\centering
\capstart

\noindent\sphinxincludegraphics[width=216\sphinxpxdimen,height=182\sphinxpxdimen]{{vqi2}.png}
\caption{Plant cover scores}\label{\detokenize{Preprocessing/vqi:id6}}\end{figure}
\end{sphinxadmonition}
\begin{enumerate}
\sphinxsetlistlabels{\arabic}{enumi}{enumii}{}{.}%
\setcounter{enumi}{3}
\item {} 
\sphinxAtStartPar
On running the model the ouputs for the elementary VQI variables should be loaded onto QGIS as temporary layers. Save the layers to your desired folder with the appropriate descriptive name.

\end{enumerate}

\begin{figure}[H]
\centering
\capstart

\noindent\sphinxincludegraphics[width=800\sphinxpxdimen,height=450\sphinxpxdimen]{{vqi5}.png}
\caption{Vegetation Qulity Index model outputs}\label{\detokenize{Preprocessing/vqi:id7}}\end{figure}

\begin{sphinxadmonition}{note}{Note:}
\sphinxAtStartPar
To save the layers with the appropriate dimensions, right click on the layer you want to save and navigate to \sphinxstyleemphasis{Export}\textgreater{}*Save as* and on the \sphinxstyleemphasis{Save as} dialog set the appropriate name and location for the output. Make sure to set the horizontal and vertical resolution option to 0.00277778 for all the outputs as shown below.
\end{sphinxadmonition}

\begin{figure}[H]
\centering
\capstart

\noindent\sphinxincludegraphics[width=839\sphinxpxdimen,height=657\sphinxpxdimen]{{vqi6}.png}
\caption{Vegetation Qulity Index model outputs}\label{\detokenize{Preprocessing/vqi:id8}}\end{figure}


\section{Data Upload to MISLAND service}
\label{\detokenize{Preprocessing/vqi:data-upload-to-misland-service}}
\sphinxstepscope


\chapter{Soil Quality Index}
\label{\detokenize{Preprocessing/sqi:soil-quality-index}}\label{\detokenize{Preprocessing/sqi::doc}}
\sphinxstepscope


\chapter{Management Qulity Index}
\label{\detokenize{Preprocessing/mqi:management-qulity-index}}\label{\detokenize{Preprocessing/mqi::doc}}

\section{Data Preprocessing in Qgis}
\label{\detokenize{Preprocessing/mqi:data-preprocessing-in-qgis}}\begin{enumerate}
\sphinxsetlistlabels{\arabic}{enumi}{enumii}{}{.}%
\item {} 
\sphinxAtStartPar
Open the downloaded country population density data on Qgis as shown in the figure so as to merge them

\end{enumerate}

\begin{figure}[H]
\centering
\capstart

\noindent\sphinxincludegraphics[width=721\sphinxpxdimen,height=400\sphinxpxdimen]{{mqi1}.png}
\caption{Opening Land cover and population data on Qgis}\label{\detokenize{Preprocessing/mqi:id1}}\end{figure}
\begin{enumerate}
\sphinxsetlistlabels{\arabic}{enumi}{enumii}{}{.}%
\setcounter{enumi}{1}
\item {} 
\sphinxAtStartPar
On the Qgis Menu\sphinxhyphen{}bar, navigate to \sphinxstyleemphasis{Raster\textgreater{}Miscelleneous\textgreater{}Merge} as shown below:

\end{enumerate}

\begin{figure}[H]
\centering
\capstart

\noindent\sphinxincludegraphics[width=594\sphinxpxdimen,height=411\sphinxpxdimen]{{mqi2}.png}
\caption{Merging Population Dataset}\label{\detokenize{Preprocessing/mqi:id2}}\end{figure}
\begin{enumerate}
\sphinxsetlistlabels{\arabic}{enumi}{enumii}{}{.}%
\setcounter{enumi}{2}
\item {} 
\sphinxAtStartPar
On the dialog that pops\sphinxhyphen{}up, click on the \sphinxstyleemphasis{Input Layer} selection option and choose \sphinxstyleemphasis{Select All}

\end{enumerate}

\begin{figure}[H]
\centering
\capstart

\noindent\sphinxincludegraphics[width=883\sphinxpxdimen,height=500\sphinxpxdimen]{{mqi3a}.png}
\caption{Merge layers dialog}\label{\detokenize{Preprocessing/mqi:id3}}\end{figure}

\begin{figure}[H]
\centering
\capstart

\noindent\sphinxincludegraphics[width=883\sphinxpxdimen,height=691\sphinxpxdimen]{{mqi3}.png}
\caption{Selecting population datasets to merge}\label{\detokenize{Preprocessing/mqi:id4}}\end{figure}

\sphinxAtStartPar
Save the merged layer to your desired location.
\begin{enumerate}
\sphinxsetlistlabels{\arabic}{enumi}{enumii}{}{.}%
\setcounter{enumi}{3}
\item {} 
\sphinxAtStartPar
With the output from step 3 above loaded onto Qgis, Load the landcover data for the same year as the population data

\end{enumerate}

\begin{figure}[H]
\centering
\capstart

\noindent\sphinxincludegraphics[width=710\sphinxpxdimen,height=400\sphinxpxdimen]{{mqi4}.png}
\caption{Load the Landcover data on Qgis}\label{\detokenize{Preprocessing/mqi:id5}}\end{figure}
\begin{enumerate}
\sphinxsetlistlabels{\arabic}{enumi}{enumii}{}{.}%
\setcounter{enumi}{4}
\item {} 
\sphinxAtStartPar
Once the layers are loaded on to Qgis, open the processing toolbox and search for ‘Management Quality Index’ in the search bar. The management quality index model shoul show up under your the \sphinxstylestrong{Models} section as shown. Click on the Model to open it.

\end{enumerate}

\begin{figure}[H]
\centering
\capstart

\noindent\sphinxincludegraphics[width=411\sphinxpxdimen,height=340\sphinxpxdimen]{{mqi5}.png}
\caption{Accessing the MQI model from the processing toolbox}\label{\detokenize{Preprocessing/mqi:id6}}\end{figure}
\begin{enumerate}
\sphinxsetlistlabels{\arabic}{enumi}{enumii}{}{.}%
\setcounter{enumi}{5}
\item {} 
\sphinxAtStartPar
Select the Landcover and Population dataset as your model inputs on the dialog that pops up as shown below:

\end{enumerate}

\begin{figure}[H]
\centering
\capstart

\noindent\sphinxincludegraphics[width=765\sphinxpxdimen,height=621\sphinxpxdimen]{{mqi6}.png}
\caption{Selecting Imput parameters for the MQI model}\label{\detokenize{Preprocessing/mqi:id7}}\end{figure}

\begin{sphinxadmonition}{note}{Note:}
\sphinxAtStartPar
According to the major land use types for assessing the management quality degree of human induced stress, the land use intensity is obtained by defining the type of land use in a certain piece of land using the land cover. The population density was used as a proxy of human pressure on the environment: Ferrara, A*., Kosmas, C., Salvati, L., Padula, A., Mancino, G., \& Nolè, A. (2020). Updating the MEDALUS‐ESA Framework for Worldwide Land Degradation and Desertification Assessment. \sphinxstyleemphasis{Land Degradation \& Development}, 31(12), 1593\sphinxhyphen{}1607.

\sphinxAtStartPar
The simplified methodology is as represnted in the workflow below:
\begin{quote}

\begin{figure}[H]
\centering
\capstart

\noindent\sphinxincludegraphics[width=856\sphinxpxdimen,height=320\sphinxpxdimen]{{mqi7}.png}
\caption{Management Quality Index Model summary}\label{\detokenize{Preprocessing/mqi:id8}}\end{figure}

\begin{figure}[H]
\centering
\capstart

\noindent\sphinxincludegraphics[width=490\sphinxpxdimen,height=520\sphinxpxdimen]{{mqi10}.png}
\caption{Land use intensity score}\label{\detokenize{Preprocessing/mqi:id9}}\end{figure}

\begin{figure}[H]
\centering
\capstart

\noindent\sphinxincludegraphics[width=242\sphinxpxdimen,height=238\sphinxpxdimen]{{mqi11}.png}
\caption{Population density score}\label{\detokenize{Preprocessing/mqi:id10}}\end{figure}
\end{quote}
\end{sphinxadmonition}
\begin{enumerate}
\sphinxsetlistlabels{\arabic}{enumi}{enumii}{}{.}%
\setcounter{enumi}{5}
\item {} 
\sphinxAtStartPar
Once the model is executed successfully the outputs will be loaded as temporary layers. You can save the layers with the desired name and set the horizontal and vertical resolution option to 0.0027778

\end{enumerate}

\begin{figure}[H]
\centering
\capstart

\noindent\sphinxincludegraphics[width=715\sphinxpxdimen,height=400\sphinxpxdimen]{{mqi8}.png}
\caption{Management Quality Index model outputs}\label{\detokenize{Preprocessing/mqi:id11}}\end{figure}

\begin{figure}[H]
\centering
\capstart

\noindent\sphinxincludegraphics[width=838\sphinxpxdimen,height=633\sphinxpxdimen]{{mqi9}.png}
\caption{Saving the Outputs of the MQI model}\label{\detokenize{Preprocessing/mqi:id12}}\end{figure}


\section{Uploading the data to MISLAND Service}
\label{\detokenize{Preprocessing/mqi:uploading-the-data-to-misland-service}}
\sphinxstepscope


\chapter{Soil Organic Carbon}
\label{\detokenize{Preprocessing/soc:soil-organic-carbon}}\label{\detokenize{Preprocessing/soc::doc}}
\sphinxstepscope


\chapter{Coastal Vulnerability Index}
\label{\detokenize{Preprocessing/CVI:coastal-vulnerability-index}}\label{\detokenize{Preprocessing/CVI::doc}}
\sphinxAtStartPar
The index integrates six physical variables in a quantifiable way to reflect the coast’s relative vulnerability to physical changes caused by sea\sphinxhyphen{}level rise. This method produces numerical data that cannot be immediately correlated to specific physical effects. However, it does emphasize the areas where the various effects of sea\sphinxhyphen{}level rise may be the most severe.

\begin{figure}[H]
\centering
\capstart

\noindent\sphinxincludegraphics[width=800\sphinxpxdimen,height=400\sphinxpxdimen]{{ranking}.png}
\caption{Ranking of the coastal vulnerability index variables}\label{\detokenize{Preprocessing/CVI:id2}}\end{figure}

\sphinxAtStartPar
Once each section of coastline is assigned a risk value for each specific data variable, the coastal vulnerability index is calculated as the square root of the geometric mean of these values divided by the total number of variables.


\section{Data Download}
\label{\detokenize{Preprocessing/CVI:data-download}}
\sphinxAtStartPar
The data was downloaded from different platforms which included the Google Earth Engine platform, Global lithological dataset and the Ecological coastal units site based on the variable in question.


\subsection{Data Processing Steps}
\label{\detokenize{Preprocessing/CVI:data-processing-steps}}

\subsection{Geomorphology}
\label{\detokenize{Preprocessing/CVI:geomorphology}}
\sphinxAtStartPar
The Geomophological dataset was downloaded from the global lithological portal given below;
\sphinxhref{https://data.mendeley.com/datasets/y8g64m2v52/1}{Global lithological dataset.}

\phantomsection\label{\detokenize{Preprocessing/CVI:datapreparationoverview}}
\sphinxAtStartPar
It was then subset to the generated study area map of the North Africa region.
On the layer styling panel, symbolize the data into 5 classes using the discrete interpolation and then use the interpolation output to classify the data into five classes to give them the degree of erosivity using the qgis reclassify by table tool as shown the the figure below;

\begin{figure}[H]
\centering
\capstart

\noindent\sphinxincludegraphics{{geomStyling}.png}
\caption{Layer Styling}\label{\detokenize{Preprocessing/CVI:id3}}\end{figure}

\begin{figure}[H]
\centering
\capstart

\noindent\sphinxincludegraphics[width=800\sphinxpxdimen,height=400\sphinxpxdimen]{{GEOM}.png}
\caption{Geomophology reclassify by table}\label{\detokenize{Preprocessing/CVI:id4}}\end{figure}

\sphinxAtStartPar
The result of the re\sphinxhyphen{}classification should be 5 classes with a risk value of 5 classes whereby value 1 is very low and value 5 is very high.

\begin{figure}[H]
\centering
\capstart

\noindent\sphinxincludegraphics[scale=0.75]{{classes}.png}
\caption{Geomophology Erosivity Classes}\label{\detokenize{Preprocessing/CVI:id5}}\end{figure}


\section{Upload to MISLAND Service}
\label{\detokenize{Preprocessing/CVI:upload-to-misland-service}}
\begin{figure}[H]
\centering
\capstart

\noindent\sphinxincludegraphics[width=642\sphinxpxdimen,height=597\sphinxpxdimen]{{geomophologyUpload}.png}
\caption{Uploading the data to MISLAND service}\label{\detokenize{Preprocessing/CVI:id6}}\end{figure}


\subsection{Coastal Slope}
\label{\detokenize{Preprocessing/CVI:coastal-slope}}
\sphinxAtStartPar
Download the elevation raster file from \sphinxhref{https://code.earthengine.google.com/}{Google Earth Engine}
Subset to the North Africa coastal region using the region’s shapefile.
To compute slope, \sphinxstylestrong{Open Qgis \textgreater{} Raster \textgreater{} Analysis \textgreater{} Slope}. The slope computation window opens as shown below:

\begin{figure}[H]
\centering
\capstart

\noindent\sphinxincludegraphics{{slopeCalc}.png}
\caption{Slope Computation}\label{\detokenize{Preprocessing/CVI:id7}}\end{figure}

\sphinxAtStartPar
After a successful slope computation, symbolize the it using the discrete interpolation method in Qgis software and then reclassify it into five classes using the Qgis reclassify by table tool.

\begin{figure}[H]
\centering
\capstart

\noindent\sphinxincludegraphics{{slopeRank}.png}
\caption{Slope ranked classes}\label{\detokenize{Preprocessing/CVI:id8}}\end{figure}

\begin{sphinxadmonition}{note}{Note}

\sphinxAtStartPar
The class ranges should be obtained from the layer styling value label whereby the lowest is assigned value 1 whereas the highest is assigned value 5.
\end{sphinxadmonition}

\begin{figure}[H]
\centering
\capstart

\noindent\sphinxincludegraphics[width=642\sphinxpxdimen,height=597\sphinxpxdimen]{{slopeUpload}.png}
\caption{Uploading the data to MISLAND service}\label{\detokenize{Preprocessing/CVI:id9}}\end{figure}


\subsection{Tidal Range}
\label{\detokenize{Preprocessing/CVI:tidal-range}}\label{\detokenize{Preprocessing/CVI:overview}}
\sphinxAtStartPar
Download the data from the \sphinxhref{https://www.esri.com/arcgis-blog/products/arcgis-living-atlas/mapping/ecus-available/}{Coastal Ecological Units (ECU)}
The data comes as a shapefile with different data attributes of 10 variables.
First convert the data into a raster using the \sphinxstylestrong{Rasterize GDAL tool in Qgis}. Remember to change the \sphinxstylestrong{burn\sphinxhyphen{}in field value} to \sphinxstylestrong{TIDAL RANGE}.
Leave the other options as default and click run.

\begin{figure}[H]
\centering
\capstart

\noindent\sphinxincludegraphics{{RasterizeTidal}.png}
\caption{Tidal Range Raster Conversion}\label{\detokenize{Preprocessing/CVI:id10}}\end{figure}

\sphinxAtStartPar
Next is to reclassify the data and assign the class ranges as it was done in the {\hyperref[\detokenize{Preprocessing/CVI:datapreparationoverview}]{\sphinxcrossref{\DUrole{std,std-ref}{Geomophology data preparation.}}}}

\begin{figure}[H]
\centering
\capstart

\noindent\sphinxincludegraphics[width=642\sphinxpxdimen,height=597\sphinxpxdimen]{{tidalUpload}.png}
\caption{Uploading the data to MISLAND service}\label{\detokenize{Preprocessing/CVI:id11}}\end{figure}


\subsection{Wave Height}
\label{\detokenize{Preprocessing/CVI:wave-height}}
\sphinxAtStartPar
Download the data from \sphinxhref{http://www.globalwavestatisticsonline.com/Help/height\_data.htm}{global wave statistics,} and subset it to the North Africa coast region.
Follow the same procedure as the one used in the {\hyperref[\detokenize{Preprocessing/CVI:datapreparationoverview}]{\sphinxcrossref{\DUrole{std,std-ref}{Geomophology data preparation.}}}} to classify the data and rank into into 5 categorical classes.

\begin{figure}[H]
\centering
\capstart

\noindent\sphinxincludegraphics[width=642\sphinxpxdimen,height=597\sphinxpxdimen]{{waveHeightUpload}.png}
\caption{Uploading the data to MISLAND service}\label{\detokenize{Preprocessing/CVI:id12}}\end{figure}


\subsection{Shoreline Erosion}
\label{\detokenize{Preprocessing/CVI:shoreline-erosion}}
\sphinxAtStartPar
Download the data from the \sphinxhref{https://www.esri.com/arcgis-blog/products/arcgis-living-atlas/mapping/ecus-available/}{Coastal Ecological Units (ECU)} and then follow the procedure as the one given in {\hyperref[\detokenize{Preprocessing/CVI:overview}]{\sphinxcrossref{\DUrole{std,std-ref}{Tidal Range data preparation}}}}.

\begin{figure}[H]
\centering
\capstart

\noindent\sphinxincludegraphics[width=642\sphinxpxdimen,height=597\sphinxpxdimen]{{accreationUpload}.png}
\caption{Uploading the data to MISLAND service}\label{\detokenize{Preprocessing/CVI:id13}}\end{figure}


\subsection{Sea Level Rise}
\label{\detokenize{Preprocessing/CVI:sea-level-rise}}
\begin{figure}[H]
\centering
\capstart

\noindent\sphinxincludegraphics[width=642\sphinxpxdimen,height=597\sphinxpxdimen]{{sealevelchangeUpload}.png}
\caption{Uploading the data to MISLAND service}\label{\detokenize{Preprocessing/CVI:id14}}\end{figure}


\chapter{Indices and tables}
\label{\detokenize{index:indices-and-tables}}\begin{itemize}
\item {} 
\sphinxAtStartPar
\DUrole{xref,std,std-ref}{genindex}

\item {} 
\sphinxAtStartPar
\DUrole{xref,std,std-ref}{modindex}

\item {} 
\sphinxAtStartPar
\DUrole{xref,std,std-ref}{search}

\end{itemize}



\renewcommand{\indexname}{Index}
\printindex
\end{document}